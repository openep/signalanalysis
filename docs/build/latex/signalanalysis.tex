%% Generated by Sphinx.
\def\sphinxdocclass{report}
\documentclass[letterpaper,10pt,english]{sphinxmanual}
\ifdefined\pdfpxdimen
   \let\sphinxpxdimen\pdfpxdimen\else\newdimen\sphinxpxdimen
\fi \sphinxpxdimen=.75bp\relax
\ifdefined\pdfimageresolution
    \pdfimageresolution= \numexpr \dimexpr1in\relax/\sphinxpxdimen\relax
\fi
%% let collapsible pdf bookmarks panel have high depth per default
\PassOptionsToPackage{bookmarksdepth=5}{hyperref}

\PassOptionsToPackage{warn}{textcomp}
\usepackage[utf8]{inputenc}
\ifdefined\DeclareUnicodeCharacter
% support both utf8 and utf8x syntaxes
  \ifdefined\DeclareUnicodeCharacterAsOptional
    \def\sphinxDUC#1{\DeclareUnicodeCharacter{"#1}}
  \else
    \let\sphinxDUC\DeclareUnicodeCharacter
  \fi
  \sphinxDUC{00A0}{\nobreakspace}
  \sphinxDUC{2500}{\sphinxunichar{2500}}
  \sphinxDUC{2502}{\sphinxunichar{2502}}
  \sphinxDUC{2514}{\sphinxunichar{2514}}
  \sphinxDUC{251C}{\sphinxunichar{251C}}
  \sphinxDUC{2572}{\textbackslash}
\fi
\usepackage{cmap}
\usepackage[T1]{fontenc}
\usepackage{amsmath,amssymb,amstext}
\usepackage{babel}



\usepackage{tgtermes}
\usepackage{tgheros}
\renewcommand{\ttdefault}{txtt}



\usepackage[Bjarne]{fncychap}
\usepackage{sphinx}

\fvset{fontsize=auto}
\usepackage{geometry}


% Include hyperref last.
\usepackage{hyperref}
% Fix anchor placement for figures with captions.
\usepackage{hypcap}% it must be loaded after hyperref.
% Set up styles of URL: it should be placed after hyperref.
\urlstyle{same}

\addto\captionsenglish{\renewcommand{\contentsname}{Contents:}}

\usepackage{sphinxmessages}
\setcounter{tocdepth}{1}



\title{signalanalysis}
\date{Oct 19, 2021}
\release{0.1}
\author{Philip Gemmell}
\newcommand{\sphinxlogo}{\vbox{}}
\renewcommand{\releasename}{Release}
\makeindex
\begin{document}

\pagestyle{empty}
\sphinxmaketitle
\pagestyle{plain}
\sphinxtableofcontents
\pagestyle{normal}
\phantomsection\label{\detokenize{index::doc}}


\sphinxAtStartPar
\sphinxstylestrong{signalanalysis} is a library including various tools for the reading, analysis and plotting of ECG and VCG data. It is designed to be as agnostic as possible for the types of data that it can read. Currently, it can read ECG data from:
\begin{enumerate}
\sphinxsetlistlabels{\arabic}{enumi}{enumii}{}{.}%
\item {} 
\sphinxAtStartPar
CARP simulations of whole torso activity, using existing projects from \sphinxtitleref{CARPutil \textless{}https://git.opencarp.org/openCARP/carputils\textgreater{}};

\item {} 
\sphinxAtStartPar
.csv and .dat records;

\item {} 
\sphinxAtStartPar
wfdb file formats, using \sphinxtitleref{wfdb\sphinxhyphen{}python \textless{}https://github.com/MIT\sphinxhyphen{}LCP/wfdb\sphinxhyphen{}python\textgreater{}}

\end{enumerate}

\sphinxAtStartPar
Futher details of how to use these functions are in {\hyperref[\detokenize{usage::doc}]{\sphinxcrossref{\DUrole{doc}{Usage}}}}, with the finer points within the files themselves.


\chapter{Usage}
\label{\detokenize{usage:usage}}\label{\detokenize{usage::doc}}

\section{Installation \& Getting Started}
\label{\detokenize{usage:installation-getting-started}}\label{\detokenize{usage:installation}}
\sphinxAtStartPar
To use signalanalysis, it is highly recommended to install using \sphinxcode{\sphinxupquote{pipenv}}, the virtual environment, to ensure that dependencies are where possible maintained. Clone the repository, then install the requirements as follows:

\begin{sphinxVerbatim}[commandchars=\\\{\}]
\PYG{g+gp}{user@home:\PYGZti{}\PYGZdl{} }git clone git@github.com:philip\PYGZhy{}gemmell/signalanalysis.git
\PYG{g+gp}{user@home:\PYGZti{}\PYGZdl{} }\PYG{n+nb}{cd} signalanalysis
\PYG{g+gp}{user@home:\PYGZti{}/signalanalysis\PYGZdl{} }pipenv install
\end{sphinxVerbatim}

\sphinxAtStartPar
Once the repository is cloned, it is currently the case that all work must be done within the Python3 environment. However, it is recommended to use the virtual environment from pipenv rather than the system\sphinxhyphen{}wide Python3 (after entering a pipenv shell, it is quit using the \sphinxcode{\sphinxupquote{exit}} command as shown)

\begin{sphinxVerbatim}[commandchars=\\\{\}]
\PYG{g+gp}{user@home:\PYGZti{}/signalanalysis\PYGZdl{} }pipenv shell
\PYG{g+gp+gpVirtualEnv}{(signalanalysis)} \PYG{g+gp}{user@home:\PYGZti{}/signalanalysis\PYGZdl{} }python3
\PYG{g+go}{\PYGZgt{}\PYGZgt{}\PYGZgt{} import signalanalysis}
\PYG{g+go}{\PYGZgt{}\PYGZgt{}\PYGZgt{} quit()}
\PYG{g+gp+gpVirtualEnv}{(signalanalysis)} \PYG{g+gp}{user@home:\PYGZti{}/signalanalysis\PYGZdl{} }\PYG{n+nb}{exit}
\PYG{g+gp}{user@home:\PYGZti{}/signalanalysis\PYGZdl{}}
\end{sphinxVerbatim}

\sphinxAtStartPar
The project is arranged into various subdivisions. The required analysis/plotting packages for the ECG/VCG are separated out, and require separate importing. See the next section, and individual help files for further details.
\begin{itemize}
\item {} 
\sphinxAtStartPar
signalanalysis
\begin{itemize}
\item {} 
\sphinxAtStartPar
signalanalysis.general

\item {} 
\sphinxAtStartPar
signalanalysis.ecg

\item {} 
\sphinxAtStartPar
signalanalysis.vcg

\end{itemize}

\item {} 
\sphinxAtStartPar
signalplot

\item {} 
\sphinxAtStartPar
tools

\end{itemize}


\section{Reading ECG/VCG data}
\label{\detokenize{usage:reading-ecg-vcg-data}}\label{\detokenize{usage:reading}}
\sphinxAtStartPar
Currently, only ECG reading is supported; VCG data is calculated from ECG data using the Kors method (see method). ECG files are read upon the instantiation of the ECG class, though it is possible to re\sphinxhyphen{}read data if required for some reason.

\begin{sphinxVerbatim}[commandchars=\\\{\}]
\PYG{g+gp}{\PYGZgt{}\PYGZgt{}\PYGZgt{} }\PYG{k+kn}{import} \PYG{n+nn}{signalanalysis}\PYG{n+nn}{.}\PYG{n+nn}{ecg}
\PYG{g+gp}{\PYGZgt{}\PYGZgt{}\PYGZgt{} }\PYG{k+kn}{import} \PYG{n+nn}{signalanalysis}\PYG{n+nn}{.}\PYG{n+nn}{vcg}
\PYG{g+gp}{\PYGZgt{}\PYGZgt{}\PYGZgt{} }\PYG{n}{ecg\PYGZus{}example} \PYG{o}{=} \PYG{n}{signalanalysis}\PYG{o}{.}\PYG{n}{ecg}\PYG{o}{.}\PYG{n}{Ecg}\PYG{p}{(}\PYG{l+s+s2}{\PYGZdq{}}\PYG{l+s+s2}{filename}\PYG{l+s+s2}{\PYGZdq{}}\PYG{p}{)}
\PYG{g+gp}{\PYGZgt{}\PYGZgt{}\PYGZgt{} }\PYG{n}{vcg\PYGZus{}example} \PYG{o}{=} \PYG{n}{signalanalysis}\PYG{o}{.}\PYG{n}{vcg}\PYG{o}{.}\PYG{n}{Vcg}\PYG{p}{(}\PYG{n}{ecg\PYGZus{}example}\PYG{p}{)}
\end{sphinxVerbatim}


\section{Shifting to classes from methods}
\label{\detokenize{usage:shifting-to-classes-from-methods}}\label{\detokenize{usage:classplan}}
\sphinxAtStartPar
Previously, all ECG/VCG data was extracted and stored in DataFrames, and most of the modules in this code currently support this format. However, it is planned to shift the main focus of the project to use classes, which allow encapsulation of linked data in one data structure. While the focus of this documentation will be future\sphinxhyphen{}facing, and look at using the classes, note that sometimes access to the raw, underlying DataFrames will still be required. To that end, the raw data can be accessed as the .data attribute:

\begin{sphinxVerbatim}[commandchars=\\\{\}]
\PYG{g+gp}{\PYGZgt{}\PYGZgt{}\PYGZgt{} }\PYG{n}{ecg\PYGZus{}class} \PYG{o}{=} \PYG{n}{signalanalysis}\PYG{o}{.}\PYG{n}{ecg}\PYG{o}{.}\PYG{n}{Ecg}\PYG{p}{(}\PYG{l+s+s2}{\PYGZdq{}}\PYG{l+s+s2}{filename}\PYG{l+s+s2}{\PYGZdq{}}\PYG{p}{)} \PYG{c+c1}{\PYGZsh{} Returns an Ecg class}
\PYG{g+gp}{\PYGZgt{}\PYGZgt{}\PYGZgt{} }\PYG{n}{vcg\PYGZus{}class} \PYG{o}{=} \PYG{n}{signalanalysis}\PYG{o}{.}\PYG{n}{vcg}\PYG{o}{.}\PYG{n}{Vcg}\PYG{p}{(}\PYG{n}{ecg\PYGZus{}class}\PYG{p}{)}  \PYG{c+c1}{\PYGZsh{} Returns a Vcg class}
\PYG{g+gp}{\PYGZgt{}\PYGZgt{}\PYGZgt{} }\PYG{n}{ecg\PYGZus{}data} \PYG{o}{=} \PYG{n}{ecg\PYGZus{}class}\PYG{o}{.}\PYG{n}{data}                      \PYG{c+c1}{\PYGZsh{} Returns a Pandas DataFrame of the underlying data}
\PYG{g+gp}{\PYGZgt{}\PYGZgt{}\PYGZgt{} }\PYG{n}{vcg\PYGZus{}data} \PYG{o}{=} \PYG{n}{vcg\PYGZus{}class}\PYG{o}{.}\PYG{n}{data}                      \PYG{c+c1}{\PYGZsh{} Returns a Pandas DataFrame of the underlying data}
\end{sphinxVerbatim}


\chapter{docutils Documentation}
\label{\detokenize{docutils:docutils-documentation}}\label{\detokenize{docutils::doc}}
\sphinxAtStartPar
This is the index of all modules and their associated methods and classes, documenting their use, parameters and return values. See {\hyperref[\detokenize{usage::doc}]{\sphinxcrossref{\DUrole{doc}{Usage}}}} for a more step\sphinxhyphen{}by\sphinxhyphen{}step introduction to the intended use cases.

\sphinxAtStartPar
Broadly speaking, the modules are split thus:
\begin{itemize}
\item {} 
\sphinxAtStartPar
\sphinxcode{\sphinxupquote{signalanalysis}} covers the analysis scripts for ECG/VCG analysis (e.g. calculating QRS duration)

\item {} 
\sphinxAtStartPar
\sphinxcode{\sphinxupquote{signalplot}} covers plotting methods (e.g. plotting the ECG leads on a single figure with annotation, plotting a 3D plot of VCG (including animation!))

\item {} 
\sphinxAtStartPar
\sphinxcode{\sphinxupquote{tools}} covers more general use tools that are not limited to ECG/VCG analysis.

\end{itemize}


\begin{savenotes}\sphinxatlongtablestart\begin{longtable}[c]{\X{1}{2}\X{1}{2}}
\hline

\endfirsthead

\multicolumn{2}{c}%
{\makebox[0pt]{\sphinxtablecontinued{\tablename\ \thetable{} \textendash{} continued from previous page}}}\\
\hline

\endhead

\hline
\multicolumn{2}{r}{\makebox[0pt][r]{\sphinxtablecontinued{continues on next page}}}\\
\endfoot

\endlastfoot

\sphinxAtStartPar
{\hyperref[\detokenize{_autosummary/signalanalysis:module-signalanalysis}]{\sphinxcrossref{\sphinxcode{\sphinxupquote{signalanalysis}}}}}
&
\sphinxAtStartPar

\\
\hline
\sphinxAtStartPar
{\hyperref[\detokenize{_autosummary/signalplot:module-signalplot}]{\sphinxcrossref{\sphinxcode{\sphinxupquote{signalplot}}}}}
&
\sphinxAtStartPar

\\
\hline
\sphinxAtStartPar
{\hyperref[\detokenize{_autosummary/tools:module-tools}]{\sphinxcrossref{\sphinxcode{\sphinxupquote{tools}}}}}
&
\sphinxAtStartPar

\\
\hline
\end{longtable}\sphinxatlongtableend\end{savenotes}


\section{signalanalysis}
\label{\detokenize{_autosummary/signalanalysis:module-signalanalysis}}\label{\detokenize{_autosummary/signalanalysis:signalanalysis}}\label{\detokenize{_autosummary/signalanalysis::doc}}\index{module@\spxentry{module}!signalanalysis@\spxentry{signalanalysis}}\index{signalanalysis@\spxentry{signalanalysis}!module@\spxentry{module}}

\begin{savenotes}\sphinxatlongtablestart\begin{longtable}[c]{\X{1}{2}\X{1}{2}}
\hline

\endfirsthead

\multicolumn{2}{c}%
{\makebox[0pt]{\sphinxtablecontinued{\tablename\ \thetable{} \textendash{} continued from previous page}}}\\
\hline

\endhead

\hline
\multicolumn{2}{r}{\makebox[0pt][r]{\sphinxtablecontinued{continues on next page}}}\\
\endfoot

\endlastfoot

\sphinxAtStartPar
{\hyperref[\detokenize{_autosummary/signalanalysis.ecg:module-signalanalysis.ecg}]{\sphinxcrossref{\sphinxcode{\sphinxupquote{signalanalysis.ecg}}}}}
&
\sphinxAtStartPar

\\
\hline
\sphinxAtStartPar
{\hyperref[\detokenize{_autosummary/signalanalysis.general:module-signalanalysis.general}]{\sphinxcrossref{\sphinxcode{\sphinxupquote{signalanalysis.general}}}}}
&
\sphinxAtStartPar

\\
\hline
\sphinxAtStartPar
{\hyperref[\detokenize{_autosummary/signalanalysis.vcg:module-signalanalysis.vcg}]{\sphinxcrossref{\sphinxcode{\sphinxupquote{signalanalysis.vcg}}}}}
&
\sphinxAtStartPar

\\
\hline
\end{longtable}\sphinxatlongtableend\end{savenotes}


\subsection{signalanalysis.ecg}
\label{\detokenize{_autosummary/signalanalysis.ecg:module-signalanalysis.ecg}}\label{\detokenize{_autosummary/signalanalysis.ecg:signalanalysis-ecg}}\label{\detokenize{_autosummary/signalanalysis.ecg::doc}}\index{module@\spxentry{module}!signalanalysis.ecg@\spxentry{signalanalysis.ecg}}\index{signalanalysis.ecg@\spxentry{signalanalysis.ecg}!module@\spxentry{module}}\subsubsection*{Functions}


\begin{savenotes}\sphinxatlongtablestart\begin{longtable}[c]{\X{1}{2}\X{1}{2}}
\hline

\endfirsthead

\multicolumn{2}{c}%
{\makebox[0pt]{\sphinxtablecontinued{\tablename\ \thetable{} \textendash{} continued from previous page}}}\\
\hline

\endhead

\hline
\multicolumn{2}{r}{\makebox[0pt][r]{\sphinxtablecontinued{continues on next page}}}\\
\endfoot

\endlastfoot

\sphinxAtStartPar
{\hyperref[\detokenize{_autosummary/signalanalysis.ecg.get_ecg_from_electrodes:signalanalysis.ecg.get_ecg_from_electrodes}]{\sphinxcrossref{\sphinxcode{\sphinxupquote{get\_ecg\_from\_electrodes}}}}}
&
\sphinxAtStartPar
Converts electrode phi\_e data to ECG lead data
\\
\hline
\sphinxAtStartPar
{\hyperref[\detokenize{_autosummary/signalanalysis.ecg.get_electrode_phie:signalanalysis.ecg.get_electrode_phie}]{\sphinxcrossref{\sphinxcode{\sphinxupquote{get\_electrode\_phie}}}}}
&
\sphinxAtStartPar
Extract phi\_e data corresponding to ECG electrode locations
\\
\hline
\sphinxAtStartPar
{\hyperref[\detokenize{_autosummary/signalanalysis.ecg.get_qrs_start:signalanalysis.ecg.get_qrs_start}]{\sphinxcrossref{\sphinxcode{\sphinxupquote{get\_qrs\_start}}}}}
&
\sphinxAtStartPar
Calculates start of QRS complex using method of Hermans et al. (2017).
\\
\hline
\sphinxAtStartPar
{\hyperref[\detokenize{_autosummary/signalanalysis.ecg.read_ecg_from_csv:signalanalysis.ecg.read_ecg_from_csv}]{\sphinxcrossref{\sphinxcode{\sphinxupquote{read\_ecg\_from\_csv}}}}}
&
\sphinxAtStartPar
Extract ECG data from CSV file exported from St Jude Medical ECG recording
\\
\hline
\sphinxAtStartPar
{\hyperref[\detokenize{_autosummary/signalanalysis.ecg.read_ecg_from_dat:signalanalysis.ecg.read_ecg_from_dat}]{\sphinxcrossref{\sphinxcode{\sphinxupquote{read\_ecg\_from\_dat}}}}}
&
\sphinxAtStartPar
Read ECG data from .dat file
\\
\hline
\sphinxAtStartPar
{\hyperref[\detokenize{_autosummary/signalanalysis.ecg.read_ecg_from_igb:signalanalysis.ecg.read_ecg_from_igb}]{\sphinxcrossref{\sphinxcode{\sphinxupquote{read\_ecg\_from\_igb}}}}}
&
\sphinxAtStartPar
Translate the phie.igb file(s) to 10\sphinxhyphen{}lead, 12\sphinxhyphen{}trace ECG data
\\
\hline
\end{longtable}\sphinxatlongtableend\end{savenotes}


\subsubsection{signalanalysis.ecg.get\_ecg\_from\_electrodes}
\label{\detokenize{_autosummary/signalanalysis.ecg.get_ecg_from_electrodes:signalanalysis-ecg-get-ecg-from-electrodes}}\label{\detokenize{_autosummary/signalanalysis.ecg.get_ecg_from_electrodes::doc}}\index{get\_ecg\_from\_electrodes() (in module signalanalysis.ecg)@\spxentry{get\_ecg\_from\_electrodes()}\spxextra{in module signalanalysis.ecg}}

\begin{fulllineitems}
\phantomsection\label{\detokenize{_autosummary/signalanalysis.ecg.get_ecg_from_electrodes:signalanalysis.ecg.get_ecg_from_electrodes}}\pysiglinewithargsret{\sphinxcode{\sphinxupquote{signalanalysis.ecg.}}\sphinxbfcode{\sphinxupquote{get\_ecg\_from\_electrodes}}}{\emph{\DUrole{n}{electrode\_data}\DUrole{p}{:} \DUrole{n}{pandas.core.frame.DataFrame}}}{{ $\rightarrow$ pandas.core.frame.DataFrame}}
\sphinxAtStartPar
Converts electrode phi\_e data to ECG lead data

\sphinxAtStartPar
Takes dictionary of phi\_e data for 10\sphinxhyphen{}lead ECG, and converts these data to standard ECG trace data
\begin{quote}\begin{description}
\item[{Parameters}] \leavevmode
\sphinxAtStartPar
\sphinxstyleliteralstrong{\sphinxupquote{electrode\_data}} (\sphinxstyleliteralemphasis{\sphinxupquote{pd.DataFrame}}) \textendash{} Dictionary with keys corresponding to lead locations

\item[{Returns}] \leavevmode
\sphinxAtStartPar
\sphinxstylestrong{ecg} \textendash{} Dictionary with keys corresponding to the ECG traces

\item[{Return type}] \leavevmode
\sphinxAtStartPar
pd.DataFrame

\end{description}\end{quote}

\end{fulllineitems}



\subsubsection{signalanalysis.ecg.get\_electrode\_phie}
\label{\detokenize{_autosummary/signalanalysis.ecg.get_electrode_phie:signalanalysis-ecg-get-electrode-phie}}\label{\detokenize{_autosummary/signalanalysis.ecg.get_electrode_phie::doc}}\index{get\_electrode\_phie() (in module signalanalysis.ecg)@\spxentry{get\_electrode\_phie()}\spxextra{in module signalanalysis.ecg}}

\begin{fulllineitems}
\phantomsection\label{\detokenize{_autosummary/signalanalysis.ecg.get_electrode_phie:signalanalysis.ecg.get_electrode_phie}}\pysiglinewithargsret{\sphinxcode{\sphinxupquote{signalanalysis.ecg.}}\sphinxbfcode{\sphinxupquote{get\_electrode\_phie}}}{\emph{\DUrole{n}{phie\_data}\DUrole{p}{:} \DUrole{n}{numpy.ndarray}}, \emph{\DUrole{n}{electrode\_file}\DUrole{p}{:} \DUrole{n}{str}}}{{ $\rightarrow$ pandas.core.frame.DataFrame}}
\sphinxAtStartPar
Extract phi\_e data corresponding to ECG electrode locations
\begin{quote}\begin{description}
\item[{Parameters}] \leavevmode\begin{itemize}
\item {} 
\sphinxAtStartPar
\sphinxstyleliteralstrong{\sphinxupquote{phie\_data}} (\sphinxstyleliteralemphasis{\sphinxupquote{np.ndarray}}) \textendash{} Numpy array that holds all phie data for all nodes in a given mesh

\item {} 
\sphinxAtStartPar
\sphinxstyleliteralstrong{\sphinxupquote{electrode\_file}} (\sphinxstyleliteralemphasis{\sphinxupquote{str}}) \textendash{} File containing entries corresponding to the nodes of the mesh which determine the location of the 10 leads
for the ECG. Will default to very project specific location. The input text file has each node on a separate
line (zero\sphinxhyphen{}indexed), with the node locations given in order: V1, V2, V3, V4, V5, V6, RA, LA, RL,
LL.

\end{itemize}

\item[{Returns}] \leavevmode
\sphinxAtStartPar
\sphinxstylestrong{electrode\_data} \textendash{} Dataframe of phie data for each node, with the dictionary key labelling which node it is.

\item[{Return type}] \leavevmode
\sphinxAtStartPar
pd.DataFrame

\end{description}\end{quote}
\subsubsection*{Notes}

\sphinxAtStartPar
For the .igb data used thus far, the \sphinxtitleref{electrode\_file} can be found at \sphinxcode{\sphinxupquote{tests/12LeadElectrodes.dat}}

\end{fulllineitems}



\subsubsection{signalanalysis.ecg.get\_qrs\_start}
\label{\detokenize{_autosummary/signalanalysis.ecg.get_qrs_start:signalanalysis-ecg-get-qrs-start}}\label{\detokenize{_autosummary/signalanalysis.ecg.get_qrs_start::doc}}\index{get\_qrs\_start() (in module signalanalysis.ecg)@\spxentry{get\_qrs\_start()}\spxextra{in module signalanalysis.ecg}}

\begin{fulllineitems}
\phantomsection\label{\detokenize{_autosummary/signalanalysis.ecg.get_qrs_start:signalanalysis.ecg.get_qrs_start}}\pysiglinewithargsret{\sphinxcode{\sphinxupquote{signalanalysis.ecg.}}\sphinxbfcode{\sphinxupquote{get\_qrs\_start}}}{\emph{\DUrole{n}{ecgs}\DUrole{p}{:} \DUrole{n}{Union\DUrole{p}{{[}}pandas.core.frame.DataFrame\DUrole{p}{, }List\DUrole{p}{{[}}pandas.core.frame.DataFrame\DUrole{p}{{]}}\DUrole{p}{{]}}}}, \emph{\DUrole{n}{unipolar\_only}\DUrole{p}{:} \DUrole{n}{bool} \DUrole{o}{=} \DUrole{default_value}{True}}, \emph{\DUrole{n}{plot\_result}\DUrole{p}{:} \DUrole{n}{bool} \DUrole{o}{=} \DUrole{default_value}{False}}}{{ $\rightarrow$ List\DUrole{p}{{[}}float\DUrole{p}{{]}}}}
\sphinxAtStartPar
Calculates start of QRS complex using method of Hermans et al. (2017)

\sphinxAtStartPar
Calculates the start of the QRS complex by a simplified version of the work presented in \sphinxstepexplicit %
\begin{footnote}[1]\phantomsection\label{\thesphinxscope.1}%
\sphinxAtStartFootnote
Hermans BJM, Vink AS, Bennis FC, Filippini LH, Meijborg VMF, Wilde AAM, Pison L, Postema PG, Delhaas T,
“The development and validation of an easy to use automatic QT\sphinxhyphen{}interval algorithm,”
PLoS ONE, 12(9), 1\textendash{}14 (2017), \sphinxurl{https://doi.org/10.1371/journal.pone.0184352}
%
\end{footnote}, wherein the point of
maximum second derivative of the ECG RMS signal is used as the start of the QRS complex
\begin{quote}\begin{description}
\item[{Parameters}] \leavevmode\begin{itemize}
\item {} 
\sphinxAtStartPar
\sphinxstyleliteralstrong{\sphinxupquote{ecgs}} (\sphinxstyleliteralemphasis{\sphinxupquote{pd.DataFrame}}\sphinxstyleliteralemphasis{\sphinxupquote{ or }}\sphinxstyleliteralemphasis{\sphinxupquote{list of pd.DataFrame}}) \textendash{} ECG data to analyse

\item {} 
\sphinxAtStartPar
\sphinxstyleliteralstrong{\sphinxupquote{unipolar\_only}} (\sphinxstyleliteralemphasis{\sphinxupquote{bool}}\sphinxstyleliteralemphasis{\sphinxupquote{, }}\sphinxstyleliteralemphasis{\sphinxupquote{optional}}) \textendash{} Whether to use only unipolar leads to calculate RMS, default=True

\item {} 
\sphinxAtStartPar
\sphinxstyleliteralstrong{\sphinxupquote{plot\_result}} (\sphinxstyleliteralemphasis{\sphinxupquote{bool}}\sphinxstyleliteralemphasis{\sphinxupquote{, }}\sphinxstyleliteralemphasis{\sphinxupquote{optional}}) \textendash{} Whether to plot the results for error\sphinxhyphen{}checking, default=False

\end{itemize}

\item[{Returns}] \leavevmode
\sphinxAtStartPar
\sphinxstylestrong{qrs\_starts} \textendash{} QRS start times

\item[{Return type}] \leavevmode
\sphinxAtStartPar
list of float

\end{description}\end{quote}
\subsubsection*{Notes}

\sphinxAtStartPar
For further details of the action of unipolar\_only, see general\_analysis.get\_signal\_rms

\sphinxAtStartPar
It is faster to use scipy.ndimage.laplace() rather than np.gradient(np.gradient)), but preliminary checks
indicated some edge problems that might throw off the results.
\subsubsection*{References}

\end{fulllineitems}



\subsubsection{signalanalysis.ecg.read\_ecg\_from\_csv}
\label{\detokenize{_autosummary/signalanalysis.ecg.read_ecg_from_csv:signalanalysis-ecg-read-ecg-from-csv}}\label{\detokenize{_autosummary/signalanalysis.ecg.read_ecg_from_csv::doc}}\index{read\_ecg\_from\_csv() (in module signalanalysis.ecg)@\spxentry{read\_ecg\_from\_csv()}\spxextra{in module signalanalysis.ecg}}

\begin{fulllineitems}
\phantomsection\label{\detokenize{_autosummary/signalanalysis.ecg.read_ecg_from_csv:signalanalysis.ecg.read_ecg_from_csv}}\pysiglinewithargsret{\sphinxcode{\sphinxupquote{signalanalysis.ecg.}}\sphinxbfcode{\sphinxupquote{read\_ecg\_from\_csv}}}{\emph{\DUrole{n}{filename}\DUrole{p}{:} \DUrole{n}{str}}, \emph{\DUrole{n}{normalise}\DUrole{p}{:} \DUrole{n}{bool} \DUrole{o}{=} \DUrole{default_value}{False}}}{{ $\rightarrow$ pandas.core.frame.DataFrame}}
\sphinxAtStartPar
Extract ECG data from CSV file exported from St Jude Medical ECG recording
\begin{quote}\begin{description}
\item[{Parameters}] \leavevmode\begin{itemize}
\item {} 
\sphinxAtStartPar
\sphinxstyleliteralstrong{\sphinxupquote{filename}} (\sphinxstyleliteralemphasis{\sphinxupquote{str}}) \textendash{} Name/location of the .dat file to read

\item {} 
\sphinxAtStartPar
\sphinxstyleliteralstrong{\sphinxupquote{normalise}} (\sphinxstyleliteralemphasis{\sphinxupquote{bool}}\sphinxstyleliteralemphasis{\sphinxupquote{, }}\sphinxstyleliteralemphasis{\sphinxupquote{optional}}) \textendash{} Whether or not to normalise the ECG signals on a per\sphinxhyphen{}lead basis, default=True

\end{itemize}

\item[{Returns}] \leavevmode
\sphinxAtStartPar
\sphinxstylestrong{ecg} \textendash{} Extracted data for the 12\sphinxhyphen{}lead ECG

\item[{Return type}] \leavevmode
\sphinxAtStartPar
list of pd.DataFrame

\end{description}\end{quote}
\subsubsection*{Notes}

\sphinxAtStartPar
Relies on the first line of the .csv labelling the 12 leads of the ECG, in the form {[}‘I’, ‘II’, ‘III’,  ‘aVR’,
‘aVL’, ‘aVF’, ‘V1’, ‘V2’, ‘V3’, ‘V4’, ‘V5’, ‘V6’{]}, and the time data under the label ‘t\_ref’

\end{fulllineitems}



\subsubsection{signalanalysis.ecg.read\_ecg\_from\_dat}
\label{\detokenize{_autosummary/signalanalysis.ecg.read_ecg_from_dat:signalanalysis-ecg-read-ecg-from-dat}}\label{\detokenize{_autosummary/signalanalysis.ecg.read_ecg_from_dat::doc}}\index{read\_ecg\_from\_dat() (in module signalanalysis.ecg)@\spxentry{read\_ecg\_from\_dat()}\spxextra{in module signalanalysis.ecg}}

\begin{fulllineitems}
\phantomsection\label{\detokenize{_autosummary/signalanalysis.ecg.read_ecg_from_dat:signalanalysis.ecg.read_ecg_from_dat}}\pysiglinewithargsret{\sphinxcode{\sphinxupquote{signalanalysis.ecg.}}\sphinxbfcode{\sphinxupquote{read\_ecg\_from\_dat}}}{\emph{\DUrole{n}{filename}\DUrole{p}{:} \DUrole{n}{str}}, \emph{\DUrole{n}{normalise}\DUrole{p}{:} \DUrole{n}{bool} \DUrole{o}{=} \DUrole{default_value}{False}}}{{ $\rightarrow$ pandas.core.frame.DataFrame}}
\sphinxAtStartPar
Read ECG data from .dat file
\begin{quote}\begin{description}
\item[{Parameters}] \leavevmode\begin{itemize}
\item {} 
\sphinxAtStartPar
\sphinxstyleliteralstrong{\sphinxupquote{filename}} (\sphinxstyleliteralemphasis{\sphinxupquote{str}}) \textendash{} Name/location of the .dat file to read

\item {} 
\sphinxAtStartPar
\sphinxstyleliteralstrong{\sphinxupquote{normalise}} (\sphinxstyleliteralemphasis{\sphinxupquote{bool}}\sphinxstyleliteralemphasis{\sphinxupquote{, }}\sphinxstyleliteralemphasis{\sphinxupquote{optional}}) \textendash{} Whether or not to normalise the ECG signals on a per\sphinxhyphen{}lead basis, default=False

\end{itemize}

\item[{Returns}] \leavevmode
\sphinxAtStartPar
\sphinxstylestrong{ecg} \textendash{} Extracted data for the 12\sphinxhyphen{}lead ECG

\item[{Return type}] \leavevmode
\sphinxAtStartPar
pd.DataFrame

\end{description}\end{quote}

\end{fulllineitems}



\subsubsection{signalanalysis.ecg.read\_ecg\_from\_igb}
\label{\detokenize{_autosummary/signalanalysis.ecg.read_ecg_from_igb:signalanalysis-ecg-read-ecg-from-igb}}\label{\detokenize{_autosummary/signalanalysis.ecg.read_ecg_from_igb::doc}}\index{read\_ecg\_from\_igb() (in module signalanalysis.ecg)@\spxentry{read\_ecg\_from\_igb()}\spxextra{in module signalanalysis.ecg}}

\begin{fulllineitems}
\phantomsection\label{\detokenize{_autosummary/signalanalysis.ecg.read_ecg_from_igb:signalanalysis.ecg.read_ecg_from_igb}}\pysiglinewithargsret{\sphinxcode{\sphinxupquote{signalanalysis.ecg.}}\sphinxbfcode{\sphinxupquote{read\_ecg\_from\_igb}}}{\emph{\DUrole{n}{filename}\DUrole{p}{:} \DUrole{n}{str}}, \emph{\DUrole{n}{electrode\_file}}, \emph{\DUrole{n}{dt}\DUrole{p}{:} \DUrole{n}{float}}, \emph{\DUrole{n}{normalise}\DUrole{p}{:} \DUrole{n}{bool} \DUrole{o}{=} \DUrole{default_value}{False}}}{{ $\rightarrow$ pandas.core.frame.DataFrame}}
\sphinxAtStartPar
Translate the phie.igb file(s) to 10\sphinxhyphen{}lead, 12\sphinxhyphen{}trace ECG data

\sphinxAtStartPar
Extracts the complete mesh data from the phie.igb file using CARPutils, which contains the data for the body
surface potential for an entire human torso, before then extracting only those nodes that are relevant to the
12\sphinxhyphen{}lead ECG, before converting to the ECG itself
\begin{quote}\begin{description}
\item[{Parameters}] \leavevmode\begin{itemize}
\item {} 
\sphinxAtStartPar
\sphinxstyleliteralstrong{\sphinxupquote{filename}} (\sphinxstyleliteralemphasis{\sphinxupquote{str}}) \textendash{} Filename for the phie.igb data to extract

\item {} 
\sphinxAtStartPar
\sphinxstyleliteralstrong{\sphinxupquote{electrode\_file}} (\sphinxstyleliteralemphasis{\sphinxupquote{str}}) \textendash{} File which contains the node indices in the mesh that correspond to the placement of the leads for the
10\sphinxhyphen{}lead ECG. Default given in get\_electrode\_phie function.

\item {} 
\sphinxAtStartPar
\sphinxstyleliteralstrong{\sphinxupquote{dt}} (\sphinxstyleliteralemphasis{\sphinxupquote{float}}) \textendash{} Time interval from which to construct the time data to associate with the ECG, default=0.002s (2ms)

\item {} 
\sphinxAtStartPar
\sphinxstyleliteralstrong{\sphinxupquote{normalise}} (\sphinxstyleliteralemphasis{\sphinxupquote{bool}}\sphinxstyleliteralemphasis{\sphinxupquote{, }}\sphinxstyleliteralemphasis{\sphinxupquote{optional}}) \textendash{} Whether or not to normalise the ECG signals on a per\sphinxhyphen{}lead basis, default=False

\end{itemize}

\item[{Returns}] \leavevmode
\sphinxAtStartPar
\sphinxstylestrong{ecgs} \textendash{} DataFrame with Vm data for each of the labelled leads (the dictionary keys are the names of the leads)

\item[{Return type}] \leavevmode
\sphinxAtStartPar
pd.DataFrame

\end{description}\end{quote}
\subsubsection*{Notes}

\sphinxAtStartPar
For the .igb data used thus far, the \sphinxtitleref{electrode\_file} can be found at \sphinxcode{\sphinxupquote{tests/12LeadElectrodes.dat}}, and \sphinxtitleref{dt} is
0.002s
\subsubsection*{References}

\sphinxAtStartPar
\sphinxurl{https://carpentry.medunigraz.at/carputils/generated/carputils.carpio.igb.IGBFile.html\#carputils.carpio.igb.IGBFile}

\end{fulllineitems}

\subsubsection*{Classes}


\begin{savenotes}\sphinxatlongtablestart\begin{longtable}[c]{\X{1}{2}\X{1}{2}}
\hline

\endfirsthead

\multicolumn{2}{c}%
{\makebox[0pt]{\sphinxtablecontinued{\tablename\ \thetable{} \textendash{} continued from previous page}}}\\
\hline

\endhead

\hline
\multicolumn{2}{r}{\makebox[0pt][r]{\sphinxtablecontinued{continues on next page}}}\\
\endfoot

\endlastfoot

\sphinxAtStartPar
{\hyperref[\detokenize{_autosummary/signalanalysis.ecg.Ecg:signalanalysis.ecg.Ecg}]{\sphinxcrossref{\sphinxcode{\sphinxupquote{Ecg}}}}}
&
\sphinxAtStartPar
Base class to encapsulate data regarding an ECG recording, inheriting from {\hyperref[\detokenize{_autosummary/signalanalysis.general.Signal:signalanalysis.general.Signal}]{\sphinxcrossref{\sphinxcode{\sphinxupquote{signalanalysis.general.Signal}}}}}
\\
\hline
\end{longtable}\sphinxatlongtableend\end{savenotes}


\subsubsection{signalanalysis.ecg.Ecg}
\label{\detokenize{_autosummary/signalanalysis.ecg.Ecg:signalanalysis-ecg-ecg}}\label{\detokenize{_autosummary/signalanalysis.ecg.Ecg::doc}}\index{Ecg (class in signalanalysis.ecg)@\spxentry{Ecg}\spxextra{class in signalanalysis.ecg}}

\begin{fulllineitems}
\phantomsection\label{\detokenize{_autosummary/signalanalysis.ecg.Ecg:signalanalysis.ecg.Ecg}}\pysiglinewithargsret{\sphinxbfcode{\sphinxupquote{class }}\sphinxcode{\sphinxupquote{signalanalysis.ecg.}}\sphinxbfcode{\sphinxupquote{Ecg}}}{\emph{\DUrole{n}{filename}\DUrole{p}{:} \DUrole{n}{str}}, \emph{\DUrole{o}{**}\DUrole{n}{kwargs}}}{}
\sphinxAtStartPar
Bases: {\hyperref[\detokenize{_autosummary/signalanalysis.general.Signal:signalanalysis.general.Signal}]{\sphinxcrossref{\sphinxcode{\sphinxupquote{signalanalysis.general.Signal}}}}}

\sphinxAtStartPar
Base class to encapsulate data regarding an ECG recording, inheriting from {\hyperref[\detokenize{_autosummary/signalanalysis.general.Signal:signalanalysis.general.Signal}]{\sphinxcrossref{\sphinxcode{\sphinxupquote{signalanalysis.general.Signal}}}}}


\sphinxstrong{See also:}
\nopagebreak


\sphinxAtStartPar
{\hyperref[\detokenize{_autosummary/signalanalysis.general.Signal:signalanalysis.general.Signal}]{\sphinxcrossref{\sphinxcode{\sphinxupquote{signalanalysis.general.Signal}}}}}


\index{read() (signalanalysis.ecg.Ecg method)@\spxentry{read()}\spxextra{signalanalysis.ecg.Ecg method}}

\begin{fulllineitems}
\phantomsection\label{\detokenize{_autosummary/signalanalysis.ecg.Ecg:signalanalysis.ecg.Ecg.read}}\pysiglinewithargsret{\sphinxbfcode{\sphinxupquote{read}}}{\emph{\DUrole{n}{filename}}}{}
\sphinxAtStartPar
Reads in the data from the original file. Called upon initialisation

\end{fulllineitems}

\index{read\_ecg\_from\_wfdb() (signalanalysis.ecg.Ecg method)@\spxentry{read\_ecg\_from\_wfdb()}\spxextra{signalanalysis.ecg.Ecg method}}

\begin{fulllineitems}
\phantomsection\label{\detokenize{_autosummary/signalanalysis.ecg.Ecg:signalanalysis.ecg.Ecg.read_ecg_from_wfdb}}\pysiglinewithargsret{\sphinxbfcode{\sphinxupquote{read\_ecg\_from\_wfdb}}}{\emph{\DUrole{n}{filename}}, \emph{\DUrole{n}{normalise}\DUrole{o}{=}\DUrole{default_value}{False}}}{}
\sphinxAtStartPar
Reads data from a WFDB type series of files, e.g. from the Lobachevsky ECG database
(\sphinxurl{https://physionet.org/content/ludb/1.0.1/})

\end{fulllineitems}

\index{get\_n\_beats() (signalanalysis.ecg.Ecg method)@\spxentry{get\_n\_beats()}\spxextra{signalanalysis.ecg.Ecg method}}

\begin{fulllineitems}
\phantomsection\label{\detokenize{_autosummary/signalanalysis.ecg.Ecg:signalanalysis.ecg.Ecg.get_n_beats}}\pysiglinewithargsret{\sphinxbfcode{\sphinxupquote{get\_n\_beats}}}{\emph{\DUrole{n}{threshold}\DUrole{o}{=}\DUrole{default_value}{0.5}}, \emph{\DUrole{n}{min\_separation}\DUrole{o}{=}\DUrole{default_value}{0.2}}}{}
\sphinxAtStartPar
Calculates the number of beats given in the recording

\end{fulllineitems}

\index{get\_qrs\_start() (signalanalysis.ecg.Ecg method)@\spxentry{get\_qrs\_start()}\spxextra{signalanalysis.ecg.Ecg method}}

\begin{fulllineitems}
\phantomsection\label{\detokenize{_autosummary/signalanalysis.ecg.Ecg:signalanalysis.ecg.Ecg.get_qrs_start}}\pysiglinewithargsret{\sphinxbfcode{\sphinxupquote{get\_qrs\_start}}}{}{}
\sphinxAtStartPar
Calculates the start of the QRS complex

\end{fulllineitems}

\subsubsection*{Methods}


\begin{savenotes}\sphinxatlongtablestart\begin{longtable}[c]{\X{1}{2}\X{1}{2}}
\hline

\endfirsthead

\multicolumn{2}{c}%
{\makebox[0pt]{\sphinxtablecontinued{\tablename\ \thetable{} \textendash{} continued from previous page}}}\\
\hline

\endhead

\hline
\multicolumn{2}{r}{\makebox[0pt][r]{\sphinxtablecontinued{continues on next page}}}\\
\endfoot

\endlastfoot

\sphinxAtStartPar
{\hyperref[\detokenize{_autosummary/signalanalysis.ecg.Ecg:signalanalysis.ecg.Ecg.apply_filter}]{\sphinxcrossref{\sphinxcode{\sphinxupquote{apply\_filter}}}}}
&
\sphinxAtStartPar
Apply a given filter to the data, using their respective arguments as required
\\
\hline
\sphinxAtStartPar
{\hyperref[\detokenize{_autosummary/signalanalysis.ecg.Ecg:id0}]{\sphinxcrossref{\sphinxcode{\sphinxupquote{get\_n\_beats}}}}}
&
\sphinxAtStartPar
Calculate the number of beats in a given signal, and save the individual beats to the object for later use
\\
\hline
\sphinxAtStartPar
{\hyperref[\detokenize{_autosummary/signalanalysis.ecg.Ecg:id1}]{\sphinxcrossref{\sphinxcode{\sphinxupquote{get\_qrs\_start}}}}}
&
\sphinxAtStartPar
Calculates start of QRS complex using method of Hermans et al. (2017).
\\
\hline
\sphinxAtStartPar
{\hyperref[\detokenize{_autosummary/signalanalysis.ecg.Ecg:signalanalysis.ecg.Ecg.get_rms}]{\sphinxcrossref{\sphinxcode{\sphinxupquote{get\_rms}}}}}
&
\sphinxAtStartPar
Supplement the {\hyperref[\detokenize{_autosummary/signalanalysis.general.Signal:id1}]{\sphinxcrossref{\sphinxcode{\sphinxupquote{signalanalysis.general.Signal.get\_rms()}}}}} with \sphinxtitleref{unipolar\_only}
\\
\hline
\sphinxAtStartPar
\sphinxcode{\sphinxupquote{get\_twave\_end}}
&
\sphinxAtStartPar

\\
\hline
\sphinxAtStartPar
{\hyperref[\detokenize{_autosummary/signalanalysis.ecg.Ecg:signalanalysis.ecg.Ecg.plot}]{\sphinxcrossref{\sphinxcode{\sphinxupquote{plot}}}}}
&
\sphinxAtStartPar
Method to plot the data of the ECG
\\
\hline
\sphinxAtStartPar
{\hyperref[\detokenize{_autosummary/signalanalysis.ecg.Ecg:id4}]{\sphinxcrossref{\sphinxcode{\sphinxupquote{read}}}}}
&
\sphinxAtStartPar
Reads in the data from the original file.
\\
\hline
\sphinxAtStartPar
{\hyperref[\detokenize{_autosummary/signalanalysis.ecg.Ecg:id5}]{\sphinxcrossref{\sphinxcode{\sphinxupquote{read\_ecg\_from\_wfdb}}}}}
&
\sphinxAtStartPar
Read data from a waveform database file format
\\
\hline
\sphinxAtStartPar
{\hyperref[\detokenize{_autosummary/signalanalysis.ecg.Ecg:signalanalysis.ecg.Ecg.reset}]{\sphinxcrossref{\sphinxcode{\sphinxupquote{reset}}}}}
&
\sphinxAtStartPar
Reset all properties of the class
\\
\hline
\end{longtable}\sphinxatlongtableend\end{savenotes}
\index{apply\_filter() (signalanalysis.ecg.Ecg method)@\spxentry{apply\_filter()}\spxextra{signalanalysis.ecg.Ecg method}}

\begin{fulllineitems}
\phantomsection\label{\detokenize{_autosummary/signalanalysis.ecg.Ecg:signalanalysis.ecg.Ecg.apply_filter}}\pysiglinewithargsret{\sphinxbfcode{\sphinxupquote{apply\_filter}}}{\emph{\DUrole{o}{**}\DUrole{n}{kwargs}}}{}
\sphinxAtStartPar
Apply a given filter to the data, using their respective arguments as required


\sphinxstrong{See also:}
\nopagebreak

\begin{description}
\item[{{\hyperref[\detokenize{_autosummary/tools.maths.filter_butterworth:tools.maths.filter_butterworth}]{\sphinxcrossref{\sphinxcode{\sphinxupquote{tools.maths.filter\_butterworth()}}}}}}] \leavevmode
\sphinxAtStartPar
Potential filter

\item[{{\hyperref[\detokenize{_autosummary/tools.maths.filter_savitzkygolay:tools.maths.filter_savitzkygolay}]{\sphinxcrossref{\sphinxcode{\sphinxupquote{tools.maths.filter\_savitzkygolay()}}}}}}] \leavevmode
\sphinxAtStartPar
Potential filter

\end{description}



\end{fulllineitems}

\index{get\_n\_beats() (signalanalysis.ecg.Ecg method)@\spxentry{get\_n\_beats()}\spxextra{signalanalysis.ecg.Ecg method}}

\begin{fulllineitems}
\phantomsection\label{\detokenize{_autosummary/signalanalysis.ecg.Ecg:id0}}\pysiglinewithargsret{\sphinxbfcode{\sphinxupquote{get\_n\_beats}}}{\emph{\DUrole{n}{threshold}\DUrole{p}{:} \DUrole{n}{float} \DUrole{o}{=} \DUrole{default_value}{0.5}}, \emph{\DUrole{n}{min\_separation}\DUrole{p}{:} \DUrole{n}{float} \DUrole{o}{=} \DUrole{default_value}{0.2}}, \emph{\DUrole{n}{reset\_index}\DUrole{p}{:} \DUrole{n}{bool} \DUrole{o}{=} \DUrole{default_value}{True}}, \emph{\DUrole{n}{plot}\DUrole{p}{:} \DUrole{n}{bool} \DUrole{o}{=} \DUrole{default_value}{False}}, \emph{\DUrole{o}{**}\DUrole{n}{kwargs}}}{}
\sphinxAtStartPar
Calculate the number of beats in a given signal, and save the individual beats to the object for later use

\sphinxAtStartPar
When given the raw data of a given signal (ECG or VCG), will estimate the number of beats recorded in the trace
based on the RMS of the signal exceeding a threshold value. The estimated individual beats will then be saved in
a list in a lossless manner, i.e. saved as {[}ECG1, ECG2, …, ECG(n){]}, where ECG1={[}0:peak2{]}, ECG2={[}peak1:peak3{]},
…, ECGn={[}peak(n\sphinxhyphen{}1):end{]}
\begin{quote}\begin{description}
\item[{Parameters}] \leavevmode\begin{itemize}
\item {} 
\sphinxAtStartPar
\sphinxstyleliteralstrong{\sphinxupquote{threshold}} (\sphinxstyleliteralemphasis{\sphinxupquote{float \{0\textless{}1\}}}) \textendash{} Minimum value to search for for a peak in RMS signal to determine when a beat has occurred, default=0.5

\item {} 
\sphinxAtStartPar
\sphinxstyleliteralstrong{\sphinxupquote{min\_separation}} (\sphinxstyleliteralemphasis{\sphinxupquote{float}}) \textendash{} Minimum time (in s) that should be used to separate separate beats, default=0.2s

\item {} 
\sphinxAtStartPar
\sphinxstyleliteralstrong{\sphinxupquote{reset\_index}} (\sphinxstyleliteralemphasis{\sphinxupquote{bool}}) \textendash{} Whether to reset the time index for the separated beats so that they all start from zero (true),
or whether to leave them with the original time index (false), default=True

\item {} 
\sphinxAtStartPar
\sphinxstyleliteralstrong{\sphinxupquote{plot}} (\sphinxstyleliteralemphasis{\sphinxupquote{bool}}) \textendash{} Whether to plot results of beat detection, default=False

\item {} 
\sphinxAtStartPar
\sphinxstyleliteralstrong{\sphinxupquote{unipolar\_only}} (\sphinxstyleliteralemphasis{\sphinxupquote{bool}}\sphinxstyleliteralemphasis{\sphinxupquote{, }}\sphinxstyleliteralemphasis{\sphinxupquote{optional}}) \textendash{} Only appropriate for ECG data. Whether to use only unipolar ECG leads to calculate RMS, default=True

\end{itemize}

\item[{Returns}] \leavevmode
\sphinxAtStartPar
\sphinxstylestrong{self.n\_beats} \textendash{} Number of beats detected in signal

\item[{Return type}] \leavevmode
\sphinxAtStartPar
int

\end{description}\end{quote}


\sphinxstrong{See also:}
\nopagebreak

\begin{description}
\item[{{\hyperref[\detokenize{_autosummary/signalanalysis.general.Signal:id1}]{\sphinxcrossref{\sphinxcode{\sphinxupquote{signalanalysis.general.Signal.get\_rms()}}}}}}] \leavevmode
\sphinxAtStartPar
RMS signal calculation required for getting n\_beats

\end{description}



\end{fulllineitems}

\index{get\_qrs\_start() (signalanalysis.ecg.Ecg method)@\spxentry{get\_qrs\_start()}\spxextra{signalanalysis.ecg.Ecg method}}

\begin{fulllineitems}
\phantomsection\label{\detokenize{_autosummary/signalanalysis.ecg.Ecg:id1}}\pysiglinewithargsret{\sphinxbfcode{\sphinxupquote{get\_qrs\_start}}}{\emph{\DUrole{n}{unipolar\_only}\DUrole{p}{:} \DUrole{n}{bool} \DUrole{o}{=} \DUrole{default_value}{True}}, \emph{\DUrole{n}{min\_separation}\DUrole{p}{:} \DUrole{n}{float} \DUrole{o}{=} \DUrole{default_value}{0.05}}, \emph{\DUrole{n}{plot\_result}\DUrole{p}{:} \DUrole{n}{bool} \DUrole{o}{=} \DUrole{default_value}{False}}}{}
\sphinxAtStartPar
Calculates start of QRS complex using method of Hermans et al. (2017)

\sphinxAtStartPar
Calculates the start of the QRS complex by a simplified version of the work presented in \sphinxstepexplicit %
\begin{footnote}[1]\phantomsection\label{\thesphinxscope.1}%
\sphinxAtStartFootnote
Hermans BJM, Vink AS, Bennis FC, Filippini LH, Meijborg VMF, Wilde AAM, Pison L, Postema PG, Delhaas T,
“The development and validation of an easy to use automatic QT\sphinxhyphen{}interval algorithm,”
PLoS ONE, 12(9), 1\textendash{}14 (2017), \sphinxurl{https://doi.org/10.1371/journal.pone.0184352}
%
\end{footnote}, wherein the
point of maximum second derivative of the ECG RMS signal is used as the start of the QRS complex
\begin{quote}\begin{description}
\item[{Parameters}] \leavevmode\begin{itemize}
\item {} 
\sphinxAtStartPar
\sphinxstyleliteralstrong{\sphinxupquote{self}} ({\hyperref[\detokenize{_autosummary/signalanalysis.ecg.Ecg:signalanalysis.ecg.Ecg}]{\sphinxcrossref{\sphinxstyleliteralemphasis{\sphinxupquote{Ecg}}}}}) \textendash{} ECG data to analyse

\item {} 
\sphinxAtStartPar
\sphinxstyleliteralstrong{\sphinxupquote{unipolar\_only}} (\sphinxstyleliteralemphasis{\sphinxupquote{bool}}\sphinxstyleliteralemphasis{\sphinxupquote{, }}\sphinxstyleliteralemphasis{\sphinxupquote{optional}}) \textendash{} Whether to use only unipolar leads to calculate RMS, default=True

\item {} 
\sphinxAtStartPar
\sphinxstyleliteralstrong{\sphinxupquote{min\_separation}} (\sphinxstyleliteralemphasis{\sphinxupquote{float}}\sphinxstyleliteralemphasis{\sphinxupquote{, }}\sphinxstyleliteralemphasis{\sphinxupquote{optional}}) \textendash{} Minimum separation from the peak used to detect various beats, default=0.05s

\item {} 
\sphinxAtStartPar
\sphinxstyleliteralstrong{\sphinxupquote{plot\_result}} (\sphinxstyleliteralemphasis{\sphinxupquote{bool}}\sphinxstyleliteralemphasis{\sphinxupquote{, }}\sphinxstyleliteralemphasis{\sphinxupquote{optional}}) \textendash{} Whether to plot the results for error\sphinxhyphen{}checking, default=False

\end{itemize}

\item[{Returns}] \leavevmode
\sphinxAtStartPar
\sphinxstylestrong{self.qrs\_start} \textendash{} QRS start times

\item[{Return type}] \leavevmode
\sphinxAtStartPar
list of float

\end{description}\end{quote}
\subsubsection*{Notes}

\sphinxAtStartPar
For further details of the action of unipolar\_only, see general\_analysis.get\_signal\_rms

\sphinxAtStartPar
It is faster to use scipy.ndimage.laplace() rather than np.gradient(np.gradient)), but preliminary checks
indicated some edge problems that might throw off the results.
\subsubsection*{References}

\end{fulllineitems}

\index{get\_rms() (signalanalysis.ecg.Ecg method)@\spxentry{get\_rms()}\spxextra{signalanalysis.ecg.Ecg method}}

\begin{fulllineitems}
\phantomsection\label{\detokenize{_autosummary/signalanalysis.ecg.Ecg:signalanalysis.ecg.Ecg.get_rms}}\pysiglinewithargsret{\sphinxbfcode{\sphinxupquote{get\_rms}}}{\emph{\DUrole{n}{preprocess\_data}\DUrole{p}{:} \DUrole{n}{Optional\DUrole{p}{{[}}pandas.core.frame.DataFrame\DUrole{p}{{]}}} \DUrole{o}{=} \DUrole{default_value}{None}}, \emph{\DUrole{n}{drop\_columns}\DUrole{p}{:} \DUrole{n}{Optional\DUrole{p}{{[}}List\DUrole{p}{{[}}str\DUrole{p}{{]}}\DUrole{p}{{]}}} \DUrole{o}{=} \DUrole{default_value}{None}}, \emph{\DUrole{n}{unipolar\_only}\DUrole{p}{:} \DUrole{n}{bool} \DUrole{o}{=} \DUrole{default_value}{True}}}{}
\sphinxAtStartPar
Supplement the {\hyperref[\detokenize{_autosummary/signalanalysis.general.Signal:id1}]{\sphinxcrossref{\sphinxcode{\sphinxupquote{signalanalysis.general.Signal.get\_rms()}}}}} with \sphinxtitleref{unipolar\_only}
\begin{quote}\begin{description}
\item[{Parameters}] \leavevmode\begin{itemize}
\item {} 
\sphinxAtStartPar
\sphinxstyleliteralstrong{\sphinxupquote{preprocess\_data}} (\sphinxstyleliteralemphasis{\sphinxupquote{pd.DataFrame}}\sphinxstyleliteralemphasis{\sphinxupquote{, }}\sphinxstyleliteralemphasis{\sphinxupquote{optional}}) \textendash{} See {\hyperref[\detokenize{_autosummary/signalanalysis.general.Signal:id1}]{\sphinxcrossref{\sphinxcode{\sphinxupquote{signalanalysis.general.Signal.get\_rms()}}}}}

\item {} 
\sphinxAtStartPar
\sphinxstyleliteralstrong{\sphinxupquote{drop\_columns}} (\sphinxstyleliteralemphasis{\sphinxupquote{list of str}}\sphinxstyleliteralemphasis{\sphinxupquote{, }}\sphinxstyleliteralemphasis{\sphinxupquote{optional}}) \textendash{} See {\hyperref[\detokenize{_autosummary/signalanalysis.general.Signal:id1}]{\sphinxcrossref{\sphinxcode{\sphinxupquote{signalanalysis.general.Signal.get\_rms()}}}}}

\item {} 
\sphinxAtStartPar
\sphinxstyleliteralstrong{\sphinxupquote{unipolar\_only}} (\sphinxstyleliteralemphasis{\sphinxupquote{optional}}) \textendash{} Whether to use only unipolar ECG leads to calculate RMS, default=True

\end{itemize}

\end{description}\end{quote}


\sphinxstrong{See also:}
\nopagebreak


\sphinxAtStartPar
{\hyperref[\detokenize{_autosummary/signalanalysis.general.Signal:id1}]{\sphinxcrossref{\sphinxcode{\sphinxupquote{signalanalysis.general.Signal.get\_rms()}}}}}


\subsubsection*{Notes}

\sphinxAtStartPar
If \sphinxtitleref{unipolar\_only} is set to true, then ECG RMS is calculated using only ‘unipolar’ leads. This uses V1\sphinxhyphen{}6,
and the non\sphinxhyphen{}augmented limb leads (VF, VL and VR)
\begin{equation*}
\begin{split}VF = LL-V_{WCT} = \frac{2}{3}aVF\end{split}
\end{equation*}\begin{equation*}
\begin{split}VL = LA-V_{WCT} = \frac{2}{3}aVL\end{split}
\end{equation*}\begin{equation*}
\begin{split}VR = RA-V_{WCT} = \frac{2}{3}aVR\end{split}
\end{equation*}
\end{fulllineitems}

\index{plot() (signalanalysis.ecg.Ecg method)@\spxentry{plot()}\spxextra{signalanalysis.ecg.Ecg method}}

\begin{fulllineitems}
\phantomsection\label{\detokenize{_autosummary/signalanalysis.ecg.Ecg:signalanalysis.ecg.Ecg.plot}}\pysiglinewithargsret{\sphinxbfcode{\sphinxupquote{plot}}}{\emph{\DUrole{n}{separate\_beats}\DUrole{p}{:} \DUrole{n}{bool} \DUrole{o}{=} \DUrole{default_value}{False}}, \emph{\DUrole{n}{separate\_figs}\DUrole{p}{:} \DUrole{n}{bool} \DUrole{o}{=} \DUrole{default_value}{False}}}{}
\sphinxAtStartPar
Method to plot the data of the ECG

\sphinxAtStartPar
Passes the function to the ECG plotting script, with the option to plot the individual beats instead of the
entire signal; also, to plot these on individual figures or overlaid on the same figure
\begin{quote}\begin{description}
\item[{Parameters}] \leavevmode\begin{itemize}
\item {} 
\sphinxAtStartPar
\sphinxstyleliteralstrong{\sphinxupquote{separate\_beats}} (\sphinxstyleliteralemphasis{\sphinxupquote{bool}}\sphinxstyleliteralemphasis{\sphinxupquote{, }}\sphinxstyleliteralemphasis{\sphinxupquote{optional}}) \textendash{} Whether to plot the entire signal (false), or to plot the individual detected beats one after the other,
default=False

\item {} 
\sphinxAtStartPar
\sphinxstyleliteralstrong{\sphinxupquote{separate\_figs}} (\sphinxstyleliteralemphasis{\sphinxupquote{bool}}\sphinxstyleliteralemphasis{\sphinxupquote{, }}\sphinxstyleliteralemphasis{\sphinxupquote{optional}}) \textendash{} When plotting the separate beats (if requested), whether to plot the individual beats one on top of the
other on a single figure (true), or on separate figures (false), default=False

\end{itemize}

\item[{Returns}] \leavevmode
\sphinxAtStartPar


\item[{Return type}] \leavevmode
\sphinxAtStartPar
fig, ax

\end{description}\end{quote}

\end{fulllineitems}

\index{read() (signalanalysis.ecg.Ecg method)@\spxentry{read()}\spxextra{signalanalysis.ecg.Ecg method}}

\begin{fulllineitems}
\phantomsection\label{\detokenize{_autosummary/signalanalysis.ecg.Ecg:id4}}\pysiglinewithargsret{\sphinxbfcode{\sphinxupquote{read}}}{\emph{\DUrole{n}{filename}\DUrole{p}{:} \DUrole{n}{str}}, \emph{\DUrole{n}{normalise}\DUrole{p}{:} \DUrole{n}{bool} \DUrole{o}{=} \DUrole{default_value}{False}}, \emph{\DUrole{o}{**}\DUrole{n}{kwargs}}}{}
\sphinxAtStartPar
Reads in the data from the original file. Called upon initialisation
\begin{quote}\begin{description}
\item[{Parameters}] \leavevmode\begin{itemize}
\item {} 
\sphinxAtStartPar
\sphinxstyleliteralstrong{\sphinxupquote{filename}} (\sphinxstyleliteralemphasis{\sphinxupquote{str}}) \textendash{} Location for data (either filename or directory)

\item {} 
\sphinxAtStartPar
\sphinxstyleliteralstrong{\sphinxupquote{normalise}} (\sphinxstyleliteralemphasis{\sphinxupquote{bool}}\sphinxstyleliteralemphasis{\sphinxupquote{, }}\sphinxstyleliteralemphasis{\sphinxupquote{optional}}) \textendash{} Whether or not to normalise the individual leads (no real biophysical rationale for doing this,
to be honest), default=False

\end{itemize}

\end{description}\end{quote}


\sphinxstrong{See also:}
\nopagebreak

\begin{description}
\item[{{\hyperref[\detokenize{_autosummary/signalanalysis.ecg.Ecg:id5}]{\sphinxcrossref{\sphinxcode{\sphinxupquote{signalanalysis.ecg.Ecg.read\_ecg\_from\_wfdb}}}}}}] \leavevmode
\sphinxAtStartPar
underlying read method for wfdb files

\end{description}



\end{fulllineitems}

\index{read\_ecg\_from\_wfdb() (signalanalysis.ecg.Ecg method)@\spxentry{read\_ecg\_from\_wfdb()}\spxextra{signalanalysis.ecg.Ecg method}}

\begin{fulllineitems}
\phantomsection\label{\detokenize{_autosummary/signalanalysis.ecg.Ecg:id5}}\pysiglinewithargsret{\sphinxbfcode{\sphinxupquote{read\_ecg\_from\_wfdb}}}{\emph{\DUrole{n}{filename}\DUrole{p}{:} \DUrole{n}{str}}, \emph{\DUrole{n}{sample\_rate}\DUrole{p}{:} \DUrole{n}{float}}, \emph{\DUrole{n}{comments\_file}\DUrole{p}{:} \DUrole{n}{Optional\DUrole{p}{{[}}str\DUrole{p}{{]}}} \DUrole{o}{=} \DUrole{default_value}{None}}, \emph{\DUrole{n}{normalise}\DUrole{p}{:} \DUrole{n}{bool} \DUrole{o}{=} \DUrole{default_value}{False}}}{}
\sphinxAtStartPar
Read data from a waveform database file format
\begin{quote}\begin{description}
\item[{Parameters}] \leavevmode\begin{itemize}
\item {} 
\sphinxAtStartPar
\sphinxstyleliteralstrong{\sphinxupquote{filename}} (\sphinxstyleliteralemphasis{\sphinxupquote{str}}) \textendash{} Base filename for data (e.g. /path/to/data/1 will read /path/to/data/1.\{avf, avl, avr, dat, hea\}

\item {} 
\sphinxAtStartPar
\sphinxstyleliteralstrong{\sphinxupquote{sample\_rate}} (\sphinxstyleliteralemphasis{\sphinxupquote{float}}) \textendash{} Rate at which data is recorded, e.g. 500 Hz

\item {} 
\sphinxAtStartPar
\sphinxstyleliteralstrong{\sphinxupquote{comments\_file}} (\sphinxstyleliteralemphasis{\sphinxupquote{str}}\sphinxstyleliteralemphasis{\sphinxupquote{, }}\sphinxstyleliteralemphasis{\sphinxupquote{optional}}) \textendash{} .csv file containing additional data for the ECG, if available

\item {} 
\sphinxAtStartPar
\sphinxstyleliteralstrong{\sphinxupquote{normalise}} (\sphinxstyleliteralemphasis{\sphinxupquote{bool}}\sphinxstyleliteralemphasis{\sphinxupquote{, }}\sphinxstyleliteralemphasis{\sphinxupquote{optional}}) \textendash{} Whether to normalise all individual leads, default=False

\end{itemize}

\end{description}\end{quote}
\subsubsection*{Notes}

\sphinxAtStartPar
The waveform database format is used by several different ECG repositories on www.physionet.org,
and the finer points of each import can be difficult to seamlessly integrate. Where values are subject to
change between datasets, they are not available as defaults. The required values are thus, but it should be
remembered that there remains a certain level of hard\sphinxhyphen{}coding (for example, in the use of \sphinxtitleref{comments\_file} to
extract data).
\begin{itemize}
\item {} 
\sphinxAtStartPar
Lobachevsky
\begin{itemize}
\item {} 
\sphinxAtStartPar
sample\_rate = 500

\end{itemize}

\item {} 
\sphinxAtStartPar
PTB\sphinxhyphen{}XL
\begin{itemize}
\item {} 
\sphinxAtStartPar
sample\_rate = 100 if from records100/{\color{red}\bfseries{}*}/{\color{red}\bfseries{}*}\_lr

\item {} 
\sphinxAtStartPar
sample\_rate = 500 if from records500/{\color{red}\bfseries{}*}/{\color{red}\bfseries{}*}\_hr

\item {} 
\sphinxAtStartPar
comments\_file = ‘ptbxl\_database.csv’

\end{itemize}

\end{itemize}

\end{fulllineitems}

\index{reset() (signalanalysis.ecg.Ecg method)@\spxentry{reset()}\spxextra{signalanalysis.ecg.Ecg method}}

\begin{fulllineitems}
\phantomsection\label{\detokenize{_autosummary/signalanalysis.ecg.Ecg:signalanalysis.ecg.Ecg.reset}}\pysiglinewithargsret{\sphinxbfcode{\sphinxupquote{reset}}}{}{}
\sphinxAtStartPar
Reset all properties of the class

\sphinxAtStartPar
Function called when reading in new data into an existing class (for some reason), which would make these
properties and attributes clash with the other data

\end{fulllineitems}


\end{fulllineitems}



\subsection{signalanalysis.general}
\label{\detokenize{_autosummary/signalanalysis.general:module-signalanalysis.general}}\label{\detokenize{_autosummary/signalanalysis.general:signalanalysis-general}}\label{\detokenize{_autosummary/signalanalysis.general::doc}}\index{module@\spxentry{module}!signalanalysis.general@\spxentry{signalanalysis.general}}\index{signalanalysis.general@\spxentry{signalanalysis.general}!module@\spxentry{module}}\subsubsection*{Functions}


\begin{savenotes}\sphinxatlongtablestart\begin{longtable}[c]{\X{1}{2}\X{1}{2}}
\hline

\endfirsthead

\multicolumn{2}{c}%
{\makebox[0pt]{\sphinxtablecontinued{\tablename\ \thetable{} \textendash{} continued from previous page}}}\\
\hline

\endhead

\hline
\multicolumn{2}{r}{\makebox[0pt][r]{\sphinxtablecontinued{continues on next page}}}\\
\endfoot

\endlastfoot

\sphinxAtStartPar
{\hyperref[\detokenize{_autosummary/signalanalysis.general.get_signal_rms:signalanalysis.general.get_signal_rms}]{\sphinxcrossref{\sphinxcode{\sphinxupquote{get\_signal\_rms}}}}}
&
\sphinxAtStartPar
Calculate the ECG(RMS) of the ECG as a scalar
\\
\hline
\sphinxAtStartPar
{\hyperref[\detokenize{_autosummary/signalanalysis.general.get_twave_end:signalanalysis.general.get_twave_end}]{\sphinxcrossref{\sphinxcode{\sphinxupquote{get\_twave\_end}}}}}
&
\sphinxAtStartPar
Return the time point at which it is estimated that the T\sphinxhyphen{}wave has been completed
\\
\hline
\end{longtable}\sphinxatlongtableend\end{savenotes}


\subsubsection{signalanalysis.general.get\_signal\_rms}
\label{\detokenize{_autosummary/signalanalysis.general.get_signal_rms:signalanalysis-general-get-signal-rms}}\label{\detokenize{_autosummary/signalanalysis.general.get_signal_rms::doc}}\index{get\_signal\_rms() (in module signalanalysis.general)@\spxentry{get\_signal\_rms()}\spxextra{in module signalanalysis.general}}

\begin{fulllineitems}
\phantomsection\label{\detokenize{_autosummary/signalanalysis.general.get_signal_rms:signalanalysis.general.get_signal_rms}}\pysiglinewithargsret{\sphinxcode{\sphinxupquote{signalanalysis.general.}}\sphinxbfcode{\sphinxupquote{get\_signal\_rms}}}{\emph{\DUrole{n}{signal}\DUrole{p}{:} \DUrole{n}{pandas.core.frame.DataFrame}}, \emph{\DUrole{n}{unipolar\_only}\DUrole{p}{:} \DUrole{n}{bool} \DUrole{o}{=} \DUrole{default_value}{True}}}{{ $\rightarrow$ List\DUrole{p}{{[}}float\DUrole{p}{{]}}}}
\sphinxAtStartPar
Calculate the ECG(RMS) of the ECG as a scalar
\begin{quote}

\sphinxAtStartPar
\DUrole{versionmodified,deprecated}{Deprecated since version The: }use of this module is deprecated, and the internal class method should be used in preference (
signalanalysis.general.Signal.get\_rms())
\begin{description}
\item[{signal: pd.DataFrame}] \leavevmode
\sphinxAtStartPar
ECG or VCG data to process

\item[{unipolar\_only}] \leavevmode{[}bool, optional{]}
\sphinxAtStartPar
Whether to use only unipolar ECG leads to calculate RMS, default=True

\end{description}
\begin{description}
\item[{signal\_rms}] \leavevmode{[}list of float{]}
\sphinxAtStartPar
Scalar RMS ECG or VCG data

\end{description}

\sphinxAtStartPar
The scalar RMS is calculated according to
\begin{equation*}
\begin{split}\sqrt{\end{split}
\end{equation*}\end{quote}

\sphinxAtStartPar
rac\{1\}\{n\}sum\_\{i=1\}\textasciicircum{}n (    extnormal\{ECG\}\_i\textasciicircum{}2(t))\}
\begin{quote}

\sphinxAtStartPar
for all leads available from the signal (12 for ECG, 3 for VCG). If unipolar\_only is set to true, then ECG RMS is
calculated using only ‘unipolar’ leads. This uses V1\sphinxhyphen{}6, and the non\sphinxhyphen{}augmented limb leads (VF, VL and VR)

\sphinxAtStartPar
..math:: VF = LL\sphinxhyphen{}V\_\{WCT\} =
\end{quote}
\begin{description}
\item[{rac\{2\}\{3\}aVF}] \leavevmode
\sphinxAtStartPar
..math:: VL = LA\sphinxhyphen{}V\_\{WCT\} =

\item[{rac\{2\}\{3\}aVL}] \leavevmode
\sphinxAtStartPar
..math:: VR = RA\sphinxhyphen{}V\_\{WCT\} =

\end{description}

\sphinxAtStartPar
rac\{2\}\{3\}aVR
\begin{quote}
\begin{description}
\item[{The development and validation of an easy to use automatic QT\sphinxhyphen{}interval algorithm}] \leavevmode
\sphinxAtStartPar
Hermans BJM, Vink AS, Bennis FC, Filippini LH, Meijborg VMF, Wilde AAM, Pison L, Postema PG, Delhaas T
PLoS ONE, 12(9), 1\textendash{}14 (2017)
\sphinxurl{https://doi.org/10.1371/journal.pone.0184352}

\end{description}
\end{quote}

\end{fulllineitems}



\subsubsection{signalanalysis.general.get\_twave\_end}
\label{\detokenize{_autosummary/signalanalysis.general.get_twave_end:signalanalysis-general-get-twave-end}}\label{\detokenize{_autosummary/signalanalysis.general.get_twave_end::doc}}\index{get\_twave\_end() (in module signalanalysis.general)@\spxentry{get\_twave\_end()}\spxextra{in module signalanalysis.general}}

\begin{fulllineitems}
\phantomsection\label{\detokenize{_autosummary/signalanalysis.general.get_twave_end:signalanalysis.general.get_twave_end}}\pysiglinewithargsret{\sphinxcode{\sphinxupquote{signalanalysis.general.}}\sphinxbfcode{\sphinxupquote{get\_twave\_end}}}{\emph{\DUrole{n}{ecgs}\DUrole{p}{:} \DUrole{n}{Union\DUrole{p}{{[}}List\DUrole{p}{{[}}pandas.core.frame.DataFrame\DUrole{p}{{]}}\DUrole{p}{, }pandas.core.frame.DataFrame\DUrole{p}{{]}}}}, \emph{\DUrole{n}{leads}\DUrole{p}{:} \DUrole{n}{Union\DUrole{p}{{[}}str\DUrole{p}{, }List\DUrole{p}{{[}}str\DUrole{p}{{]}}\DUrole{p}{{]}}} \DUrole{o}{=} \DUrole{default_value}{\textquotesingle{}LII\textquotesingle{}}}, \emph{\DUrole{n}{i\_distance}\DUrole{p}{:} \DUrole{n}{int} \DUrole{o}{=} \DUrole{default_value}{200}}, \emph{\DUrole{n}{filter\_signal}\DUrole{p}{:} \DUrole{n}{Optional\DUrole{p}{{[}}str\DUrole{p}{{]}}} \DUrole{o}{=} \DUrole{default_value}{None}}, \emph{\DUrole{n}{baseline\_adjust}\DUrole{p}{:} \DUrole{n}{Optional\DUrole{p}{{[}}Union\DUrole{p}{{[}}float\DUrole{p}{, }List\DUrole{p}{{[}}float\DUrole{p}{{]}}\DUrole{p}{{]}}\DUrole{p}{{]}}} \DUrole{o}{=} \DUrole{default_value}{None}}, \emph{\DUrole{n}{return\_median}\DUrole{p}{:} \DUrole{n}{bool} \DUrole{o}{=} \DUrole{default_value}{True}}, \emph{\DUrole{n}{remove\_outliers}\DUrole{p}{:} \DUrole{n}{bool} \DUrole{o}{=} \DUrole{default_value}{True}}, \emph{\DUrole{n}{plot\_result}\DUrole{p}{:} \DUrole{n}{bool} \DUrole{o}{=} \DUrole{default_value}{False}}}{{ $\rightarrow$ List\DUrole{p}{{[}}pandas.core.frame.DataFrame\DUrole{p}{{]}}}}
\sphinxAtStartPar
Return the time point at which it is estimated that the T\sphinxhyphen{}wave has been completed
\begin{quote}\begin{description}
\item[{Parameters}] \leavevmode\begin{itemize}
\item {} 
\sphinxAtStartPar
\sphinxstyleliteralstrong{\sphinxupquote{ecgs}} (\sphinxstyleliteralemphasis{\sphinxupquote{pd.DataFrame}}\sphinxstyleliteralemphasis{\sphinxupquote{ or }}\sphinxstyleliteralemphasis{\sphinxupquote{list of pd.DataFrame}}) \textendash{} Signal data, either ECG or VCG

\item {} 
\sphinxAtStartPar
\sphinxstyleliteralstrong{\sphinxupquote{leads}} (\sphinxstyleliteralemphasis{\sphinxupquote{str}}\sphinxstyleliteralemphasis{\sphinxupquote{, }}\sphinxstyleliteralemphasis{\sphinxupquote{optional}}) \textendash{} Which lead to check for the T\sphinxhyphen{}wave \sphinxhyphen{} usually this is either ‘LII’ or ‘V5’, but can be set to a list of
various leads. If set to ‘global’, then all T\sphinxhyphen{}wave values will be calculated. Will return all values unless
return\_median flag is set. Default ‘LII’

\item {} 
\sphinxAtStartPar
\sphinxstyleliteralstrong{\sphinxupquote{i\_distance}} (\sphinxstyleliteralemphasis{\sphinxupquote{int}}\sphinxstyleliteralemphasis{\sphinxupquote{, }}\sphinxstyleliteralemphasis{\sphinxupquote{optional}}) \textendash{} Distance between peaks in the gradient, i.e. will direct that the function will only find the points of
maximum gradient (representing T\sphinxhyphen{}wave, etc.) with a minimum distance given here (in terms of indices,
rather than time). Helps prevent being overly sensitive to ‘wobbles’ in the ecg. Default=200

\item {} 
\sphinxAtStartPar
\sphinxstyleliteralstrong{\sphinxupquote{filter\_signal}} (\sphinxstyleliteralemphasis{\sphinxupquote{\{\textquotesingle{}butterworth\textquotesingle{}}}\sphinxstyleliteralemphasis{\sphinxupquote{, }}\sphinxstyleliteralemphasis{\sphinxupquote{\textquotesingle{}savitzky\sphinxhyphen{}golay\textquotesingle{}\}}}\sphinxstyleliteralemphasis{\sphinxupquote{, }}\sphinxstyleliteralemphasis{\sphinxupquote{optional}}) \textendash{} Whether or not to apply a filter to the data prior to trying to find the actual T\sphinxhyphen{}wave gradient. Can pass
either a Butterworth filter or a Savitzky\sphinxhyphen{}Golay filter, in which case the required kwargs for each can be
provided. Default=None (no filter applied)

\item {} 
\sphinxAtStartPar
\sphinxstyleliteralstrong{\sphinxupquote{baseline\_adjust}} (\sphinxstyleliteralemphasis{\sphinxupquote{float}}\sphinxstyleliteralemphasis{\sphinxupquote{ or }}\sphinxstyleliteralemphasis{\sphinxupquote{list of float}}\sphinxstyleliteralemphasis{\sphinxupquote{, }}\sphinxstyleliteralemphasis{\sphinxupquote{optional}}) \textendash{} Point from which to calculate the adjusted baseline for calculating the T\sphinxhyphen{}wave, rather than using the
zeroline. In line with Hermans et al., this is usually the start of the QRS complex, with the baseline
calculated as the median amplitude of the 30ms before this point.

\item {} 
\sphinxAtStartPar
\sphinxstyleliteralstrong{\sphinxupquote{return\_median}} (\sphinxstyleliteralemphasis{\sphinxupquote{bool}}\sphinxstyleliteralemphasis{\sphinxupquote{, }}\sphinxstyleliteralemphasis{\sphinxupquote{optional}}) \textendash{} Whether or not to return an average of the leads requested, default=True

\item {} 
\sphinxAtStartPar
\sphinxstyleliteralstrong{\sphinxupquote{remove\_outliers}} (\sphinxstyleliteralemphasis{\sphinxupquote{bool}}\sphinxstyleliteralemphasis{\sphinxupquote{, }}\sphinxstyleliteralemphasis{\sphinxupquote{optional}}) \textendash{} Whether to remove T\sphinxhyphen{}wave end values that are greater than 1 standard deviation from the mean from the data. Only
has an effect if more than one lead is provided, and return\_average is True. Default=True

\item {} 
\sphinxAtStartPar
\sphinxstyleliteralstrong{\sphinxupquote{plot\_result}} (\sphinxstyleliteralemphasis{\sphinxupquote{bool}}\sphinxstyleliteralemphasis{\sphinxupquote{, }}\sphinxstyleliteralemphasis{\sphinxupquote{optional}}) \textendash{} Whether to plot the results or not, default=False

\end{itemize}

\item[{Returns}] \leavevmode
\sphinxAtStartPar
\sphinxstylestrong{twave\_ends} \textendash{} Time value for when T\sphinxhyphen{}wave is estimated to have ended.

\item[{Return type}] \leavevmode
\sphinxAtStartPar
list of pd.DataFrame

\end{description}\end{quote}
\subsubsection*{Notes}

\sphinxAtStartPar
Calculates the end of the T\sphinxhyphen{}wave as the time at which the T\sphinxhyphen{}wave’s maximum gradient tangent returns to the
baseline. The baseline is either set to zero, or set to the median value of 30ms prior to the start of the QRS
complex (the value of which has to be passed in the \sphinxtitleref{baseline\_adjust} variable).
\subsubsection*{References}

\end{fulllineitems}

\subsubsection*{Classes}


\begin{savenotes}\sphinxatlongtablestart\begin{longtable}[c]{\X{1}{2}\X{1}{2}}
\hline

\endfirsthead

\multicolumn{2}{c}%
{\makebox[0pt]{\sphinxtablecontinued{\tablename\ \thetable{} \textendash{} continued from previous page}}}\\
\hline

\endhead

\hline
\multicolumn{2}{r}{\makebox[0pt][r]{\sphinxtablecontinued{continues on next page}}}\\
\endfoot

\endlastfoot

\sphinxAtStartPar
{\hyperref[\detokenize{_autosummary/signalanalysis.general.Signal:signalanalysis.general.Signal}]{\sphinxcrossref{\sphinxcode{\sphinxupquote{Signal}}}}}
&
\sphinxAtStartPar
Base class for general signal, either ECG or VCG
\\
\hline
\end{longtable}\sphinxatlongtableend\end{savenotes}


\subsubsection{signalanalysis.general.Signal}
\label{\detokenize{_autosummary/signalanalysis.general.Signal:signalanalysis-general-signal}}\label{\detokenize{_autosummary/signalanalysis.general.Signal::doc}}\index{Signal (class in signalanalysis.general)@\spxentry{Signal}\spxextra{class in signalanalysis.general}}

\begin{fulllineitems}
\phantomsection\label{\detokenize{_autosummary/signalanalysis.general.Signal:signalanalysis.general.Signal}}\pysiglinewithargsret{\sphinxbfcode{\sphinxupquote{class }}\sphinxcode{\sphinxupquote{signalanalysis.general.}}\sphinxbfcode{\sphinxupquote{Signal}}}{\emph{\DUrole{o}{**}\DUrole{n}{kwargs}}}{}
\sphinxAtStartPar
Bases: \sphinxcode{\sphinxupquote{object}}

\sphinxAtStartPar
Base class for general signal, either ECG or VCG
\index{data (signalanalysis.general.Signal attribute)@\spxentry{data}\spxextra{signalanalysis.general.Signal attribute}}

\begin{fulllineitems}
\phantomsection\label{\detokenize{_autosummary/signalanalysis.general.Signal:signalanalysis.general.Signal.data}}\pysigline{\sphinxbfcode{\sphinxupquote{data}}}
\sphinxAtStartPar
Raw ECG data for the different leads
\begin{quote}\begin{description}
\item[{Type}] \leavevmode
\sphinxAtStartPar
pd.DataFrame

\end{description}\end{quote}

\end{fulllineitems}

\index{filename (signalanalysis.general.Signal attribute)@\spxentry{filename}\spxextra{signalanalysis.general.Signal attribute}}

\begin{fulllineitems}
\phantomsection\label{\detokenize{_autosummary/signalanalysis.general.Signal:signalanalysis.general.Signal.filename}}\pysigline{\sphinxbfcode{\sphinxupquote{filename}}}
\sphinxAtStartPar
Filename for the location of the data
\begin{quote}\begin{description}
\item[{Type}] \leavevmode
\sphinxAtStartPar
str

\end{description}\end{quote}

\end{fulllineitems}

\index{normalised (signalanalysis.general.Signal attribute)@\spxentry{normalised}\spxextra{signalanalysis.general.Signal attribute}}

\begin{fulllineitems}
\phantomsection\label{\detokenize{_autosummary/signalanalysis.general.Signal:signalanalysis.general.Signal.normalised}}\pysigline{\sphinxbfcode{\sphinxupquote{normalised}}}
\sphinxAtStartPar
Whether or not the data for the leads have been normalised
\begin{quote}\begin{description}
\item[{Type}] \leavevmode
\sphinxAtStartPar
bool

\end{description}\end{quote}

\end{fulllineitems}

\index{n\_beats (signalanalysis.general.Signal attribute)@\spxentry{n\_beats}\spxextra{signalanalysis.general.Signal attribute}}

\begin{fulllineitems}
\phantomsection\label{\detokenize{_autosummary/signalanalysis.general.Signal:signalanalysis.general.Signal.n_beats}}\pysigline{\sphinxbfcode{\sphinxupquote{n\_beats}}}
\sphinxAtStartPar
Number of beats recorded in the trace. Set to 0 if not calculated
\begin{quote}\begin{description}
\item[{Type}] \leavevmode
\sphinxAtStartPar
int

\end{description}\end{quote}

\end{fulllineitems}

\index{qrs\_start (signalanalysis.general.Signal attribute)@\spxentry{qrs\_start}\spxextra{signalanalysis.general.Signal attribute}}

\begin{fulllineitems}
\phantomsection\label{\detokenize{_autosummary/signalanalysis.general.Signal:signalanalysis.general.Signal.qrs_start}}\pysigline{\sphinxbfcode{\sphinxupquote{qrs\_start}}}
\sphinxAtStartPar
Times calculated for the start of the QRS complex
\begin{quote}\begin{description}
\item[{Type}] \leavevmode
\sphinxAtStartPar
list of float

\end{description}\end{quote}

\end{fulllineitems}

\index{qrs\_end (signalanalysis.general.Signal attribute)@\spxentry{qrs\_end}\spxextra{signalanalysis.general.Signal attribute}}

\begin{fulllineitems}
\phantomsection\label{\detokenize{_autosummary/signalanalysis.general.Signal:signalanalysis.general.Signal.qrs_end}}\pysigline{\sphinxbfcode{\sphinxupquote{qrs\_end}}}
\sphinxAtStartPar
Times calculated for the end of the QRS complex
\begin{quote}\begin{description}
\item[{Type}] \leavevmode
\sphinxAtStartPar
end

\end{description}\end{quote}

\end{fulllineitems}

\index{data\_source (signalanalysis.general.Signal attribute)@\spxentry{data\_source}\spxextra{signalanalysis.general.Signal attribute}}

\begin{fulllineitems}
\phantomsection\label{\detokenize{_autosummary/signalanalysis.general.Signal:signalanalysis.general.Signal.data_source}}\pysigline{\sphinxbfcode{\sphinxupquote{data\_source}}}
\sphinxAtStartPar
Source for the data, if known e.g. Staff III database, CARP simulation, etc.
\begin{quote}\begin{description}
\item[{Type}] \leavevmode
\sphinxAtStartPar
str

\end{description}\end{quote}

\end{fulllineitems}

\index{comments (signalanalysis.general.Signal attribute)@\spxentry{comments}\spxextra{signalanalysis.general.Signal attribute}}

\begin{fulllineitems}
\phantomsection\label{\detokenize{_autosummary/signalanalysis.general.Signal:signalanalysis.general.Signal.comments}}\pysigline{\sphinxbfcode{\sphinxupquote{comments}}}
\sphinxAtStartPar
Any further details known about the data, e.g. sex, age, etc.
\begin{quote}\begin{description}
\item[{Type}] \leavevmode
\sphinxAtStartPar
str

\end{description}\end{quote}

\end{fulllineitems}

\index{get\_rms() (signalanalysis.general.Signal method)@\spxentry{get\_rms()}\spxextra{signalanalysis.general.Signal method}}

\begin{fulllineitems}
\phantomsection\label{\detokenize{_autosummary/signalanalysis.general.Signal:signalanalysis.general.Signal.get_rms}}\pysiglinewithargsret{\sphinxbfcode{\sphinxupquote{get\_rms}}}{\emph{\DUrole{n}{unipolar\_only}\DUrole{o}{=}\DUrole{default_value}{True}}}{}
\sphinxAtStartPar
Returns the RMS of the combined signal

\end{fulllineitems}

\index{get\_n\_beats() (signalanalysis.general.Signal method)@\spxentry{get\_n\_beats()}\spxextra{signalanalysis.general.Signal method}}

\begin{fulllineitems}
\phantomsection\label{\detokenize{_autosummary/signalanalysis.general.Signal:signalanalysis.general.Signal.get_n_beats}}\pysiglinewithargsret{\sphinxbfcode{\sphinxupquote{get\_n\_beats}}}{\emph{threshold=0.5}, \emph{separation=0.2}, \emph{plot=False}, \emph{\textbackslash{}*\textbackslash{}*kwargs}}{}
\sphinxAtStartPar
Splits the full signal into individual beats

\end{fulllineitems}

\index{plot() (signalanalysis.general.Signal method)@\spxentry{plot()}\spxextra{signalanalysis.general.Signal method}}

\begin{fulllineitems}
\phantomsection\label{\detokenize{_autosummary/signalanalysis.general.Signal:signalanalysis.general.Signal.plot}}\pysiglinewithargsret{\sphinxbfcode{\sphinxupquote{plot}}}{\emph{\DUrole{n}{separate\_beats}\DUrole{o}{=}\DUrole{default_value}{False}}}{}
\sphinxAtStartPar
Plot the ECG data

\end{fulllineitems}

\subsubsection*{Methods}


\begin{savenotes}\sphinxatlongtablestart\begin{longtable}[c]{\X{1}{2}\X{1}{2}}
\hline

\endfirsthead

\multicolumn{2}{c}%
{\makebox[0pt]{\sphinxtablecontinued{\tablename\ \thetable{} \textendash{} continued from previous page}}}\\
\hline

\endhead

\hline
\multicolumn{2}{r}{\makebox[0pt][r]{\sphinxtablecontinued{continues on next page}}}\\
\endfoot

\endlastfoot

\sphinxAtStartPar
{\hyperref[\detokenize{_autosummary/signalanalysis.general.Signal:signalanalysis.general.Signal.apply_filter}]{\sphinxcrossref{\sphinxcode{\sphinxupquote{apply\_filter}}}}}
&
\sphinxAtStartPar
Apply a given filter to the data, using their respective arguments as required
\\
\hline
\sphinxAtStartPar
{\hyperref[\detokenize{_autosummary/signalanalysis.general.Signal:id0}]{\sphinxcrossref{\sphinxcode{\sphinxupquote{get\_n\_beats}}}}}
&
\sphinxAtStartPar
Calculate the number of beats in a given signal, and save the individual beats to the object for later use
\\
\hline
\sphinxAtStartPar
{\hyperref[\detokenize{_autosummary/signalanalysis.general.Signal:id1}]{\sphinxcrossref{\sphinxcode{\sphinxupquote{get\_rms}}}}}
&
\sphinxAtStartPar
Returns the RMS of the combined signal
\\
\hline
\sphinxAtStartPar
\sphinxcode{\sphinxupquote{get\_twave\_end}}
&
\sphinxAtStartPar

\\
\hline
\sphinxAtStartPar
{\hyperref[\detokenize{_autosummary/signalanalysis.general.Signal:signalanalysis.general.Signal.reset}]{\sphinxcrossref{\sphinxcode{\sphinxupquote{reset}}}}}
&
\sphinxAtStartPar
Reset all properties of the class
\\
\hline
\end{longtable}\sphinxatlongtableend\end{savenotes}
\index{apply\_filter() (signalanalysis.general.Signal method)@\spxentry{apply\_filter()}\spxextra{signalanalysis.general.Signal method}}

\begin{fulllineitems}
\phantomsection\label{\detokenize{_autosummary/signalanalysis.general.Signal:signalanalysis.general.Signal.apply_filter}}\pysiglinewithargsret{\sphinxbfcode{\sphinxupquote{apply\_filter}}}{\emph{\DUrole{o}{**}\DUrole{n}{kwargs}}}{}
\sphinxAtStartPar
Apply a given filter to the data, using their respective arguments as required


\sphinxstrong{See also:}
\nopagebreak

\begin{description}
\item[{{\hyperref[\detokenize{_autosummary/tools.maths.filter_butterworth:tools.maths.filter_butterworth}]{\sphinxcrossref{\sphinxcode{\sphinxupquote{tools.maths.filter\_butterworth()}}}}}}] \leavevmode
\sphinxAtStartPar
Potential filter

\item[{{\hyperref[\detokenize{_autosummary/tools.maths.filter_savitzkygolay:tools.maths.filter_savitzkygolay}]{\sphinxcrossref{\sphinxcode{\sphinxupquote{tools.maths.filter\_savitzkygolay()}}}}}}] \leavevmode
\sphinxAtStartPar
Potential filter

\end{description}



\end{fulllineitems}

\index{get\_n\_beats() (signalanalysis.general.Signal method)@\spxentry{get\_n\_beats()}\spxextra{signalanalysis.general.Signal method}}

\begin{fulllineitems}
\phantomsection\label{\detokenize{_autosummary/signalanalysis.general.Signal:id0}}\pysiglinewithargsret{\sphinxbfcode{\sphinxupquote{get\_n\_beats}}}{\emph{\DUrole{n}{threshold}\DUrole{p}{:} \DUrole{n}{float} \DUrole{o}{=} \DUrole{default_value}{0.5}}, \emph{\DUrole{n}{min\_separation}\DUrole{p}{:} \DUrole{n}{float} \DUrole{o}{=} \DUrole{default_value}{0.2}}, \emph{\DUrole{n}{reset\_index}\DUrole{p}{:} \DUrole{n}{bool} \DUrole{o}{=} \DUrole{default_value}{True}}, \emph{\DUrole{n}{plot}\DUrole{p}{:} \DUrole{n}{bool} \DUrole{o}{=} \DUrole{default_value}{False}}, \emph{\DUrole{o}{**}\DUrole{n}{kwargs}}}{}
\sphinxAtStartPar
Calculate the number of beats in a given signal, and save the individual beats to the object for later use

\sphinxAtStartPar
When given the raw data of a given signal (ECG or VCG), will estimate the number of beats recorded in the trace
based on the RMS of the signal exceeding a threshold value. The estimated individual beats will then be saved in
a list in a lossless manner, i.e. saved as {[}ECG1, ECG2, …, ECG(n){]}, where ECG1={[}0:peak2{]}, ECG2={[}peak1:peak3{]},
…, ECGn={[}peak(n\sphinxhyphen{}1):end{]}
\begin{quote}\begin{description}
\item[{Parameters}] \leavevmode\begin{itemize}
\item {} 
\sphinxAtStartPar
\sphinxstyleliteralstrong{\sphinxupquote{threshold}} (\sphinxstyleliteralemphasis{\sphinxupquote{float \{0\textless{}1\}}}) \textendash{} Minimum value to search for for a peak in RMS signal to determine when a beat has occurred, default=0.5

\item {} 
\sphinxAtStartPar
\sphinxstyleliteralstrong{\sphinxupquote{min\_separation}} (\sphinxstyleliteralemphasis{\sphinxupquote{float}}) \textendash{} Minimum time (in s) that should be used to separate separate beats, default=0.2s

\item {} 
\sphinxAtStartPar
\sphinxstyleliteralstrong{\sphinxupquote{reset\_index}} (\sphinxstyleliteralemphasis{\sphinxupquote{bool}}) \textendash{} Whether to reset the time index for the separated beats so that they all start from zero (true),
or whether to leave them with the original time index (false), default=True

\item {} 
\sphinxAtStartPar
\sphinxstyleliteralstrong{\sphinxupquote{plot}} (\sphinxstyleliteralemphasis{\sphinxupquote{bool}}) \textendash{} Whether to plot results of beat detection, default=False

\item {} 
\sphinxAtStartPar
\sphinxstyleliteralstrong{\sphinxupquote{unipolar\_only}} (\sphinxstyleliteralemphasis{\sphinxupquote{bool}}\sphinxstyleliteralemphasis{\sphinxupquote{, }}\sphinxstyleliteralemphasis{\sphinxupquote{optional}}) \textendash{} Only appropriate for ECG data. Whether to use only unipolar ECG leads to calculate RMS, default=True

\end{itemize}

\item[{Returns}] \leavevmode
\sphinxAtStartPar
\sphinxstylestrong{self.n\_beats} \textendash{} Number of beats detected in signal

\item[{Return type}] \leavevmode
\sphinxAtStartPar
int

\end{description}\end{quote}


\sphinxstrong{See also:}
\nopagebreak

\begin{description}
\item[{{\hyperref[\detokenize{_autosummary/signalanalysis.general.Signal:id1}]{\sphinxcrossref{\sphinxcode{\sphinxupquote{signalanalysis.general.Signal.get\_rms()}}}}}}] \leavevmode
\sphinxAtStartPar
RMS signal calculation required for getting n\_beats

\end{description}



\end{fulllineitems}

\index{get\_rms() (signalanalysis.general.Signal method)@\spxentry{get\_rms()}\spxextra{signalanalysis.general.Signal method}}

\begin{fulllineitems}
\phantomsection\label{\detokenize{_autosummary/signalanalysis.general.Signal:id1}}\pysiglinewithargsret{\sphinxbfcode{\sphinxupquote{get\_rms}}}{\emph{\DUrole{n}{preprocess\_data}\DUrole{p}{:} \DUrole{n}{Optional\DUrole{p}{{[}}pandas.core.frame.DataFrame\DUrole{p}{{]}}} \DUrole{o}{=} \DUrole{default_value}{None}}, \emph{\DUrole{n}{drop\_columns}\DUrole{p}{:} \DUrole{n}{Optional\DUrole{p}{{[}}List\DUrole{p}{{[}}str\DUrole{p}{{]}}\DUrole{p}{{]}}} \DUrole{o}{=} \DUrole{default_value}{None}}}{}
\sphinxAtStartPar
Returns the RMS of the combined signal
\begin{quote}\begin{description}
\item[{Parameters}] \leavevmode\begin{itemize}
\item {} 
\sphinxAtStartPar
\sphinxstyleliteralstrong{\sphinxupquote{preprocess\_data}} (\sphinxstyleliteralemphasis{\sphinxupquote{pd.DataFrame}}\sphinxstyleliteralemphasis{\sphinxupquote{, }}\sphinxstyleliteralemphasis{\sphinxupquote{optional}}) \textendash{} Only passed if there is some extant data that is to be used for getting the RMS (for example,
if the unipolar data only from ECG is being used, and the data is thus preprocessed in a manner specific
for ECG data in the ECG routine)

\item {} 
\sphinxAtStartPar
\sphinxstyleliteralstrong{\sphinxupquote{drop\_columns}} (\sphinxstyleliteralemphasis{\sphinxupquote{list of str}}\sphinxstyleliteralemphasis{\sphinxupquote{, }}\sphinxstyleliteralemphasis{\sphinxupquote{optional}}) \textendash{} List of any columns to drop from the raw data before calculating the RMS. Can be used in conjunction with
preprocess\_data

\end{itemize}

\item[{Returns}] \leavevmode
\sphinxAtStartPar
\sphinxstylestrong{self.rms} \textendash{} RMS signal of all combined leads

\item[{Return type}] \leavevmode
\sphinxAtStartPar
pd.Series

\end{description}\end{quote}
\subsubsection*{Notes}

\sphinxAtStartPar
The scalar RMS is calculated according to
\begin{equation*}
\begin{split}\sqrt{ \frac{1}{n}\sum_{i=1}^n (\textnormal{ECG}_i^2(t)) }\end{split}
\end{equation*}
\sphinxAtStartPar
for all leads available from the signal (12 for ECG, 3 for VCG), unless some leads are excluded via the
drop\_columns parameter.

\end{fulllineitems}

\index{reset() (signalanalysis.general.Signal method)@\spxentry{reset()}\spxextra{signalanalysis.general.Signal method}}

\begin{fulllineitems}
\phantomsection\label{\detokenize{_autosummary/signalanalysis.general.Signal:signalanalysis.general.Signal.reset}}\pysiglinewithargsret{\sphinxbfcode{\sphinxupquote{reset}}}{}{}
\sphinxAtStartPar
Reset all properties of the class

\sphinxAtStartPar
Function called when reading in new data into an existing class (for some reason), which would make these
properties and attributes clash with the other data

\end{fulllineitems}


\end{fulllineitems}



\subsection{signalanalysis.vcg}
\label{\detokenize{_autosummary/signalanalysis.vcg:module-signalanalysis.vcg}}\label{\detokenize{_autosummary/signalanalysis.vcg:signalanalysis-vcg}}\label{\detokenize{_autosummary/signalanalysis.vcg::doc}}\index{module@\spxentry{module}!signalanalysis.vcg@\spxentry{signalanalysis.vcg}}\index{signalanalysis.vcg@\spxentry{signalanalysis.vcg}!module@\spxentry{module}}\subsubsection*{Functions}


\begin{savenotes}\sphinxatlongtablestart\begin{longtable}[c]{\X{1}{2}\X{1}{2}}
\hline

\endfirsthead

\multicolumn{2}{c}%
{\makebox[0pt]{\sphinxtablecontinued{\tablename\ \thetable{} \textendash{} continued from previous page}}}\\
\hline

\endhead

\hline
\multicolumn{2}{r}{\makebox[0pt][r]{\sphinxtablecontinued{continues on next page}}}\\
\endfoot

\endlastfoot

\sphinxAtStartPar
{\hyperref[\detokenize{_autosummary/signalanalysis.vcg.calculate_delta_dipole_angle:signalanalysis.vcg.calculate_delta_dipole_angle}]{\sphinxcrossref{\sphinxcode{\sphinxupquote{calculate\_delta\_dipole\_angle}}}}}
&
\sphinxAtStartPar
Calculates the angular difference between two VCGs based on difference in azimuthal and elevation angles.
\\
\hline
\sphinxAtStartPar
{\hyperref[\detokenize{_autosummary/signalanalysis.vcg.compare_dipole_angles:signalanalysis.vcg.compare_dipole_angles}]{\sphinxcrossref{\sphinxcode{\sphinxupquote{compare\_dipole\_angles}}}}}
&
\sphinxAtStartPar
Calculates the angular differences between two VCGs at multiple points during their evolution
\\
\hline
\sphinxAtStartPar
{\hyperref[\detokenize{_autosummary/signalanalysis.vcg.get_azimuth_elevation:signalanalysis.vcg.get_azimuth_elevation}]{\sphinxcrossref{\sphinxcode{\sphinxupquote{get\_azimuth\_elevation}}}}}
&
\sphinxAtStartPar
Calculate azimuth and elevation angles for a specified section of the VCG.
\\
\hline
\sphinxAtStartPar
{\hyperref[\detokenize{_autosummary/signalanalysis.vcg.get_dipole_magnitudes:signalanalysis.vcg.get_dipole_magnitudes}]{\sphinxcrossref{\sphinxcode{\sphinxupquote{get\_dipole\_magnitudes}}}}}
&
\sphinxAtStartPar
Calculates metrics relating to the magnitude of the weighted dipole of the VCG
\\
\hline
\sphinxAtStartPar
{\hyperref[\detokenize{_autosummary/signalanalysis.vcg.get_qrs_start_end:signalanalysis.vcg.get_qrs_start_end}]{\sphinxcrossref{\sphinxcode{\sphinxupquote{get\_qrs\_start\_end}}}}}
&
\sphinxAtStartPar
Calculate the extent of the VCG QRS complex on the basis of max derivative
\\
\hline
\sphinxAtStartPar
{\hyperref[\detokenize{_autosummary/signalanalysis.vcg.get_single_vcg_azimuth_elevation:signalanalysis.vcg.get_single_vcg_azimuth_elevation}]{\sphinxcrossref{\sphinxcode{\sphinxupquote{get\_single\_vcg\_azimuth\_elevation}}}}}
&
\sphinxAtStartPar
Get the azimuth and elevation data for a single VCG trace, along with the average dipole magnitude.
\\
\hline
\sphinxAtStartPar
{\hyperref[\detokenize{_autosummary/signalanalysis.vcg.get_spatial_velocity:signalanalysis.vcg.get_spatial_velocity}]{\sphinxcrossref{\sphinxcode{\sphinxupquote{get\_spatial\_velocity}}}}}
&
\sphinxAtStartPar
Calculate spatial velocity
\\
\hline
\sphinxAtStartPar
{\hyperref[\detokenize{_autosummary/signalanalysis.vcg.get_vcg_area:signalanalysis.vcg.get_vcg_area}]{\sphinxcrossref{\sphinxcode{\sphinxupquote{get\_vcg\_area}}}}}
&
\sphinxAtStartPar
Calculate area under VCG curve for a given section (e.g.
\\
\hline
\sphinxAtStartPar
{\hyperref[\detokenize{_autosummary/signalanalysis.vcg.get_vcg_from_ecg:signalanalysis.vcg.get_vcg_from_ecg}]{\sphinxcrossref{\sphinxcode{\sphinxupquote{get\_vcg\_from\_ecg}}}}}
&
\sphinxAtStartPar
Convert ECG data to vectorcardiogram (VCG) data using the Kors matrix method
\\
\hline
\sphinxAtStartPar
{\hyperref[\detokenize{_autosummary/signalanalysis.vcg.get_weighted_dipole_angles:signalanalysis.vcg.get_weighted_dipole_angles}]{\sphinxcrossref{\sphinxcode{\sphinxupquote{get\_weighted\_dipole\_angles}}}}}
&
\sphinxAtStartPar
Calculate metrics relating to the angles of the weighted dipole of the VCG.
\\
\hline
\sphinxAtStartPar
{\hyperref[\detokenize{_autosummary/signalanalysis.vcg.plot_density_effect:signalanalysis.vcg.plot_density_effect}]{\sphinxcrossref{\sphinxcode{\sphinxupquote{plot\_density\_effect}}}}}
&
\sphinxAtStartPar
Plot the effect of density on metrics.
\\
\hline
\sphinxAtStartPar
{\hyperref[\detokenize{_autosummary/signalanalysis.vcg.plot_metric_change:signalanalysis.vcg.plot_metric_change}]{\sphinxcrossref{\sphinxcode{\sphinxupquote{plot\_metric\_change}}}}}
&
\sphinxAtStartPar
Function to plot all the various figures for trend analysis in one go.
\\
\hline
\sphinxAtStartPar
{\hyperref[\detokenize{_autosummary/signalanalysis.vcg.plot_metric_change_barplot:signalanalysis.vcg.plot_metric_change_barplot}]{\sphinxcrossref{\sphinxcode{\sphinxupquote{plot\_metric\_change\_barplot}}}}}
&
\sphinxAtStartPar
Plots a bar chart for the observed metrics.
\\
\hline
\end{longtable}\sphinxatlongtableend\end{savenotes}


\subsubsection{signalanalysis.vcg.calculate\_delta\_dipole\_angle}
\label{\detokenize{_autosummary/signalanalysis.vcg.calculate_delta_dipole_angle:signalanalysis-vcg-calculate-delta-dipole-angle}}\label{\detokenize{_autosummary/signalanalysis.vcg.calculate_delta_dipole_angle::doc}}\index{calculate\_delta\_dipole\_angle() (in module signalanalysis.vcg)@\spxentry{calculate\_delta\_dipole\_angle()}\spxextra{in module signalanalysis.vcg}}

\begin{fulllineitems}
\phantomsection\label{\detokenize{_autosummary/signalanalysis.vcg.calculate_delta_dipole_angle:signalanalysis.vcg.calculate_delta_dipole_angle}}\pysiglinewithargsret{\sphinxcode{\sphinxupquote{signalanalysis.vcg.}}\sphinxbfcode{\sphinxupquote{calculate\_delta\_dipole\_angle}}}{\emph{\DUrole{n}{azimuth1}\DUrole{p}{:} \DUrole{n}{List\DUrole{p}{{[}}float\DUrole{p}{{]}}}}, \emph{\DUrole{n}{elevation1}\DUrole{p}{:} \DUrole{n}{List\DUrole{p}{{[}}float\DUrole{p}{{]}}}}, \emph{\DUrole{n}{azimuth2}\DUrole{p}{:} \DUrole{n}{List\DUrole{p}{{[}}float\DUrole{p}{{]}}}}, \emph{\DUrole{n}{elevation2}\DUrole{p}{:} \DUrole{n}{List\DUrole{p}{{[}}float\DUrole{p}{{]}}}}, \emph{\DUrole{n}{convert\_to\_degrees}\DUrole{p}{:} \DUrole{n}{bool} \DUrole{o}{=} \DUrole{default_value}{False}}}{{ $\rightarrow$ List\DUrole{p}{{[}}float\DUrole{p}{{]}}}}
\sphinxAtStartPar
Calculates the angular difference between two VCGs based on difference in azimuthal and elevation angles.

\sphinxAtStartPar
Useful for calculating difference between weighted averages.
\begin{quote}\begin{description}
\item[{Parameters}] \leavevmode\begin{itemize}
\item {} 
\sphinxAtStartPar
\sphinxstyleliteralstrong{\sphinxupquote{azimuth1}} (\sphinxstyleliteralemphasis{\sphinxupquote{list of float}}) \textendash{} Azimuth angles for the first dipole

\item {} 
\sphinxAtStartPar
\sphinxstyleliteralstrong{\sphinxupquote{elevation1}} (\sphinxstyleliteralemphasis{\sphinxupquote{list of float}}) \textendash{} Elevation angles for the first dipole

\item {} 
\sphinxAtStartPar
\sphinxstyleliteralstrong{\sphinxupquote{azimuth2}} (\sphinxstyleliteralemphasis{\sphinxupquote{list of float}}) \textendash{} Azimuth angles for the second dipole

\item {} 
\sphinxAtStartPar
\sphinxstyleliteralstrong{\sphinxupquote{elevation2}} (\sphinxstyleliteralemphasis{\sphinxupquote{list of float}}) \textendash{} Elevation angles for the second dipole

\item {} 
\sphinxAtStartPar
\sphinxstyleliteralstrong{\sphinxupquote{convert\_to\_degrees}} (\sphinxstyleliteralemphasis{\sphinxupquote{bool}}\sphinxstyleliteralemphasis{\sphinxupquote{, }}\sphinxstyleliteralemphasis{\sphinxupquote{optional}}) \textendash{} Whether to convert the angle from radians to degrees, default=False

\end{itemize}

\item[{Returns}] \leavevmode
\sphinxAtStartPar
\sphinxstylestrong{dt} \textendash{} List of angles between a series of dipoles, either in radians (default) or degrees depending on input argument

\item[{Return type}] \leavevmode
\sphinxAtStartPar
list of float

\end{description}\end{quote}

\end{fulllineitems}



\subsubsection{signalanalysis.vcg.compare\_dipole\_angles}
\label{\detokenize{_autosummary/signalanalysis.vcg.compare_dipole_angles:signalanalysis-vcg-compare-dipole-angles}}\label{\detokenize{_autosummary/signalanalysis.vcg.compare_dipole_angles::doc}}\index{compare\_dipole\_angles() (in module signalanalysis.vcg)@\spxentry{compare\_dipole\_angles()}\spxextra{in module signalanalysis.vcg}}

\begin{fulllineitems}
\phantomsection\label{\detokenize{_autosummary/signalanalysis.vcg.compare_dipole_angles:signalanalysis.vcg.compare_dipole_angles}}\pysiglinewithargsret{\sphinxcode{\sphinxupquote{signalanalysis.vcg.}}\sphinxbfcode{\sphinxupquote{compare\_dipole\_angles}}}{\emph{\DUrole{n}{vcg1}\DUrole{p}{:} \DUrole{n}{pandas.core.frame.DataFrame}}, \emph{\DUrole{n}{vcg2}\DUrole{p}{:} \DUrole{n}{pandas.core.frame.DataFrame}}, \emph{\DUrole{n}{t\_start1}\DUrole{p}{:} \DUrole{n}{float} \DUrole{o}{=} \DUrole{default_value}{0}}, \emph{\DUrole{n}{t\_end1}\DUrole{p}{:} \DUrole{n}{Optional\DUrole{p}{{[}}float\DUrole{p}{{]}}} \DUrole{o}{=} \DUrole{default_value}{None}}, \emph{\DUrole{n}{t\_start2}\DUrole{p}{:} \DUrole{n}{float} \DUrole{o}{=} \DUrole{default_value}{0}}, \emph{\DUrole{n}{t\_end2}\DUrole{p}{:} \DUrole{n}{Optional\DUrole{p}{{[}}float\DUrole{p}{{]}}} \DUrole{o}{=} \DUrole{default_value}{None}}, \emph{\DUrole{n}{n\_compare}\DUrole{p}{:} \DUrole{n}{int} \DUrole{o}{=} \DUrole{default_value}{10}}, \emph{\DUrole{n}{convert\_to\_degrees}\DUrole{p}{:} \DUrole{n}{bool} \DUrole{o}{=} \DUrole{default_value}{False}}, \emph{\DUrole{n}{matlab\_match}\DUrole{p}{:} \DUrole{n}{bool} \DUrole{o}{=} \DUrole{default_value}{False}}}{{ $\rightarrow$ List\DUrole{p}{{[}}float\DUrole{p}{{]}}}}
\sphinxAtStartPar
Calculates the angular differences between two VCGs at multiple points during their evolution

\sphinxAtStartPar
To compensate for the fact that the two VCG traces may not be of the same length, the comparison does not occur
at every moment of the VCG; rather, the dipoles are calculated for certain fractional points during the VCG.
\begin{quote}\begin{description}
\item[{Parameters}] \leavevmode\begin{itemize}
\item {} 
\sphinxAtStartPar
\sphinxstyleliteralstrong{\sphinxupquote{vcg1}} (\sphinxstyleliteralemphasis{\sphinxupquote{pd.DataFrame}}) \textendash{} First VCG trace to consider

\item {} 
\sphinxAtStartPar
\sphinxstyleliteralstrong{\sphinxupquote{vcg2}} (\sphinxstyleliteralemphasis{\sphinxupquote{pd.DataFrame}}) \textendash{} Second VCG trace to consider

\item {} 
\sphinxAtStartPar
\sphinxstyleliteralstrong{\sphinxupquote{t\_start1}} (\sphinxstyleliteralemphasis{\sphinxupquote{float}}\sphinxstyleliteralemphasis{\sphinxupquote{, }}\sphinxstyleliteralemphasis{\sphinxupquote{optional}}) \textendash{} Time from which to consider the data from the first VCG trace, default=0

\item {} 
\sphinxAtStartPar
\sphinxstyleliteralstrong{\sphinxupquote{t\_end1}} (\sphinxstyleliteralemphasis{\sphinxupquote{float}}\sphinxstyleliteralemphasis{\sphinxupquote{, }}\sphinxstyleliteralemphasis{\sphinxupquote{optional}}) \textendash{} Time until which to consider the data from the first VCG trace, default=end

\item {} 
\sphinxAtStartPar
\sphinxstyleliteralstrong{\sphinxupquote{t\_start2}} (\sphinxstyleliteralemphasis{\sphinxupquote{float}}\sphinxstyleliteralemphasis{\sphinxupquote{, }}\sphinxstyleliteralemphasis{\sphinxupquote{optional}}) \textendash{} Time from which to consider the data from the second VCG trace, default=0

\item {} 
\sphinxAtStartPar
\sphinxstyleliteralstrong{\sphinxupquote{t\_end2}} (\sphinxstyleliteralemphasis{\sphinxupquote{float}}\sphinxstyleliteralemphasis{\sphinxupquote{, }}\sphinxstyleliteralemphasis{\sphinxupquote{optional}}) \textendash{} Time until which to consider the data from the second VCG trace, default=end

\item {} 
\sphinxAtStartPar
\sphinxstyleliteralstrong{\sphinxupquote{n\_compare}} (\sphinxstyleliteralemphasis{\sphinxupquote{int}}\sphinxstyleliteralemphasis{\sphinxupquote{, }}\sphinxstyleliteralemphasis{\sphinxupquote{optional}}) \textendash{} Number of points during the VCGs at which to calculate the dipole angle. If set to \sphinxhyphen{}1, will calculate at
every point during the VCG, but requires VCG traces to be the same length, default=10

\item {} 
\sphinxAtStartPar
\sphinxstyleliteralstrong{\sphinxupquote{convert\_to\_degrees}} (\sphinxstyleliteralemphasis{\sphinxupquote{bool}}\sphinxstyleliteralemphasis{\sphinxupquote{, }}\sphinxstyleliteralemphasis{\sphinxupquote{optional}}) \textendash{} Whether to convert the angles from radians to degrees, default=False

\item {} 
\sphinxAtStartPar
\sphinxstyleliteralstrong{\sphinxupquote{matlab\_match}} (\sphinxstyleliteralemphasis{\sphinxupquote{bool}}\sphinxstyleliteralemphasis{\sphinxupquote{, }}\sphinxstyleliteralemphasis{\sphinxupquote{optional}}) \textendash{} Whether to extract the data segment to match Matlab output or to use simpler Python, default=False

\end{itemize}

\item[{Returns}] \leavevmode
\sphinxAtStartPar
\sphinxstylestrong{dt} \textendash{} Angle between two given VCGs at n points during the VCG, where n is given as input

\item[{Return type}] \leavevmode
\sphinxAtStartPar
list of float

\end{description}\end{quote}

\end{fulllineitems}



\subsubsection{signalanalysis.vcg.get\_azimuth\_elevation}
\label{\detokenize{_autosummary/signalanalysis.vcg.get_azimuth_elevation:signalanalysis-vcg-get-azimuth-elevation}}\label{\detokenize{_autosummary/signalanalysis.vcg.get_azimuth_elevation::doc}}\index{get\_azimuth\_elevation() (in module signalanalysis.vcg)@\spxentry{get\_azimuth\_elevation()}\spxextra{in module signalanalysis.vcg}}

\begin{fulllineitems}
\phantomsection\label{\detokenize{_autosummary/signalanalysis.vcg.get_azimuth_elevation:signalanalysis.vcg.get_azimuth_elevation}}\pysiglinewithargsret{\sphinxcode{\sphinxupquote{signalanalysis.vcg.}}\sphinxbfcode{\sphinxupquote{get\_azimuth\_elevation}}}{\emph{\DUrole{n}{vcgs}\DUrole{p}{:} \DUrole{n}{Union\DUrole{p}{{[}}List\DUrole{p}{{[}}pandas.core.frame.DataFrame\DUrole{p}{{]}}\DUrole{p}{, }pandas.core.frame.DataFrame\DUrole{p}{{]}}}}, \emph{\DUrole{n}{t\_start}\DUrole{p}{:} \DUrole{n}{Optional\DUrole{p}{{[}}List\DUrole{p}{{[}}float\DUrole{p}{{]}}\DUrole{p}{{]}}} \DUrole{o}{=} \DUrole{default_value}{None}}, \emph{\DUrole{n}{t\_end}\DUrole{p}{:} \DUrole{n}{Optional\DUrole{p}{{[}}List\DUrole{p}{{[}}float\DUrole{p}{{]}}\DUrole{p}{{]}}} \DUrole{o}{=} \DUrole{default_value}{None}}}{{ $\rightarrow$ Tuple\DUrole{p}{{[}}List\DUrole{p}{{[}}Iterable\DUrole{p}{{[}}float\DUrole{p}{{]}}\DUrole{p}{{]}}\DUrole{p}{, }List\DUrole{p}{{[}}Iterable\DUrole{p}{{[}}float\DUrole{p}{{]}}\DUrole{p}{{]}}\DUrole{p}{{]}}}}
\sphinxAtStartPar
Calculate azimuth and elevation angles for a specified section of the VCG.

\sphinxAtStartPar
Will calculate the azimuth and elevation angles for the VCG at each recorded point, potentially within specified
limits (e.g. start/end of QRS)
\begin{quote}\begin{description}
\item[{Parameters}] \leavevmode\begin{itemize}
\item {} 
\sphinxAtStartPar
\sphinxstyleliteralstrong{\sphinxupquote{vcgs}} (\sphinxstyleliteralemphasis{\sphinxupquote{pd.DataFrame}}\sphinxstyleliteralemphasis{\sphinxupquote{ or }}\sphinxstyleliteralemphasis{\sphinxupquote{list of pd.DataFrame}}) \textendash{} VCG data to calculate

\item {} 
\sphinxAtStartPar
\sphinxstyleliteralstrong{\sphinxupquote{t\_start}} (\sphinxstyleliteralemphasis{\sphinxupquote{list of float}}\sphinxstyleliteralemphasis{\sphinxupquote{, }}\sphinxstyleliteralemphasis{\sphinxupquote{optional}}) \textendash{} Start time from which to calculate the angles, default=0

\item {} 
\sphinxAtStartPar
\sphinxstyleliteralstrong{\sphinxupquote{t\_end}} (\sphinxstyleliteralemphasis{\sphinxupquote{list of float}}\sphinxstyleliteralemphasis{\sphinxupquote{, }}\sphinxstyleliteralemphasis{\sphinxupquote{optional}}) \textendash{} End time until which to calculate the angles, default=end

\end{itemize}

\item[{Returns}] \leavevmode
\sphinxAtStartPar
\begin{itemize}
\item {} 
\sphinxAtStartPar
\sphinxstylestrong{azimuth} (\sphinxstyleemphasis{list of list of float}) \textendash{} List (one entry for each passed VCG) of azimuth angles (in radians) for the dipole for every time point during
the specified range

\item {} 
\sphinxAtStartPar
\sphinxstylestrong{elevation} (\sphinxstyleemphasis{list of list of float}) \textendash{} List (one entry for each passed VCG) of elevation angles (in radians) for the dipole for every time point during
the specified range

\end{itemize}


\end{description}\end{quote}

\end{fulllineitems}



\subsubsection{signalanalysis.vcg.get\_dipole\_magnitudes}
\label{\detokenize{_autosummary/signalanalysis.vcg.get_dipole_magnitudes:signalanalysis-vcg-get-dipole-magnitudes}}\label{\detokenize{_autosummary/signalanalysis.vcg.get_dipole_magnitudes::doc}}\index{get\_dipole\_magnitudes() (in module signalanalysis.vcg)@\spxentry{get\_dipole\_magnitudes()}\spxextra{in module signalanalysis.vcg}}

\begin{fulllineitems}
\phantomsection\label{\detokenize{_autosummary/signalanalysis.vcg.get_dipole_magnitudes:signalanalysis.vcg.get_dipole_magnitudes}}\pysiglinewithargsret{\sphinxcode{\sphinxupquote{signalanalysis.vcg.}}\sphinxbfcode{\sphinxupquote{get\_dipole\_magnitudes}}}{\emph{\DUrole{n}{vcgs}\DUrole{p}{:} \DUrole{n}{Union\DUrole{p}{{[}}List\DUrole{p}{{[}}pandas.core.frame.DataFrame\DUrole{p}{{]}}\DUrole{p}{, }pandas.core.frame.DataFrame\DUrole{p}{{]}}}}, \emph{\DUrole{n}{t\_start}\DUrole{p}{:} \DUrole{n}{Union\DUrole{p}{{[}}float\DUrole{p}{, }List\DUrole{p}{{[}}float\DUrole{p}{{]}}\DUrole{p}{{]}}} \DUrole{o}{=} \DUrole{default_value}{0}}, \emph{\DUrole{n}{t\_end}\DUrole{p}{:} \DUrole{n}{Union\DUrole{p}{{[}}float\DUrole{p}{, }List\DUrole{p}{{[}}float\DUrole{p}{{]}}\DUrole{p}{{]}}} \DUrole{o}{=} \DUrole{default_value}{\sphinxhyphen{} 1}}}{{ $\rightarrow$ Tuple\DUrole{p}{{[}}List\DUrole{p}{{[}}numpy.ndarray\DUrole{p}{{]}}\DUrole{p}{, }List\DUrole{p}{{[}}float\DUrole{p}{{]}}\DUrole{p}{, }List\DUrole{p}{{[}}float\DUrole{p}{{]}}\DUrole{p}{, }List\DUrole{p}{{[}}List\DUrole{p}{{[}}float\DUrole{p}{{]}}\DUrole{p}{{]}}\DUrole{p}{, }List\DUrole{p}{{]}}}}
\sphinxAtStartPar
Calculates metrics relating to the magnitude of the weighted dipole of the VCG

\sphinxAtStartPar
Returns the mean weighted dipole, maximum dipole magnitude,(x,y.z) components of the maximum dipole and the time
at which the maximum dipole occurs
\begin{quote}\begin{description}
\item[{Parameters}] \leavevmode\begin{itemize}
\item {} 
\sphinxAtStartPar
\sphinxstyleliteralstrong{\sphinxupquote{vcgs}} (\sphinxstyleliteralemphasis{\sphinxupquote{pd.DataFrame}}\sphinxstyleliteralemphasis{\sphinxupquote{ or }}\sphinxstyleliteralemphasis{\sphinxupquote{list of pd.DataFrame}}) \textendash{} VCG data to calculate

\item {} 
\sphinxAtStartPar
\sphinxstyleliteralstrong{\sphinxupquote{t\_start}} (\sphinxstyleliteralemphasis{\sphinxupquote{list of float}}\sphinxstyleliteralemphasis{\sphinxupquote{, }}\sphinxstyleliteralemphasis{\sphinxupquote{optional}}) \textendash{} Start time from which to calculate the magnitude, default=0 (for any other value to be recognisable,
time variable must be given)

\item {} 
\sphinxAtStartPar
\sphinxstyleliteralstrong{\sphinxupquote{t\_end}} (\sphinxstyleliteralemphasis{\sphinxupquote{list of float}}\sphinxstyleliteralemphasis{\sphinxupquote{, }}\sphinxstyleliteralemphasis{\sphinxupquote{optional}}) \textendash{} End time until which to calculate the magnitudes, default=end (for any other value to be recognisable,
time variable must be given)

\end{itemize}

\item[{Returns}] \leavevmode
\sphinxAtStartPar
\begin{itemize}
\item {} 
\sphinxAtStartPar
\sphinxstylestrong{dipole\_magnitude} (\sphinxstyleemphasis{list of np.ndarray}) \textendash{} Magnitude time courses for each VCG

\item {} 
\sphinxAtStartPar
\sphinxstylestrong{weighted\_magnitude} (\sphinxstyleemphasis{list of float}) \textendash{} Mean magnitude of the VCG

\item {} 
\sphinxAtStartPar
\sphinxstylestrong{max\_dipole\_magnitude} (\sphinxstyleemphasis{list of float}) \textendash{} Maximum magnitude of the VCG

\item {} 
\sphinxAtStartPar
\sphinxstylestrong{max\_dipole\_components} (\sphinxstyleemphasis{list of list of float}) \textendash{} x, y, z components of the dipole at is maximum value

\item {} 
\sphinxAtStartPar
\sphinxstylestrong{max\_dipole\_time} (\sphinxstyleemphasis{list of float}) \textendash{} Time at which the maximum magnitude of the VCG occurs

\end{itemize}


\end{description}\end{quote}

\end{fulllineitems}



\subsubsection{signalanalysis.vcg.get\_qrs\_start\_end}
\label{\detokenize{_autosummary/signalanalysis.vcg.get_qrs_start_end:signalanalysis-vcg-get-qrs-start-end}}\label{\detokenize{_autosummary/signalanalysis.vcg.get_qrs_start_end::doc}}\index{get\_qrs\_start\_end() (in module signalanalysis.vcg)@\spxentry{get\_qrs\_start\_end()}\spxextra{in module signalanalysis.vcg}}

\begin{fulllineitems}
\phantomsection\label{\detokenize{_autosummary/signalanalysis.vcg.get_qrs_start_end:signalanalysis.vcg.get_qrs_start_end}}\pysiglinewithargsret{\sphinxcode{\sphinxupquote{signalanalysis.vcg.}}\sphinxbfcode{\sphinxupquote{get\_qrs\_start\_end}}}{\emph{\DUrole{n}{vcgs}\DUrole{p}{:} \DUrole{n}{Union\DUrole{p}{{[}}List\DUrole{p}{{[}}pandas.core.frame.DataFrame\DUrole{p}{{]}}\DUrole{p}{, }pandas.core.frame.DataFrame\DUrole{p}{{]}}}}, \emph{\DUrole{n}{velocity\_offset}\DUrole{p}{:} \DUrole{n}{int} \DUrole{o}{=} \DUrole{default_value}{2}}, \emph{\DUrole{n}{low\_p}\DUrole{p}{:} \DUrole{n}{float} \DUrole{o}{=} \DUrole{default_value}{40}}, \emph{\DUrole{n}{order}\DUrole{p}{:} \DUrole{n}{int} \DUrole{o}{=} \DUrole{default_value}{2}}, \emph{\DUrole{n}{threshold\_frac\_start}\DUrole{p}{:} \DUrole{n}{float} \DUrole{o}{=} \DUrole{default_value}{0.22}}, \emph{\DUrole{n}{threshold\_frac\_end}\DUrole{p}{:} \DUrole{n}{float} \DUrole{o}{=} \DUrole{default_value}{0.54}}, \emph{\DUrole{n}{filter\_sv}\DUrole{p}{:} \DUrole{n}{bool} \DUrole{o}{=} \DUrole{default_value}{True}}, \emph{\DUrole{n}{qrs\_window}\DUrole{p}{:} \DUrole{n}{float} \DUrole{o}{=} \DUrole{default_value}{180}}, \emph{\DUrole{n}{ecgs}\DUrole{p}{:} \DUrole{n}{Optional\DUrole{p}{{[}}Union\DUrole{p}{{[}}List\DUrole{p}{{[}}pandas.core.frame.DataFrame\DUrole{p}{{]}}\DUrole{p}{, }pandas.core.frame.DataFrame\DUrole{p}{{]}}\DUrole{p}{{]}}} \DUrole{o}{=} \DUrole{default_value}{None}}}{{ $\rightarrow$ Tuple\DUrole{p}{{[}}List\DUrole{p}{{[}}float\DUrole{p}{{]}}\DUrole{p}{, }List\DUrole{p}{{[}}float\DUrole{p}{{]}}\DUrole{p}{, }List\DUrole{p}{{[}}float\DUrole{p}{{]}}\DUrole{p}{{]}}}}
\sphinxAtStartPar
Calculate the extent of the VCG QRS complex on the basis of max derivative

\sphinxAtStartPar
TODO: Check whether i\_qrs\_start variable is needed, or can be simplified using DataFrame function

\sphinxAtStartPar
Calculate the start and end points, and hence duration, of the QRS complex of a list of VCGs. It does this by
finding the time at which the spatial velocity of the VCG exceeds a threshold value (the start time), then searches
backwards from the end of the VCG to find when this threshold is exceeded (the end time); the start and end
thresholds do not necessarily have to be the same.
\begin{quote}\begin{description}
\item[{Parameters}] \leavevmode\begin{itemize}
\item {} 
\sphinxAtStartPar
\sphinxstyleliteralstrong{\sphinxupquote{vcgs}} (\sphinxstyleliteralemphasis{\sphinxupquote{list of pd.DataFrame}}\sphinxstyleliteralemphasis{\sphinxupquote{ or }}\sphinxstyleliteralemphasis{\sphinxupquote{pd.DataFrame}}) \textendash{} List of VCG data to get QRS start and end points for

\item {} 
\sphinxAtStartPar
\sphinxstyleliteralstrong{\sphinxupquote{velocity\_offset}} (\sphinxstyleliteralemphasis{\sphinxupquote{int}}\sphinxstyleliteralemphasis{\sphinxupquote{, }}\sphinxstyleliteralemphasis{\sphinxupquote{optional}}) \textendash{} Offset between values in VCG over which to calculate spatial velocity, i.e. 1 will use neighbouring values to
calculate the gradient/velocity. Default=2

\item {} 
\sphinxAtStartPar
\sphinxstyleliteralstrong{\sphinxupquote{low\_p}} (\sphinxstyleliteralemphasis{\sphinxupquote{float}}\sphinxstyleliteralemphasis{\sphinxupquote{, }}\sphinxstyleliteralemphasis{\sphinxupquote{optional}}) \textendash{} Low frequency for bandpass filter, default=40

\item {} 
\sphinxAtStartPar
\sphinxstyleliteralstrong{\sphinxupquote{order}} (\sphinxstyleliteralemphasis{\sphinxupquote{int}}\sphinxstyleliteralemphasis{\sphinxupquote{, }}\sphinxstyleliteralemphasis{\sphinxupquote{optional}}) \textendash{} Order for Butterworth filter, default=2

\item {} 
\sphinxAtStartPar
\sphinxstyleliteralstrong{\sphinxupquote{threshold\_frac\_start}} (\sphinxstyleliteralemphasis{\sphinxupquote{float}}\sphinxstyleliteralemphasis{\sphinxupquote{, }}\sphinxstyleliteralemphasis{\sphinxupquote{optional}}) \textendash{} Fraction of maximum spatial velocity to trigger start of QRS detection, default=0.15

\item {} 
\sphinxAtStartPar
\sphinxstyleliteralstrong{\sphinxupquote{threshold\_frac\_end}} (\sphinxstyleliteralemphasis{\sphinxupquote{float}}\sphinxstyleliteralemphasis{\sphinxupquote{, }}\sphinxstyleliteralemphasis{\sphinxupquote{optional}}) \textendash{} Fraction of maximum spatial velocity to trigger end of QRS detection, default=0.15

\item {} 
\sphinxAtStartPar
\sphinxstyleliteralstrong{\sphinxupquote{filter\_sv}} (\sphinxstyleliteralemphasis{\sphinxupquote{bool}}\sphinxstyleliteralemphasis{\sphinxupquote{, }}\sphinxstyleliteralemphasis{\sphinxupquote{optional}}) \textendash{} Whether or not to apply filtering to spatial velocity prior to finding the start/end points for the threshold

\item {} 
\sphinxAtStartPar
\sphinxstyleliteralstrong{\sphinxupquote{qrs\_window}} (\sphinxstyleliteralemphasis{\sphinxupquote{float}}\sphinxstyleliteralemphasis{\sphinxupquote{, }}\sphinxstyleliteralemphasis{\sphinxupquote{optional}}) \textendash{} Default size of ‘window’ in which to search for end of QRS complex, default=180ms

\item {} 
\sphinxAtStartPar
\sphinxstyleliteralstrong{\sphinxupquote{ecgs}} (\sphinxstyleliteralemphasis{\sphinxupquote{list of pd.DataFrame}}\sphinxstyleliteralemphasis{\sphinxupquote{ or }}\sphinxstyleliteralemphasis{\sphinxupquote{pd.DataFrame}}\sphinxstyleliteralemphasis{\sphinxupquote{, }}\sphinxstyleliteralemphasis{\sphinxupquote{optional}}) \textendash{} ECG data associated with VCG data. Only used if having trouble establishing QRS start, in which case will be
used to plot ECG data to allow user to determine whether or not the QRS is occurring at the start of the
simulation, or whether there is a more deep\sphinxhyphen{}seated issue with the data.

\end{itemize}

\item[{Returns}] \leavevmode
\sphinxAtStartPar
\begin{itemize}
\item {} 
\sphinxAtStartPar
\sphinxstylestrong{qrs\_start} (\sphinxstyleemphasis{list of float}) \textendash{} List of start time of QRS complexes of provided VCGs

\item {} 
\sphinxAtStartPar
\sphinxstylestrong{qrs\_end} (\sphinxstyleemphasis{list of float}) \textendash{} List of end time of QRS complex of provided VCGs

\item {} 
\sphinxAtStartPar
\sphinxstylestrong{qrs\_duration} (\sphinxstyleemphasis{list of float}) \textendash{} List of duration of QRS complex of provided VCGs

\end{itemize}


\end{description}\end{quote}

\end{fulllineitems}



\subsubsection{signalanalysis.vcg.get\_single\_vcg\_azimuth\_elevation}
\label{\detokenize{_autosummary/signalanalysis.vcg.get_single_vcg_azimuth_elevation:signalanalysis-vcg-get-single-vcg-azimuth-elevation}}\label{\detokenize{_autosummary/signalanalysis.vcg.get_single_vcg_azimuth_elevation::doc}}\index{get\_single\_vcg\_azimuth\_elevation() (in module signalanalysis.vcg)@\spxentry{get\_single\_vcg\_azimuth\_elevation()}\spxextra{in module signalanalysis.vcg}}

\begin{fulllineitems}
\phantomsection\label{\detokenize{_autosummary/signalanalysis.vcg.get_single_vcg_azimuth_elevation:signalanalysis.vcg.get_single_vcg_azimuth_elevation}}\pysiglinewithargsret{\sphinxcode{\sphinxupquote{signalanalysis.vcg.}}\sphinxbfcode{\sphinxupquote{get\_single\_vcg\_azimuth\_elevation}}}{\emph{\DUrole{n}{vcg}\DUrole{p}{:} \DUrole{n}{pandas.core.frame.DataFrame}}, \emph{\DUrole{n}{t\_start}\DUrole{p}{:} \DUrole{n}{float}}, \emph{\DUrole{n}{t\_end}\DUrole{p}{:} \DUrole{n}{float}}, \emph{\DUrole{n}{weighted}\DUrole{p}{:} \DUrole{n}{bool} \DUrole{o}{=} \DUrole{default_value}{True}}}{{ $\rightarrow$ Tuple\DUrole{p}{{[}}List\DUrole{p}{{[}}float\DUrole{p}{{]}}\DUrole{p}{, }List\DUrole{p}{{[}}float\DUrole{p}{{]}}\DUrole{p}{, }numpy.ndarray\DUrole{p}{{]}}}}
\sphinxAtStartPar
Get the azimuth and elevation data for a single VCG trace, along with the average dipole magnitude.

\sphinxAtStartPar
Returns the azimuth and elevation angles for a single given VCG trace. Can analyse only a segment of the
VCG if required, and can weight the angles according to the dipole magnitude. Primarily designed as a helper
function for get\_azimuth\_elevation and get\_weighted\_dipole\_angles.
\begin{quote}\begin{description}
\item[{Parameters}] \leavevmode\begin{itemize}
\item {} 
\sphinxAtStartPar
\sphinxstyleliteralstrong{\sphinxupquote{vcg}} (\sphinxstyleliteralemphasis{\sphinxupquote{pd.DataFrame}}) \textendash{} VCG data to calculate

\item {} 
\sphinxAtStartPar
\sphinxstyleliteralstrong{\sphinxupquote{t\_start}} (\sphinxstyleliteralemphasis{\sphinxupquote{float}}) \textendash{} Start time from which to calculate the angles

\item {} 
\sphinxAtStartPar
\sphinxstyleliteralstrong{\sphinxupquote{t\_end}} (\sphinxstyleliteralemphasis{\sphinxupquote{float}}) \textendash{} End time until which to calculate the angles

\item {} 
\sphinxAtStartPar
\sphinxstyleliteralstrong{\sphinxupquote{weighted}} (\sphinxstyleliteralemphasis{\sphinxupquote{bool}}\sphinxstyleliteralemphasis{\sphinxupquote{, }}\sphinxstyleliteralemphasis{\sphinxupquote{optional}}) \textendash{} Whether or not to weight the returned angles by the magnitude of the dipole at the same moment, default=True

\end{itemize}

\item[{Returns}] \leavevmode
\sphinxAtStartPar
\begin{itemize}
\item {} 
\sphinxAtStartPar
\sphinxstylestrong{theta} (\sphinxstyleemphasis{list of float}) \textendash{} List of the azimuth angles for the VCG dipole, potentially weighted according to the dipole magnitude at the
associated time

\item {} 
\sphinxAtStartPar
\sphinxstylestrong{phi} (\sphinxstyleemphasis{list of float}) \textendash{} List of the elevation above xy\sphinxhyphen{}plane angles for the VCG dipole, potentially weighted according to the dipole
magnitude at the associated time

\item {} 
\sphinxAtStartPar
\sphinxstylestrong{dipole\_magnitude} (\sphinxstyleemphasis{np.ndarray}) \textendash{} Array containing the dipole magnitude at all points throughout the VCG

\end{itemize}


\end{description}\end{quote}

\end{fulllineitems}



\subsubsection{signalanalysis.vcg.get\_spatial\_velocity}
\label{\detokenize{_autosummary/signalanalysis.vcg.get_spatial_velocity:signalanalysis-vcg-get-spatial-velocity}}\label{\detokenize{_autosummary/signalanalysis.vcg.get_spatial_velocity::doc}}\index{get\_spatial\_velocity() (in module signalanalysis.vcg)@\spxentry{get\_spatial\_velocity()}\spxextra{in module signalanalysis.vcg}}

\begin{fulllineitems}
\phantomsection\label{\detokenize{_autosummary/signalanalysis.vcg.get_spatial_velocity:signalanalysis.vcg.get_spatial_velocity}}\pysiglinewithargsret{\sphinxcode{\sphinxupquote{signalanalysis.vcg.}}\sphinxbfcode{\sphinxupquote{get\_spatial\_velocity}}}{\emph{\DUrole{n}{vcgs}\DUrole{p}{:} \DUrole{n}{Union\DUrole{p}{{[}}List\DUrole{p}{{[}}pandas.core.frame.DataFrame\DUrole{p}{{]}}\DUrole{p}{, }pandas.core.frame.DataFrame\DUrole{p}{{]}}}}, \emph{\DUrole{n}{velocity\_offset}\DUrole{p}{:} \DUrole{n}{int} \DUrole{o}{=} \DUrole{default_value}{2}}, \emph{\DUrole{n}{filter\_sv}\DUrole{p}{:} \DUrole{n}{bool} \DUrole{o}{=} \DUrole{default_value}{True}}, \emph{\DUrole{n}{low\_p}\DUrole{p}{:} \DUrole{n}{float} \DUrole{o}{=} \DUrole{default_value}{40}}, \emph{\DUrole{n}{order}\DUrole{p}{:} \DUrole{n}{int} \DUrole{o}{=} \DUrole{default_value}{2}}}{{ $\rightarrow$ List\DUrole{p}{{[}}pandas.core.frame.DataFrame\DUrole{p}{{]}}}}
\sphinxAtStartPar
Calculate spatial velocity

\sphinxAtStartPar
Calculate the spatial velocity of a VCG, in terms of calculating the gradient of the VCG in each of its x,
y and z components, before combining these components in a Euclidian norm. Will then find the point at which the
spatial velocity exceeds a threshold value, and the point at which it declines below another threshold value.
\begin{quote}\begin{description}
\item[{Parameters}] \leavevmode\begin{itemize}
\item {} 
\sphinxAtStartPar
\sphinxstyleliteralstrong{\sphinxupquote{vcgs}} (\sphinxstyleliteralemphasis{\sphinxupquote{list of pd.DataFrame}}\sphinxstyleliteralemphasis{\sphinxupquote{ or }}\sphinxstyleliteralemphasis{\sphinxupquote{pd.DataFrame}}) \textendash{} VCG data to analyse

\item {} 
\sphinxAtStartPar
\sphinxstyleliteralstrong{\sphinxupquote{velocity\_offset}} (\sphinxstyleliteralemphasis{\sphinxupquote{int}}\sphinxstyleliteralemphasis{\sphinxupquote{, }}\sphinxstyleliteralemphasis{\sphinxupquote{optional}}) \textendash{} Offset between values in VCG over which to calculate spatial velocity, i.e. 1 will use neighbouring values to
calculate the gradient/velocity. Default=2

\item {} 
\sphinxAtStartPar
\sphinxstyleliteralstrong{\sphinxupquote{filter\_sv}} (\sphinxstyleliteralemphasis{\sphinxupquote{bool}}\sphinxstyleliteralemphasis{\sphinxupquote{, }}\sphinxstyleliteralemphasis{\sphinxupquote{optional}}) \textendash{} Whether or not to apply filtering to spatial velocity, default=True

\item {} 
\sphinxAtStartPar
\sphinxstyleliteralstrong{\sphinxupquote{low\_p}} (\sphinxstyleliteralemphasis{\sphinxupquote{float}}\sphinxstyleliteralemphasis{\sphinxupquote{, }}\sphinxstyleliteralemphasis{\sphinxupquote{optional}}) \textendash{} Low frequency for bandpass filter, default=40

\item {} 
\sphinxAtStartPar
\sphinxstyleliteralstrong{\sphinxupquote{order}} (\sphinxstyleliteralemphasis{\sphinxupquote{int}}\sphinxstyleliteralemphasis{\sphinxupquote{, }}\sphinxstyleliteralemphasis{\sphinxupquote{optional}}) \textendash{} Order for Butterworth filter, default=2

\end{itemize}

\item[{Returns}] \leavevmode
\sphinxAtStartPar
\sphinxstylestrong{sv} \textendash{} Spatial velocity data, filtered according to input parameters

\item[{Return type}] \leavevmode
\sphinxAtStartPar
list of pd.DataFrame

\end{description}\end{quote}
\subsubsection*{Notes}

\sphinxAtStartPar
Calculation of spatial velocity based on \sphinxstepexplicit %
\begin{footnote}[1]\phantomsection\label{\thesphinxscope.1}%
\sphinxAtStartFootnote
Kors JA, van Herpen G, “Methodology of QT\sphinxhyphen{}interval measurement in the modular ECG analysis system (MEANS)”
Ann Noninvasive Electrocardiol. 2009 Jan;14 Suppl 1:S48\sphinxhyphen{}53. doi: 10.1111/j.1542\sphinxhyphen{}474X.2008.00261.x.
%
\end{footnote}, \sphinxstepexplicit %
\begin{footnote}[2]\phantomsection\label{\thesphinxscope.2}%
\sphinxAtStartFootnote
Xue JQ, “Robust QT Interval Estimation—From Algorithm to Validation”
Ann Noninvasive Electrocardiol. 2009 Jan;14 Suppl 1:S35\sphinxhyphen{}41. doi: 10.1111/j.1542\sphinxhyphen{}474X.2008.00264.x.
%
\end{footnote}, \sphinxstepexplicit %
\begin{footnote}[3]\phantomsection\label{\thesphinxscope.3}%
\sphinxAtStartFootnote
Sörnmo L, “A model\sphinxhyphen{}based approach to QRS delineation”
Comput Biomed Res. 1987 Dec;20(6):526\sphinxhyphen{}42.
%
\end{footnote}
\subsubsection*{References}

\end{fulllineitems}



\subsubsection{signalanalysis.vcg.get\_vcg\_area}
\label{\detokenize{_autosummary/signalanalysis.vcg.get_vcg_area:signalanalysis-vcg-get-vcg-area}}\label{\detokenize{_autosummary/signalanalysis.vcg.get_vcg_area::doc}}\index{get\_vcg\_area() (in module signalanalysis.vcg)@\spxentry{get\_vcg\_area()}\spxextra{in module signalanalysis.vcg}}

\begin{fulllineitems}
\phantomsection\label{\detokenize{_autosummary/signalanalysis.vcg.get_vcg_area:signalanalysis.vcg.get_vcg_area}}\pysiglinewithargsret{\sphinxcode{\sphinxupquote{signalanalysis.vcg.}}\sphinxbfcode{\sphinxupquote{get\_vcg\_area}}}{\emph{\DUrole{n}{vcgs}\DUrole{p}{:} \DUrole{n}{Union\DUrole{p}{{[}}List\DUrole{p}{{[}}pandas.core.frame.DataFrame\DUrole{p}{{]}}\DUrole{p}{, }pandas.core.frame.DataFrame\DUrole{p}{{]}}}}, \emph{\DUrole{n}{limits\_start}\DUrole{p}{:} \DUrole{n}{Optional\DUrole{p}{{[}}List\DUrole{p}{{[}}float\DUrole{p}{{]}}\DUrole{p}{{]}}} \DUrole{o}{=} \DUrole{default_value}{None}}, \emph{\DUrole{n}{limits\_end}\DUrole{p}{:} \DUrole{n}{Optional\DUrole{p}{{[}}List\DUrole{p}{{[}}float\DUrole{p}{{]}}\DUrole{p}{{]}}} \DUrole{o}{=} \DUrole{default_value}{None}}, \emph{\DUrole{n}{method}\DUrole{p}{:} \DUrole{n}{str} \DUrole{o}{=} \DUrole{default_value}{\textquotesingle{}pythag\textquotesingle{}}}, \emph{\DUrole{n}{matlab\_match}\DUrole{p}{:} \DUrole{n}{bool} \DUrole{o}{=} \DUrole{default_value}{False}}}{{ $\rightarrow$ List\DUrole{p}{{[}}float\DUrole{p}{{]}}}}
\sphinxAtStartPar
Calculate area under VCG curve for a given section (e.g. QRS complex).

\sphinxAtStartPar
Calculate the area under the VCG between two intervals (usually QRS start and QRS end). This is calculated in two
ways: a ‘Pythagorean’ method, wherein the area under each of the VCG(x), VCG(y) and VCG(z) curves are calculated,
then combined in a Euclidean norm, or a ‘3D’ method, wherein the area of the arc traced in 3D space between
successive timepoints is calculated, then summed.
\begin{quote}\begin{description}
\item[{Parameters}] \leavevmode\begin{itemize}
\item {} 
\sphinxAtStartPar
\sphinxstyleliteralstrong{\sphinxupquote{vcgs}} (\sphinxstyleliteralemphasis{\sphinxupquote{pd.DataFrame}}\sphinxstyleliteralemphasis{\sphinxupquote{ or }}\sphinxstyleliteralemphasis{\sphinxupquote{list of pd.DataFrame}}) \textendash{} VCG data from which to get area

\item {} 
\sphinxAtStartPar
\sphinxstyleliteralstrong{\sphinxupquote{limits\_start}} (\sphinxstyleliteralemphasis{\sphinxupquote{list of float}}\sphinxstyleliteralemphasis{\sphinxupquote{, }}\sphinxstyleliteralemphasis{\sphinxupquote{optional}}) \textendash{} Start times (NOT INDICES) for where to calculate area under curve from, default=0

\item {} 
\sphinxAtStartPar
\sphinxstyleliteralstrong{\sphinxupquote{limits\_end}} (\sphinxstyleliteralemphasis{\sphinxupquote{list of float}}\sphinxstyleliteralemphasis{\sphinxupquote{, }}\sphinxstyleliteralemphasis{\sphinxupquote{optional}}) \textendash{} End times (NOT INDICES) for where to calculate are under curve until, default=end

\item {} 
\sphinxAtStartPar
\sphinxstyleliteralstrong{\sphinxupquote{method}} (\sphinxstyleliteralemphasis{\sphinxupquote{\{\textquotesingle{}pythag\textquotesingle{}}}\sphinxstyleliteralemphasis{\sphinxupquote{, }}\sphinxstyleliteralemphasis{\sphinxupquote{\textquotesingle{}3d\textquotesingle{}\}}}\sphinxstyleliteralemphasis{\sphinxupquote{, }}\sphinxstyleliteralemphasis{\sphinxupquote{optional}}) \textendash{} Which method to use to calculate the area under the VCG curve, default=’pythag’

\item {} 
\sphinxAtStartPar
\sphinxstyleliteralstrong{\sphinxupquote{matlab\_match}} (\sphinxstyleliteralemphasis{\sphinxupquote{bool}}\sphinxstyleliteralemphasis{\sphinxupquote{, }}\sphinxstyleliteralemphasis{\sphinxupquote{optional}}) \textendash{} Whether to alter the calculation for start and end indices to match the original Matlab output, from which this
module is based, default=False

\end{itemize}

\item[{Returns}] \leavevmode
\sphinxAtStartPar
\begin{itemize}
\item {} 
\sphinxAtStartPar
\sphinxstylestrong{qrs\_area\_3d} (\sphinxstyleemphasis{list of float}) \textendash{} Values for the area under the curve (as defined by the 3D method) between the provided limits for each of the
VCGs

\item {} 
\sphinxAtStartPar
\sphinxstylestrong{qrs\_area\_pythag} (\sphinxstyleemphasis{list of float}) \textendash{} Values for the area under the curve (as defined by the Pythagorean method) between the provided limits for each
of the VCGs

\item {} 
\sphinxAtStartPar
\sphinxstylestrong{qrs\_area\_components} (\sphinxstyleemphasis{list of list of float}) \textendash{} Areas under the individual x, y, z curves of the VCG, for each of the supplied VCGs

\end{itemize}


\end{description}\end{quote}

\end{fulllineitems}



\subsubsection{signalanalysis.vcg.get\_vcg\_from\_ecg}
\label{\detokenize{_autosummary/signalanalysis.vcg.get_vcg_from_ecg:signalanalysis-vcg-get-vcg-from-ecg}}\label{\detokenize{_autosummary/signalanalysis.vcg.get_vcg_from_ecg::doc}}\index{get\_vcg\_from\_ecg() (in module signalanalysis.vcg)@\spxentry{get\_vcg\_from\_ecg()}\spxextra{in module signalanalysis.vcg}}

\begin{fulllineitems}
\phantomsection\label{\detokenize{_autosummary/signalanalysis.vcg.get_vcg_from_ecg:signalanalysis.vcg.get_vcg_from_ecg}}\pysiglinewithargsret{\sphinxcode{\sphinxupquote{signalanalysis.vcg.}}\sphinxbfcode{\sphinxupquote{get\_vcg\_from\_ecg}}}{\emph{\DUrole{n}{ecgs}\DUrole{p}{:} \DUrole{n}{Union\DUrole{p}{{[}}List\DUrole{p}{{[}}pandas.core.frame.DataFrame\DUrole{p}{{]}}\DUrole{p}{, }pandas.core.frame.DataFrame\DUrole{p}{{]}}}}}{{ $\rightarrow$ List\DUrole{p}{{[}}pandas.core.frame.DataFrame\DUrole{p}{{]}}}}
\sphinxAtStartPar
Convert ECG data to vectorcardiogram (VCG) data using the Kors matrix method

\sphinxAtStartPar
\DUrole{versionmodified,deprecated}{Deprecated since version The: }use of this module is deprecated, and the internal class method should be used in preference (
signalanalysis.vcg.Vcg.get\_from\_ecg())
\begin{quote}\begin{description}
\item[{Parameters}] \leavevmode
\sphinxAtStartPar
\sphinxstyleliteralstrong{\sphinxupquote{ecgs}} (\sphinxstyleliteralemphasis{\sphinxupquote{list of pd.DataFrame}}\sphinxstyleliteralemphasis{\sphinxupquote{ or }}\sphinxstyleliteralemphasis{\sphinxupquote{pd.DataFrame}}) \textendash{} List of ECG dataframe data, or ECG dataframe data directly, with dict keys corresponding to ECG outputs

\item[{Returns}] \leavevmode
\sphinxAtStartPar
\sphinxstylestrong{vcgs} \textendash{} List of VCG output data

\item[{Return type}] \leavevmode
\sphinxAtStartPar
list of pd.DataFrame

\end{description}\end{quote}
\subsubsection*{References}
\begin{description}
\item[{Kors JA, van Herpen G, Sittig AC, van Bemmel JH.}] \leavevmode
\sphinxAtStartPar
Reconstruction of the Frank vectorcardiogram from standard electrocardiographic leads: diagnostic comparison
of different methods
Eur Heart J. 1990 Dec;11(12):1083\sphinxhyphen{}92.

\end{description}

\end{fulllineitems}



\subsubsection{signalanalysis.vcg.get\_weighted\_dipole\_angles}
\label{\detokenize{_autosummary/signalanalysis.vcg.get_weighted_dipole_angles:signalanalysis-vcg-get-weighted-dipole-angles}}\label{\detokenize{_autosummary/signalanalysis.vcg.get_weighted_dipole_angles::doc}}\index{get\_weighted\_dipole\_angles() (in module signalanalysis.vcg)@\spxentry{get\_weighted\_dipole\_angles()}\spxextra{in module signalanalysis.vcg}}

\begin{fulllineitems}
\phantomsection\label{\detokenize{_autosummary/signalanalysis.vcg.get_weighted_dipole_angles:signalanalysis.vcg.get_weighted_dipole_angles}}\pysiglinewithargsret{\sphinxcode{\sphinxupquote{signalanalysis.vcg.}}\sphinxbfcode{\sphinxupquote{get\_weighted\_dipole\_angles}}}{\emph{\DUrole{n}{vcgs}\DUrole{p}{:} \DUrole{n}{Union\DUrole{p}{{[}}List\DUrole{p}{{[}}pandas.core.frame.DataFrame\DUrole{p}{{]}}\DUrole{p}{, }pandas.core.frame.DataFrame\DUrole{p}{{]}}}}, \emph{\DUrole{n}{t\_start}\DUrole{p}{:} \DUrole{n}{Optional\DUrole{p}{{[}}List\DUrole{p}{{[}}float\DUrole{p}{{]}}\DUrole{p}{{]}}} \DUrole{o}{=} \DUrole{default_value}{None}}, \emph{\DUrole{n}{t\_end}\DUrole{p}{:} \DUrole{n}{Optional\DUrole{p}{{[}}List\DUrole{p}{{[}}float\DUrole{p}{{]}}\DUrole{p}{{]}}} \DUrole{o}{=} \DUrole{default_value}{None}}}{{ $\rightarrow$ Tuple\DUrole{p}{{[}}List\DUrole{p}{{[}}float\DUrole{p}{{]}}\DUrole{p}{, }List\DUrole{p}{{[}}float\DUrole{p}{{]}}\DUrole{p}{, }List\DUrole{p}{{[}}List\DUrole{p}{{[}}float\DUrole{p}{{]}}\DUrole{p}{{]}}\DUrole{p}{{]}}}}
\sphinxAtStartPar
Calculate metrics relating to the angles of the weighted dipole of the VCG. Usually used with QRS limits.

\sphinxAtStartPar
Calculates the weighted averages of both the azimuth and the elevation (inclination above the xy\sphinxhyphen{}plane) for a
given section of the VCG. Based on these weighted averages of the angles, the unit weighted dipole for that
section of the VCG is returned as well.
\begin{quote}\begin{description}
\item[{Parameters}] \leavevmode\begin{itemize}
\item {} 
\sphinxAtStartPar
\sphinxstyleliteralstrong{\sphinxupquote{vcgs}} (\sphinxstyleliteralemphasis{\sphinxupquote{pd.DataFrame}}\sphinxstyleliteralemphasis{\sphinxupquote{ or }}\sphinxstyleliteralemphasis{\sphinxupquote{list of pd.DataFrame}}) \textendash{} VCG data to calculate

\item {} 
\sphinxAtStartPar
\sphinxstyleliteralstrong{\sphinxupquote{t\_start}} (\sphinxstyleliteralemphasis{\sphinxupquote{list of float}}\sphinxstyleliteralemphasis{\sphinxupquote{, }}\sphinxstyleliteralemphasis{\sphinxupquote{optional}}) \textendash{} Start time from which to calculate the angles, default=0

\item {} 
\sphinxAtStartPar
\sphinxstyleliteralstrong{\sphinxupquote{t\_end}} (\sphinxstyleliteralemphasis{\sphinxupquote{list of float}}\sphinxstyleliteralemphasis{\sphinxupquote{, }}\sphinxstyleliteralemphasis{\sphinxupquote{optional}}) \textendash{} End time until which to calculate the angles, default=end

\end{itemize}

\item[{Returns}] \leavevmode
\sphinxAtStartPar
\begin{itemize}
\item {} 
\sphinxAtStartPar
\sphinxstylestrong{waa} (\sphinxstyleemphasis{list of float}) \textendash{} List of Weighted Average Azimuth angles (in radians) for each given VCG

\item {} 
\sphinxAtStartPar
\sphinxstylestrong{wae} (\sphinxstyleemphasis{list of float}) \textendash{} List of Weighted Average Elevation (above xy\sphinxhyphen{}plane) angles (in radians) for each given VCG

\item {} 
\sphinxAtStartPar
\sphinxstylestrong{uwd} (\sphinxstyleemphasis{list of list of float}) \textendash{} x, y, z coordinates for the unit mean weighted dipole for the given (section of) VCGs

\end{itemize}


\end{description}\end{quote}

\end{fulllineitems}



\subsubsection{signalanalysis.vcg.plot\_density\_effect}
\label{\detokenize{_autosummary/signalanalysis.vcg.plot_density_effect:signalanalysis-vcg-plot-density-effect}}\label{\detokenize{_autosummary/signalanalysis.vcg.plot_density_effect::doc}}\index{plot\_density\_effect() (in module signalanalysis.vcg)@\spxentry{plot\_density\_effect()}\spxextra{in module signalanalysis.vcg}}

\begin{fulllineitems}
\phantomsection\label{\detokenize{_autosummary/signalanalysis.vcg.plot_density_effect:signalanalysis.vcg.plot_density_effect}}\pysiglinewithargsret{\sphinxcode{\sphinxupquote{signalanalysis.vcg.}}\sphinxbfcode{\sphinxupquote{plot\_density\_effect}}}{\emph{\DUrole{n}{metrics}}, \emph{\DUrole{n}{metric\_name}}, \emph{\DUrole{n}{metric\_labels}\DUrole{o}{=}\DUrole{default_value}{None}}, \emph{\DUrole{n}{density\_labels}\DUrole{o}{=}\DUrole{default_value}{None}}, \emph{\DUrole{n}{linestyles}\DUrole{o}{=}\DUrole{default_value}{None}}, \emph{\DUrole{n}{colours}\DUrole{o}{=}\DUrole{default_value}{None}}, \emph{\DUrole{n}{markers}\DUrole{o}{=}\DUrole{default_value}{None}}}{}
\sphinxAtStartPar
Plot the effect of density on metrics.

\end{fulllineitems}



\subsubsection{signalanalysis.vcg.plot\_metric\_change}
\label{\detokenize{_autosummary/signalanalysis.vcg.plot_metric_change:signalanalysis-vcg-plot-metric-change}}\label{\detokenize{_autosummary/signalanalysis.vcg.plot_metric_change::doc}}\index{plot\_metric\_change() (in module signalanalysis.vcg)@\spxentry{plot\_metric\_change()}\spxextra{in module signalanalysis.vcg}}

\begin{fulllineitems}
\phantomsection\label{\detokenize{_autosummary/signalanalysis.vcg.plot_metric_change:signalanalysis.vcg.plot_metric_change}}\pysiglinewithargsret{\sphinxcode{\sphinxupquote{signalanalysis.vcg.}}\sphinxbfcode{\sphinxupquote{plot\_metric\_change}}}{\emph{\DUrole{n}{metrics}}, \emph{\DUrole{n}{metrics\_phi}}, \emph{\DUrole{n}{metrics\_rho}}, \emph{\DUrole{n}{metrics\_z}}, \emph{\DUrole{n}{metric\_name}}, \emph{\DUrole{n}{metrics\_lv}\DUrole{o}{=}\DUrole{default_value}{None}}, \emph{\DUrole{n}{labels}\DUrole{o}{=}\DUrole{default_value}{None}}, \emph{\DUrole{n}{scattermarkers}\DUrole{o}{=}\DUrole{default_value}{None}}, \emph{\DUrole{n}{linemarkers}\DUrole{o}{=}\DUrole{default_value}{None}}, \emph{\DUrole{n}{colours}\DUrole{o}{=}\DUrole{default_value}{None}}, \emph{\DUrole{n}{linestyles}\DUrole{o}{=}\DUrole{default_value}{None}}, \emph{\DUrole{n}{layout}\DUrole{o}{=}\DUrole{default_value}{None}}, \emph{\DUrole{n}{axis\_match}\DUrole{o}{=}\DUrole{default_value}{True}}, \emph{\DUrole{n}{no\_labels}\DUrole{o}{=}\DUrole{default_value}{False}}}{}
\sphinxAtStartPar
Function to plot all the various figures for trend analysis in one go.

\end{fulllineitems}



\subsubsection{signalanalysis.vcg.plot\_metric\_change\_barplot}
\label{\detokenize{_autosummary/signalanalysis.vcg.plot_metric_change_barplot:signalanalysis-vcg-plot-metric-change-barplot}}\label{\detokenize{_autosummary/signalanalysis.vcg.plot_metric_change_barplot::doc}}\index{plot\_metric\_change\_barplot() (in module signalanalysis.vcg)@\spxentry{plot\_metric\_change\_barplot()}\spxextra{in module signalanalysis.vcg}}

\begin{fulllineitems}
\phantomsection\label{\detokenize{_autosummary/signalanalysis.vcg.plot_metric_change_barplot:signalanalysis.vcg.plot_metric_change_barplot}}\pysiglinewithargsret{\sphinxcode{\sphinxupquote{signalanalysis.vcg.}}\sphinxbfcode{\sphinxupquote{plot\_metric\_change\_barplot}}}{\emph{\DUrole{n}{metrics\_cont}}, \emph{\DUrole{n}{metrics\_lv}}, \emph{\DUrole{n}{metrics\_sept}}, \emph{\DUrole{n}{metric\_labels}}, \emph{\DUrole{n}{layout}\DUrole{o}{=}\DUrole{default_value}{None}}}{}
\sphinxAtStartPar
Plots a bar chart for the observed metrics.

\end{fulllineitems}

\subsubsection*{Classes}


\begin{savenotes}\sphinxatlongtablestart\begin{longtable}[c]{\X{1}{2}\X{1}{2}}
\hline

\endfirsthead

\multicolumn{2}{c}%
{\makebox[0pt]{\sphinxtablecontinued{\tablename\ \thetable{} \textendash{} continued from previous page}}}\\
\hline

\endhead

\hline
\multicolumn{2}{r}{\makebox[0pt][r]{\sphinxtablecontinued{continues on next page}}}\\
\endfoot

\endlastfoot

\sphinxAtStartPar
{\hyperref[\detokenize{_autosummary/signalanalysis.vcg.Vcg:signalanalysis.vcg.Vcg}]{\sphinxcrossref{\sphinxcode{\sphinxupquote{Vcg}}}}}
&
\sphinxAtStartPar
Base class to encapsulate data from VCG
\\
\hline
\end{longtable}\sphinxatlongtableend\end{savenotes}


\subsubsection{signalanalysis.vcg.Vcg}
\label{\detokenize{_autosummary/signalanalysis.vcg.Vcg:signalanalysis-vcg-vcg}}\label{\detokenize{_autosummary/signalanalysis.vcg.Vcg::doc}}\index{Vcg (class in signalanalysis.vcg)@\spxentry{Vcg}\spxextra{class in signalanalysis.vcg}}

\begin{fulllineitems}
\phantomsection\label{\detokenize{_autosummary/signalanalysis.vcg.Vcg:signalanalysis.vcg.Vcg}}\pysiglinewithargsret{\sphinxbfcode{\sphinxupquote{class }}\sphinxcode{\sphinxupquote{signalanalysis.vcg.}}\sphinxbfcode{\sphinxupquote{Vcg}}}{\emph{\DUrole{n}{ecg}\DUrole{p}{:} \DUrole{n}{{\hyperref[\detokenize{_autosummary/signalanalysis.ecg.Ecg:signalanalysis.ecg.Ecg}]{\sphinxcrossref{signalanalysis.ecg.Ecg}}}}}, \emph{\DUrole{o}{**}\DUrole{n}{kwargs}}}{}
\sphinxAtStartPar
Bases: {\hyperref[\detokenize{_autosummary/signalanalysis.general.Signal:signalanalysis.general.Signal}]{\sphinxcrossref{\sphinxcode{\sphinxupquote{signalanalysis.general.Signal}}}}}

\sphinxAtStartPar
Base class to encapsulate data from VCG
\index{ecg\_filter (signalanalysis.vcg.Vcg attribute)@\spxentry{ecg\_filter}\spxextra{signalanalysis.vcg.Vcg attribute}}

\begin{fulllineitems}
\phantomsection\label{\detokenize{_autosummary/signalanalysis.vcg.Vcg:signalanalysis.vcg.Vcg.ecg_filter}}\pysigline{\sphinxbfcode{\sphinxupquote{ecg\_filter}}}
\sphinxAtStartPar
Whether or not a filter was applied to the original ECG data before transformation to VCG
\begin{quote}\begin{description}
\item[{Type}] \leavevmode
\sphinxAtStartPar
\{‘butterworth’, ‘savitzky\sphinxhyphen{}golay’\}, optional

\end{description}\end{quote}

\end{fulllineitems}

\index{get\_from\_ecg() (signalanalysis.vcg.Vcg method)@\spxentry{get\_from\_ecg()}\spxextra{signalanalysis.vcg.Vcg method}}

\begin{fulllineitems}
\phantomsection\label{\detokenize{_autosummary/signalanalysis.vcg.Vcg:signalanalysis.vcg.Vcg.get_from_ecg}}\pysiglinewithargsret{\sphinxbfcode{\sphinxupquote{get\_from\_ecg}}}{\emph{\DUrole{n}{ecg}}}{}
\sphinxAtStartPar
Converts ECG to VCG

\end{fulllineitems}



\sphinxstrong{See also:}
\nopagebreak


\sphinxAtStartPar
\sphinxcode{\sphinxupquote{signalanalysis.general.Signal : Base class
:py:class:\textasciigrave{}signalanalysis.ecg.Ecg}} : Related class, from which the VCG signal is obtained


\subsubsection*{Methods}


\begin{savenotes}\sphinxatlongtablestart\begin{longtable}[c]{\X{1}{2}\X{1}{2}}
\hline

\endfirsthead

\multicolumn{2}{c}%
{\makebox[0pt]{\sphinxtablecontinued{\tablename\ \thetable{} \textendash{} continued from previous page}}}\\
\hline

\endhead

\hline
\multicolumn{2}{r}{\makebox[0pt][r]{\sphinxtablecontinued{continues on next page}}}\\
\endfoot

\endlastfoot

\sphinxAtStartPar
{\hyperref[\detokenize{_autosummary/signalanalysis.vcg.Vcg:signalanalysis.vcg.Vcg.apply_filter}]{\sphinxcrossref{\sphinxcode{\sphinxupquote{apply\_filter}}}}}
&
\sphinxAtStartPar
Apply a given filter to the data, using their respective arguments as required
\\
\hline
\sphinxAtStartPar
{\hyperref[\detokenize{_autosummary/signalanalysis.vcg.Vcg:id0}]{\sphinxcrossref{\sphinxcode{\sphinxupquote{get\_from\_ecg}}}}}
&
\sphinxAtStartPar
Convert ECG data to vectorcardiogram (VCG) data using the Kors matrix method
\\
\hline
\sphinxAtStartPar
{\hyperref[\detokenize{_autosummary/signalanalysis.vcg.Vcg:signalanalysis.vcg.Vcg.get_n_beats}]{\sphinxcrossref{\sphinxcode{\sphinxupquote{get\_n\_beats}}}}}
&
\sphinxAtStartPar
Calculate the number of beats in a given signal, and save the individual beats to the object for later use
\\
\hline
\sphinxAtStartPar
{\hyperref[\detokenize{_autosummary/signalanalysis.vcg.Vcg:signalanalysis.vcg.Vcg.get_rms}]{\sphinxcrossref{\sphinxcode{\sphinxupquote{get\_rms}}}}}
&
\sphinxAtStartPar
Returns the RMS of the combined signal
\\
\hline
\sphinxAtStartPar
\sphinxcode{\sphinxupquote{get\_twave\_end}}
&
\sphinxAtStartPar

\\
\hline
\sphinxAtStartPar
{\hyperref[\detokenize{_autosummary/signalanalysis.vcg.Vcg:signalanalysis.vcg.Vcg.reset}]{\sphinxcrossref{\sphinxcode{\sphinxupquote{reset}}}}}
&
\sphinxAtStartPar
Reset all properties of the class
\\
\hline
\end{longtable}\sphinxatlongtableend\end{savenotes}
\index{apply\_filter() (signalanalysis.vcg.Vcg method)@\spxentry{apply\_filter()}\spxextra{signalanalysis.vcg.Vcg method}}

\begin{fulllineitems}
\phantomsection\label{\detokenize{_autosummary/signalanalysis.vcg.Vcg:signalanalysis.vcg.Vcg.apply_filter}}\pysiglinewithargsret{\sphinxbfcode{\sphinxupquote{apply\_filter}}}{\emph{\DUrole{o}{**}\DUrole{n}{kwargs}}}{}
\sphinxAtStartPar
Apply a given filter to the data, using their respective arguments as required


\sphinxstrong{See also:}
\nopagebreak

\begin{description}
\item[{{\hyperref[\detokenize{_autosummary/tools.maths.filter_butterworth:tools.maths.filter_butterworth}]{\sphinxcrossref{\sphinxcode{\sphinxupquote{tools.maths.filter\_butterworth()}}}}}}] \leavevmode
\sphinxAtStartPar
Potential filter

\item[{{\hyperref[\detokenize{_autosummary/tools.maths.filter_savitzkygolay:tools.maths.filter_savitzkygolay}]{\sphinxcrossref{\sphinxcode{\sphinxupquote{tools.maths.filter\_savitzkygolay()}}}}}}] \leavevmode
\sphinxAtStartPar
Potential filter

\end{description}



\end{fulllineitems}

\index{get\_from\_ecg() (signalanalysis.vcg.Vcg method)@\spxentry{get\_from\_ecg()}\spxextra{signalanalysis.vcg.Vcg method}}

\begin{fulllineitems}
\phantomsection\label{\detokenize{_autosummary/signalanalysis.vcg.Vcg:id0}}\pysiglinewithargsret{\sphinxbfcode{\sphinxupquote{get\_from\_ecg}}}{\emph{\DUrole{n}{ecg}\DUrole{p}{:} \DUrole{n}{{\hyperref[\detokenize{_autosummary/signalanalysis.ecg.Ecg:signalanalysis.ecg.Ecg}]{\sphinxcrossref{signalanalysis.ecg.Ecg}}}}}}{}
\sphinxAtStartPar
Convert ECG data to vectorcardiogram (VCG) data using the Kors matrix method
\begin{quote}\begin{description}
\item[{Parameters}] \leavevmode
\sphinxAtStartPar
\sphinxstyleliteralstrong{\sphinxupquote{ecg}} ({\hyperref[\detokenize{_autosummary/signalanalysis.ecg.Ecg:signalanalysis.ecg.Ecg}]{\sphinxcrossref{\sphinxstyleliteralemphasis{\sphinxupquote{signalanalysis.ecg.Ecg}}}}}) \textendash{} List of ECG dataframe data, or ECG dataframe data directly, with dict keys corresponding to ECG outputs

\end{description}\end{quote}
\subsubsection*{References}
\begin{description}
\item[{Kors JA, van Herpen G, Sittig AC, van Bemmel JH.}] \leavevmode
\sphinxAtStartPar
Reconstruction of the Frank vectorcardiogram from standard electrocardiographic leads: diagnostic comparison
of different methods
Eur Heart J. 1990 Dec;11(12):1083\sphinxhyphen{}92.

\end{description}

\end{fulllineitems}

\index{get\_n\_beats() (signalanalysis.vcg.Vcg method)@\spxentry{get\_n\_beats()}\spxextra{signalanalysis.vcg.Vcg method}}

\begin{fulllineitems}
\phantomsection\label{\detokenize{_autosummary/signalanalysis.vcg.Vcg:signalanalysis.vcg.Vcg.get_n_beats}}\pysiglinewithargsret{\sphinxbfcode{\sphinxupquote{get\_n\_beats}}}{\emph{\DUrole{n}{threshold}\DUrole{p}{:} \DUrole{n}{float} \DUrole{o}{=} \DUrole{default_value}{0.5}}, \emph{\DUrole{n}{min\_separation}\DUrole{p}{:} \DUrole{n}{float} \DUrole{o}{=} \DUrole{default_value}{0.2}}, \emph{\DUrole{n}{reset\_index}\DUrole{p}{:} \DUrole{n}{bool} \DUrole{o}{=} \DUrole{default_value}{True}}, \emph{\DUrole{n}{plot}\DUrole{p}{:} \DUrole{n}{bool} \DUrole{o}{=} \DUrole{default_value}{False}}, \emph{\DUrole{o}{**}\DUrole{n}{kwargs}}}{}
\sphinxAtStartPar
Calculate the number of beats in a given signal, and save the individual beats to the object for later use

\sphinxAtStartPar
When given the raw data of a given signal (ECG or VCG), will estimate the number of beats recorded in the trace
based on the RMS of the signal exceeding a threshold value. The estimated individual beats will then be saved in
a list in a lossless manner, i.e. saved as {[}ECG1, ECG2, …, ECG(n){]}, where ECG1={[}0:peak2{]}, ECG2={[}peak1:peak3{]},
…, ECGn={[}peak(n\sphinxhyphen{}1):end{]}
\begin{quote}\begin{description}
\item[{Parameters}] \leavevmode\begin{itemize}
\item {} 
\sphinxAtStartPar
\sphinxstyleliteralstrong{\sphinxupquote{threshold}} (\sphinxstyleliteralemphasis{\sphinxupquote{float \{0\textless{}1\}}}) \textendash{} Minimum value to search for for a peak in RMS signal to determine when a beat has occurred, default=0.5

\item {} 
\sphinxAtStartPar
\sphinxstyleliteralstrong{\sphinxupquote{min\_separation}} (\sphinxstyleliteralemphasis{\sphinxupquote{float}}) \textendash{} Minimum time (in s) that should be used to separate separate beats, default=0.2s

\item {} 
\sphinxAtStartPar
\sphinxstyleliteralstrong{\sphinxupquote{reset\_index}} (\sphinxstyleliteralemphasis{\sphinxupquote{bool}}) \textendash{} Whether to reset the time index for the separated beats so that they all start from zero (true),
or whether to leave them with the original time index (false), default=True

\item {} 
\sphinxAtStartPar
\sphinxstyleliteralstrong{\sphinxupquote{plot}} (\sphinxstyleliteralemphasis{\sphinxupquote{bool}}) \textendash{} Whether to plot results of beat detection, default=False

\item {} 
\sphinxAtStartPar
\sphinxstyleliteralstrong{\sphinxupquote{unipolar\_only}} (\sphinxstyleliteralemphasis{\sphinxupquote{bool}}\sphinxstyleliteralemphasis{\sphinxupquote{, }}\sphinxstyleliteralemphasis{\sphinxupquote{optional}}) \textendash{} Only appropriate for ECG data. Whether to use only unipolar ECG leads to calculate RMS, default=True

\end{itemize}

\item[{Returns}] \leavevmode
\sphinxAtStartPar
\sphinxstylestrong{self.n\_beats} \textendash{} Number of beats detected in signal

\item[{Return type}] \leavevmode
\sphinxAtStartPar
int

\end{description}\end{quote}


\sphinxstrong{See also:}
\nopagebreak

\begin{description}
\item[{{\hyperref[\detokenize{_autosummary/signalanalysis.general.Signal:id1}]{\sphinxcrossref{\sphinxcode{\sphinxupquote{signalanalysis.general.Signal.get\_rms()}}}}}}] \leavevmode
\sphinxAtStartPar
RMS signal calculation required for getting n\_beats

\end{description}



\end{fulllineitems}

\index{get\_rms() (signalanalysis.vcg.Vcg method)@\spxentry{get\_rms()}\spxextra{signalanalysis.vcg.Vcg method}}

\begin{fulllineitems}
\phantomsection\label{\detokenize{_autosummary/signalanalysis.vcg.Vcg:signalanalysis.vcg.Vcg.get_rms}}\pysiglinewithargsret{\sphinxbfcode{\sphinxupquote{get\_rms}}}{\emph{\DUrole{n}{preprocess\_data}\DUrole{p}{:} \DUrole{n}{Optional\DUrole{p}{{[}}pandas.core.frame.DataFrame\DUrole{p}{{]}}} \DUrole{o}{=} \DUrole{default_value}{None}}, \emph{\DUrole{n}{drop\_columns}\DUrole{p}{:} \DUrole{n}{Optional\DUrole{p}{{[}}List\DUrole{p}{{[}}str\DUrole{p}{{]}}\DUrole{p}{{]}}} \DUrole{o}{=} \DUrole{default_value}{None}}}{}
\sphinxAtStartPar
Returns the RMS of the combined signal
\begin{quote}\begin{description}
\item[{Parameters}] \leavevmode\begin{itemize}
\item {} 
\sphinxAtStartPar
\sphinxstyleliteralstrong{\sphinxupquote{preprocess\_data}} (\sphinxstyleliteralemphasis{\sphinxupquote{pd.DataFrame}}\sphinxstyleliteralemphasis{\sphinxupquote{, }}\sphinxstyleliteralemphasis{\sphinxupquote{optional}}) \textendash{} Only passed if there is some extant data that is to be used for getting the RMS (for example,
if the unipolar data only from ECG is being used, and the data is thus preprocessed in a manner specific
for ECG data in the ECG routine)

\item {} 
\sphinxAtStartPar
\sphinxstyleliteralstrong{\sphinxupquote{drop\_columns}} (\sphinxstyleliteralemphasis{\sphinxupquote{list of str}}\sphinxstyleliteralemphasis{\sphinxupquote{, }}\sphinxstyleliteralemphasis{\sphinxupquote{optional}}) \textendash{} List of any columns to drop from the raw data before calculating the RMS. Can be used in conjunction with
preprocess\_data

\end{itemize}

\item[{Returns}] \leavevmode
\sphinxAtStartPar
\sphinxstylestrong{self.rms} \textendash{} RMS signal of all combined leads

\item[{Return type}] \leavevmode
\sphinxAtStartPar
pd.Series

\end{description}\end{quote}
\subsubsection*{Notes}

\sphinxAtStartPar
The scalar RMS is calculated according to
\begin{equation*}
\begin{split}\sqrt{ \frac{1}{n}\sum_{i=1}^n (\textnormal{ECG}_i^2(t)) }\end{split}
\end{equation*}
\sphinxAtStartPar
for all leads available from the signal (12 for ECG, 3 for VCG), unless some leads are excluded via the
drop\_columns parameter.

\end{fulllineitems}

\index{reset() (signalanalysis.vcg.Vcg method)@\spxentry{reset()}\spxextra{signalanalysis.vcg.Vcg method}}

\begin{fulllineitems}
\phantomsection\label{\detokenize{_autosummary/signalanalysis.vcg.Vcg:signalanalysis.vcg.Vcg.reset}}\pysiglinewithargsret{\sphinxbfcode{\sphinxupquote{reset}}}{}{}
\sphinxAtStartPar
Reset all properties of the class

\sphinxAtStartPar
Function called when reading in new data into an existing class (for some reason), which would make these
properties and attributes clash with the other data

\end{fulllineitems}


\end{fulllineitems}



\section{signalplot}
\label{\detokenize{_autosummary/signalplot:module-signalplot}}\label{\detokenize{_autosummary/signalplot:signalplot}}\label{\detokenize{_autosummary/signalplot::doc}}\index{module@\spxentry{module}!signalplot@\spxentry{signalplot}}\index{signalplot@\spxentry{signalplot}!module@\spxentry{module}}

\begin{savenotes}\sphinxatlongtablestart\begin{longtable}[c]{\X{1}{2}\X{1}{2}}
\hline

\endfirsthead

\multicolumn{2}{c}%
{\makebox[0pt]{\sphinxtablecontinued{\tablename\ \thetable{} \textendash{} continued from previous page}}}\\
\hline

\endhead

\hline
\multicolumn{2}{r}{\makebox[0pt][r]{\sphinxtablecontinued{continues on next page}}}\\
\endfoot

\endlastfoot

\sphinxAtStartPar
{\hyperref[\detokenize{_autosummary/signalplot.ecg:module-signalplot.ecg}]{\sphinxcrossref{\sphinxcode{\sphinxupquote{signalplot.ecg}}}}}
&
\sphinxAtStartPar

\\
\hline
\sphinxAtStartPar
{\hyperref[\detokenize{_autosummary/signalplot.general:module-signalplot.general}]{\sphinxcrossref{\sphinxcode{\sphinxupquote{signalplot.general}}}}}
&
\sphinxAtStartPar

\\
\hline
\sphinxAtStartPar
{\hyperref[\detokenize{_autosummary/signalplot.vcg:module-signalplot.vcg}]{\sphinxcrossref{\sphinxcode{\sphinxupquote{signalplot.vcg}}}}}
&
\sphinxAtStartPar

\\
\hline
\end{longtable}\sphinxatlongtableend\end{savenotes}


\subsection{signalplot.ecg}
\label{\detokenize{_autosummary/signalplot.ecg:module-signalplot.ecg}}\label{\detokenize{_autosummary/signalplot.ecg:signalplot-ecg}}\label{\detokenize{_autosummary/signalplot.ecg::doc}}\index{module@\spxentry{module}!signalplot.ecg@\spxentry{signalplot.ecg}}\index{signalplot.ecg@\spxentry{signalplot.ecg}!module@\spxentry{module}}\subsubsection*{Functions}


\begin{savenotes}\sphinxatlongtablestart\begin{longtable}[c]{\X{1}{2}\X{1}{2}}
\hline

\endfirsthead

\multicolumn{2}{c}%
{\makebox[0pt]{\sphinxtablecontinued{\tablename\ \thetable{} \textendash{} continued from previous page}}}\\
\hline

\endhead

\hline
\multicolumn{2}{r}{\makebox[0pt][r]{\sphinxtablecontinued{continues on next page}}}\\
\endfoot

\endlastfoot

\sphinxAtStartPar
{\hyperref[\detokenize{_autosummary/signalplot.ecg.plot:signalplot.ecg.plot}]{\sphinxcrossref{\sphinxcode{\sphinxupquote{plot}}}}}
&
\sphinxAtStartPar
Plot and label the ECG data from simulation(s).
\\
\hline
\end{longtable}\sphinxatlongtableend\end{savenotes}


\subsubsection{signalplot.ecg.plot}
\label{\detokenize{_autosummary/signalplot.ecg.plot:signalplot-ecg-plot}}\label{\detokenize{_autosummary/signalplot.ecg.plot::doc}}\index{plot() (in module signalplot.ecg)@\spxentry{plot()}\spxextra{in module signalplot.ecg}}

\begin{fulllineitems}
\phantomsection\label{\detokenize{_autosummary/signalplot.ecg.plot:signalplot.ecg.plot}}\pysiglinewithargsret{\sphinxcode{\sphinxupquote{signalplot.ecg.}}\sphinxbfcode{\sphinxupquote{plot}}}{\emph{\DUrole{n}{ecgs}\DUrole{p}{:} \DUrole{n}{Union\DUrole{p}{{[}}List\DUrole{p}{{[}}pandas.core.frame.DataFrame\DUrole{p}{{]}}\DUrole{p}{, }pandas.core.frame.DataFrame\DUrole{p}{{]}}}}, \emph{\DUrole{n}{legend\_ecg}\DUrole{p}{:} \DUrole{n}{Optional\DUrole{p}{{[}}List\DUrole{p}{{[}}str\DUrole{p}{{]}}\DUrole{p}{{]}}} \DUrole{o}{=} \DUrole{default_value}{None}}, \emph{\DUrole{n}{linewidths\_ecg}\DUrole{p}{:} \DUrole{n}{float} \DUrole{o}{=} \DUrole{default_value}{2}}, \emph{\DUrole{n}{limits}\DUrole{p}{:} \DUrole{n}{Optional\DUrole{p}{{[}}Union\DUrole{p}{{[}}list\DUrole{p}{, }float\DUrole{p}{{]}}\DUrole{p}{{]}}} \DUrole{o}{=} \DUrole{default_value}{None}}, \emph{\DUrole{n}{legend\_limits}\DUrole{p}{:} \DUrole{n}{Optional\DUrole{p}{{[}}List\DUrole{p}{{[}}str\DUrole{p}{{]}}\DUrole{p}{{]}}} \DUrole{o}{=} \DUrole{default_value}{None}}, \emph{\DUrole{n}{plot\_sequence}\DUrole{p}{:} \DUrole{n}{Optional\DUrole{p}{{[}}List\DUrole{p}{{[}}str\DUrole{p}{{]}}\DUrole{p}{{]}}} \DUrole{o}{=} \DUrole{default_value}{None}}, \emph{\DUrole{n}{single\_fig}\DUrole{p}{:} \DUrole{n}{bool} \DUrole{o}{=} \DUrole{default_value}{True}}, \emph{\DUrole{n}{colours\_ecg}\DUrole{p}{:} \DUrole{n}{Optional\DUrole{p}{{[}}Union\DUrole{p}{{[}}List\DUrole{p}{{[}}str\DUrole{p}{{]}}\DUrole{p}{, }List\DUrole{p}{{[}}List\DUrole{p}{{[}}float\DUrole{p}{{]}}\DUrole{p}{{]}}\DUrole{p}{, }List\DUrole{p}{{[}}Tuple\DUrole{p}{{[}}float\DUrole{p}{{]}}\DUrole{p}{{]}}\DUrole{p}{{]}}\DUrole{p}{{]}}} \DUrole{o}{=} \DUrole{default_value}{None}}, \emph{\DUrole{n}{linestyles\_ecg}\DUrole{p}{:} \DUrole{n}{Optional\DUrole{p}{{[}}List\DUrole{p}{{[}}str\DUrole{p}{{]}}\DUrole{p}{{]}}} \DUrole{o}{=} \DUrole{default_value}{\textquotesingle{}\sphinxhyphen{}\textquotesingle{}}}, \emph{\DUrole{n}{colours\_limits}\DUrole{p}{:} \DUrole{n}{Optional\DUrole{p}{{[}}Union\DUrole{p}{{[}}List\DUrole{p}{{[}}str\DUrole{p}{{]}}\DUrole{p}{, }List\DUrole{p}{{[}}List\DUrole{p}{{[}}float\DUrole{p}{{]}}\DUrole{p}{{]}}\DUrole{p}{, }List\DUrole{p}{{[}}Tuple\DUrole{p}{{[}}float\DUrole{p}{{]}}\DUrole{p}{{]}}\DUrole{p}{{]}}\DUrole{p}{{]}}} \DUrole{o}{=} \DUrole{default_value}{None}}, \emph{\DUrole{n}{linestyles\_limits}\DUrole{p}{:} \DUrole{n}{Optional\DUrole{p}{{[}}List\DUrole{p}{{[}}str\DUrole{p}{{]}}\DUrole{p}{{]}}} \DUrole{o}{=} \DUrole{default_value}{None}}, \emph{\DUrole{n}{fig}\DUrole{p}{:} \DUrole{n}{Optional\DUrole{p}{{[}}matplotlib.pyplot.figure\DUrole{p}{{]}}} \DUrole{o}{=} \DUrole{default_value}{None}}, \emph{\DUrole{n}{ax}\DUrole{o}{=}\DUrole{default_value}{None}}}{{ $\rightarrow$ tuple}}
\sphinxAtStartPar
Plot and label the ECG data from simulation(s). Optional to add in QRS start/end boundaries for plotting
\begin{quote}\begin{description}
\item[{Parameters}] \leavevmode\begin{itemize}
\item {} 
\sphinxAtStartPar
\sphinxstyleliteralstrong{\sphinxupquote{ecgs}} (\sphinxstyleliteralemphasis{\sphinxupquote{pd.DataFrame}}\sphinxstyleliteralemphasis{\sphinxupquote{ or }}\sphinxstyleliteralemphasis{\sphinxupquote{list of pd.DataFrame}}) \textendash{} Dataframe or list of dataframes for ECG data, with keys corresponding to the trace name and index to the time
data

\item {} 
\sphinxAtStartPar
\sphinxstyleliteralstrong{\sphinxupquote{legend\_ecg}} (\sphinxstyleliteralemphasis{\sphinxupquote{list of str}}\sphinxstyleliteralemphasis{\sphinxupquote{, }}\sphinxstyleliteralemphasis{\sphinxupquote{optional}}) \textendash{} List of names for each given set of ECG data e.g. {[}‘BCL=300ms’, ‘BCL=600ms’{]}, default=None

\item {} 
\sphinxAtStartPar
\sphinxstyleliteralstrong{\sphinxupquote{linewidths\_ecg}} (\sphinxstyleliteralemphasis{\sphinxupquote{float}}\sphinxstyleliteralemphasis{\sphinxupquote{, }}\sphinxstyleliteralemphasis{\sphinxupquote{optional}}) \textendash{} Width to use for plotting lines, default=3

\item {} 
\sphinxAtStartPar
\sphinxstyleliteralstrong{\sphinxupquote{limits}} (\sphinxstyleliteralemphasis{\sphinxupquote{float}}\sphinxstyleliteralemphasis{\sphinxupquote{ or }}\sphinxstyleliteralemphasis{\sphinxupquote{list of float}}\sphinxstyleliteralemphasis{\sphinxupquote{ or }}\sphinxstyleliteralemphasis{\sphinxupquote{pd.DataFrame}}\sphinxstyleliteralemphasis{\sphinxupquote{, }}\sphinxstyleliteralemphasis{\sphinxupquote{optional}}) \textendash{} Optional temporal limits (e.g. QRS limits) to add to ECG plots. Can add multiple limits, which will be
plotted identically on all axes. If provided as a dataframe, will plot the limits on the relevant axis

\item {} 
\sphinxAtStartPar
\sphinxstyleliteralstrong{\sphinxupquote{legend\_limits}} (\sphinxstyleliteralemphasis{\sphinxupquote{list of str}}\sphinxstyleliteralemphasis{\sphinxupquote{, }}\sphinxstyleliteralemphasis{\sphinxupquote{optional}}) \textendash{} List of names for each given set of limits e.g. {[}‘QRS start’, ‘QRS end’{]}, default=None

\item {} 
\sphinxAtStartPar
\sphinxstyleliteralstrong{\sphinxupquote{plot\_sequence}} (\sphinxstyleliteralemphasis{\sphinxupquote{list of str}}\sphinxstyleliteralemphasis{\sphinxupquote{, }}\sphinxstyleliteralemphasis{\sphinxupquote{optional}}) \textendash{} Sequence in which to plot the ECG traces. Will default to: V1, V2, V3, V4, V5, V6, LI, LII, LIII, aVR, aVL, aVF

\item {} 
\sphinxAtStartPar
\sphinxstyleliteralstrong{\sphinxupquote{single\_fig}} (\sphinxstyleliteralemphasis{\sphinxupquote{bool}}\sphinxstyleliteralemphasis{\sphinxupquote{, }}\sphinxstyleliteralemphasis{\sphinxupquote{optional}}) \textendash{} If true, will plot all axes on a single figure window. If false, will plot each axis on a separate figure
window. Default is True

\item {} 
\sphinxAtStartPar
\sphinxstyleliteralstrong{\sphinxupquote{colours\_ecg}} (\sphinxstyleliteralemphasis{\sphinxupquote{str}}\sphinxstyleliteralemphasis{\sphinxupquote{ or }}\sphinxstyleliteralemphasis{\sphinxupquote{list of str}}\sphinxstyleliteralemphasis{\sphinxupquote{ or }}\sphinxstyleliteralemphasis{\sphinxupquote{list of list/tuple of float}}\sphinxstyleliteralemphasis{\sphinxupquote{, }}\sphinxstyleliteralemphasis{\sphinxupquote{optional}}) \textendash{} Colours to be used to plot ECG traces. Can provide as either string (e.g. ‘b’) or as RGB values (floats). Will
default to ca.get\_plot\_colours()

\item {} 
\sphinxAtStartPar
\sphinxstyleliteralstrong{\sphinxupquote{linestyles\_ecg}} (\sphinxstyleliteralemphasis{\sphinxupquote{str}}\sphinxstyleliteralemphasis{\sphinxupquote{ or }}\sphinxstyleliteralemphasis{\sphinxupquote{list}}\sphinxstyleliteralemphasis{\sphinxupquote{, }}\sphinxstyleliteralemphasis{\sphinxupquote{optional}}) \textendash{} Linestyles to be used to plot ECG traces. Will default to ca.get\_plot\_lines()

\item {} 
\sphinxAtStartPar
\sphinxstyleliteralstrong{\sphinxupquote{colours\_limits}} (\sphinxstyleliteralemphasis{\sphinxupquote{str}}\sphinxstyleliteralemphasis{\sphinxupquote{ or }}\sphinxstyleliteralemphasis{\sphinxupquote{list of str}}\sphinxstyleliteralemphasis{\sphinxupquote{ or }}\sphinxstyleliteralemphasis{\sphinxupquote{list of list/tuple of float}}\sphinxstyleliteralemphasis{\sphinxupquote{, }}\sphinxstyleliteralemphasis{\sphinxupquote{optional}}) \textendash{} Colours to be used to plot limits. Can provide as either string (e.g. ‘b’) or as RGB values (floats). Will
default to ca.get\_plot\_colours()

\item {} 
\sphinxAtStartPar
\sphinxstyleliteralstrong{\sphinxupquote{linestyles\_limits}} (\sphinxstyleliteralemphasis{\sphinxupquote{str}}\sphinxstyleliteralemphasis{\sphinxupquote{ or }}\sphinxstyleliteralemphasis{\sphinxupquote{list}}\sphinxstyleliteralemphasis{\sphinxupquote{, }}\sphinxstyleliteralemphasis{\sphinxupquote{optional}}) \textendash{} Linestyles to be used to plot limits. Will default to ca.get\_plot\_lines()

\item {} 
\sphinxAtStartPar
\sphinxstyleliteralstrong{\sphinxupquote{fig}} (\sphinxstyleliteralemphasis{\sphinxupquote{optional}}) \textendash{} If given, will plot data on existing figure window

\item {} 
\sphinxAtStartPar
\sphinxstyleliteralstrong{\sphinxupquote{ax}} (\sphinxstyleliteralemphasis{\sphinxupquote{optional}}) \textendash{} If given, will plot data using existing axis handles

\end{itemize}

\item[{Returns}] \leavevmode
\sphinxAtStartPar
\begin{itemize}
\item {} 
\sphinxAtStartPar
\sphinxstyleemphasis{fig} \textendash{} Handle to output figure window, or dictionary to several handles if traces are all plotted in separate figure
windows (if single\_fig=False)

\item {} 
\sphinxAtStartPar
\sphinxstylestrong{ax} (\sphinxstyleemphasis{dict}) \textendash{} Dictionary to axis handles for ECG traces

\end{itemize}


\item[{Raises}] \leavevmode\begin{itemize}
\item {} 
\sphinxAtStartPar
\sphinxstyleliteralstrong{\sphinxupquote{AssertionError}} \textendash{} Checks that various list lengths are the same

\item {} 
\sphinxAtStartPar
\sphinxstyleliteralstrong{\sphinxupquote{TypeError}} \textendash{} If input argument is given in an unexpected format

\end{itemize}

\end{description}\end{quote}

\end{fulllineitems}



\subsection{signalplot.general}
\label{\detokenize{_autosummary/signalplot.general:module-signalplot.general}}\label{\detokenize{_autosummary/signalplot.general:signalplot-general}}\label{\detokenize{_autosummary/signalplot.general::doc}}\index{module@\spxentry{module}!signalplot.general@\spxentry{signalplot.general}}\index{signalplot.general@\spxentry{signalplot.general}!module@\spxentry{module}}

\subsection{signalplot.vcg}
\label{\detokenize{_autosummary/signalplot.vcg:module-signalplot.vcg}}\label{\detokenize{_autosummary/signalplot.vcg:signalplot-vcg}}\label{\detokenize{_autosummary/signalplot.vcg::doc}}\index{module@\spxentry{module}!signalplot.vcg@\spxentry{signalplot.vcg}}\index{signalplot.vcg@\spxentry{signalplot.vcg}!module@\spxentry{module}}\subsubsection*{Functions}


\begin{savenotes}\sphinxatlongtablestart\begin{longtable}[c]{\X{1}{2}\X{1}{2}}
\hline

\endfirsthead

\multicolumn{2}{c}%
{\makebox[0pt]{\sphinxtablecontinued{\tablename\ \thetable{} \textendash{} continued from previous page}}}\\
\hline

\endhead

\hline
\multicolumn{2}{r}{\makebox[0pt][r]{\sphinxtablecontinued{continues on next page}}}\\
\endfoot

\endlastfoot

\sphinxAtStartPar
{\hyperref[\detokenize{_autosummary/signalplot.vcg.add_unit_sphere:signalplot.vcg.add_unit_sphere}]{\sphinxcrossref{\sphinxcode{\sphinxupquote{add\_unit\_sphere}}}}}
&
\sphinxAtStartPar
Add a unit sphere to a 3D plot
\\
\hline
\sphinxAtStartPar
{\hyperref[\detokenize{_autosummary/signalplot.vcg.animate_3d:signalplot.vcg.animate_3d}]{\sphinxcrossref{\sphinxcode{\sphinxupquote{animate\_3d}}}}}
&
\sphinxAtStartPar
Animate the evolution of the VCG in 3D space, saving that animation to a file.
\\
\hline
\sphinxAtStartPar
{\hyperref[\detokenize{_autosummary/signalplot.vcg.plot_2d:signalplot.vcg.plot_2d}]{\sphinxcrossref{\sphinxcode{\sphinxupquote{plot\_2d}}}}}
&
\sphinxAtStartPar
Plot x vs y (or y vs z, or other combination) for VCG trace, with line colour shifting to show time progression.
\\
\hline
\sphinxAtStartPar
{\hyperref[\detokenize{_autosummary/signalplot.vcg.plot_3d:signalplot.vcg.plot_3d}]{\sphinxcrossref{\sphinxcode{\sphinxupquote{plot\_3d}}}}}
&
\sphinxAtStartPar
Plot the evolution of VCG in 3D space
\\
\hline
\sphinxAtStartPar
{\hyperref[\detokenize{_autosummary/signalplot.vcg.plot_arc3d:signalplot.vcg.plot_arc3d}]{\sphinxcrossref{\sphinxcode{\sphinxupquote{plot\_arc3d}}}}}
&
\sphinxAtStartPar
Plot arc between two given vectors in 3D space.
\\
\hline
\sphinxAtStartPar
{\hyperref[\detokenize{_autosummary/signalplot.vcg.plot_density_effect:signalplot.vcg.plot_density_effect}]{\sphinxcrossref{\sphinxcode{\sphinxupquote{plot\_density\_effect}}}}}
&
\sphinxAtStartPar
Plot the effect of density on metrics.
\\
\hline
\sphinxAtStartPar
{\hyperref[\detokenize{_autosummary/signalplot.vcg.plot_metric_change:signalplot.vcg.plot_metric_change}]{\sphinxcrossref{\sphinxcode{\sphinxupquote{plot\_metric\_change}}}}}
&
\sphinxAtStartPar
Function to plot all the various figures for trend analysis in one go.
\\
\hline
\sphinxAtStartPar
{\hyperref[\detokenize{_autosummary/signalplot.vcg.plot_metric_change_barplot:signalplot.vcg.plot_metric_change_barplot}]{\sphinxcrossref{\sphinxcode{\sphinxupquote{plot\_metric\_change\_barplot}}}}}
&
\sphinxAtStartPar
Plots a bar chart for the observed metrics.
\\
\hline
\sphinxAtStartPar
{\hyperref[\detokenize{_autosummary/signalplot.vcg.plot_spatial_velocity:signalplot.vcg.plot_spatial_velocity}]{\sphinxcrossref{\sphinxcode{\sphinxupquote{plot\_spatial\_velocity}}}}}
&
\sphinxAtStartPar
Plot the spatial velocity for given VCG data
\\
\hline
\sphinxAtStartPar
{\hyperref[\detokenize{_autosummary/signalplot.vcg.plot_xyz_components:signalplot.vcg.plot_xyz_components}]{\sphinxcrossref{\sphinxcode{\sphinxupquote{plot\_xyz\_components}}}}}
&
\sphinxAtStartPar
Plot x, y, z components of VCG data
\\
\hline
\sphinxAtStartPar
{\hyperref[\detokenize{_autosummary/signalplot.vcg.plot_xyz_vector:signalplot.vcg.plot_xyz_vector}]{\sphinxcrossref{\sphinxcode{\sphinxupquote{plot\_xyz\_vector}}}}}
&
\sphinxAtStartPar
Plots a specific vector in 3D space (e.g.
\\
\hline
\end{longtable}\sphinxatlongtableend\end{savenotes}


\subsubsection{signalplot.vcg.add\_unit\_sphere}
\label{\detokenize{_autosummary/signalplot.vcg.add_unit_sphere:signalplot-vcg-add-unit-sphere}}\label{\detokenize{_autosummary/signalplot.vcg.add_unit_sphere::doc}}\index{add\_unit\_sphere() (in module signalplot.vcg)@\spxentry{add\_unit\_sphere()}\spxextra{in module signalplot.vcg}}

\begin{fulllineitems}
\phantomsection\label{\detokenize{_autosummary/signalplot.vcg.add_unit_sphere:signalplot.vcg.add_unit_sphere}}\pysiglinewithargsret{\sphinxcode{\sphinxupquote{signalplot.vcg.}}\sphinxbfcode{\sphinxupquote{add\_unit\_sphere}}}{\emph{\DUrole{n}{ax}}}{{ $\rightarrow$ None}}
\sphinxAtStartPar
Add a unit sphere to a 3D plot
\begin{quote}\begin{description}
\item[{Parameters}] \leavevmode
\sphinxAtStartPar
\sphinxstyleliteralstrong{\sphinxupquote{ax}} \textendash{} Handles to axes

\end{description}\end{quote}

\end{fulllineitems}



\subsubsection{signalplot.vcg.animate\_3d}
\label{\detokenize{_autosummary/signalplot.vcg.animate_3d:signalplot-vcg-animate-3d}}\label{\detokenize{_autosummary/signalplot.vcg.animate_3d::doc}}\index{animate\_3d() (in module signalplot.vcg)@\spxentry{animate\_3d()}\spxextra{in module signalplot.vcg}}

\begin{fulllineitems}
\phantomsection\label{\detokenize{_autosummary/signalplot.vcg.animate_3d:signalplot.vcg.animate_3d}}\pysiglinewithargsret{\sphinxcode{\sphinxupquote{signalplot.vcg.}}\sphinxbfcode{\sphinxupquote{animate\_3d}}}{\emph{\DUrole{n}{vcg}\DUrole{p}{:} \DUrole{n}{numpy.ndarray}}, \emph{\DUrole{n}{limits}\DUrole{p}{:} \DUrole{n}{Optional\DUrole{p}{{[}}Union\DUrole{p}{{[}}float\DUrole{p}{, }List\DUrole{p}{{[}}float\DUrole{p}{{]}}\DUrole{p}{, }List\DUrole{p}{{[}}List\DUrole{p}{{[}}float\DUrole{p}{{]}}\DUrole{p}{{]}}\DUrole{p}{{]}}\DUrole{p}{{]}}} \DUrole{o}{=} \DUrole{default_value}{None}}, \emph{\DUrole{n}{linestyle}\DUrole{p}{:} \DUrole{n}{Optional\DUrole{p}{{[}}str\DUrole{p}{{]}}} \DUrole{o}{=} \DUrole{default_value}{\textquotesingle{}\sphinxhyphen{}\textquotesingle{}}}, \emph{\DUrole{n}{colourmap}\DUrole{p}{:} \DUrole{n}{Optional\DUrole{p}{{[}}str\DUrole{p}{{]}}} \DUrole{o}{=} \DUrole{default_value}{\textquotesingle{}viridis\textquotesingle{}}}, \emph{\DUrole{n}{linewidth}\DUrole{p}{:} \DUrole{n}{Optional\DUrole{p}{{[}}float\DUrole{p}{{]}}} \DUrole{o}{=} \DUrole{default_value}{3}}, \emph{\DUrole{n}{output\_file}\DUrole{p}{:} \DUrole{n}{Optional\DUrole{p}{{[}}str\DUrole{p}{{]}}} \DUrole{o}{=} \DUrole{default_value}{\textquotesingle{}vcg\_xyz.mp4\textquotesingle{}}}}{{ $\rightarrow$ None}}
\sphinxAtStartPar
Animate the evolution of the VCG in 3D space, saving that animation to a file.
\begin{quote}\begin{description}
\item[{Parameters}] \leavevmode\begin{itemize}
\item {} 
\sphinxAtStartPar
\sphinxstyleliteralstrong{\sphinxupquote{vcg}} (\sphinxstyleliteralemphasis{\sphinxupquote{np.ndarray}}) \textendash{} VCG data

\item {} 
\sphinxAtStartPar
\sphinxstyleliteralstrong{\sphinxupquote{limits}} (\sphinxstyleliteralemphasis{\sphinxupquote{float}}\sphinxstyleliteralemphasis{\sphinxupquote{ or }}\sphinxstyleliteralemphasis{\sphinxupquote{list of float}}\sphinxstyleliteralemphasis{\sphinxupquote{ or }}\sphinxstyleliteralemphasis{\sphinxupquote{list of list of floats}}\sphinxstyleliteralemphasis{\sphinxupquote{, }}\sphinxstyleliteralemphasis{\sphinxupquote{optional}}) \textendash{} \begin{description}
\item[{Limits for the axes. If none, will set to the min/max values of the provided data. Can provide either as:}] \leavevmode\begin{enumerate}
\sphinxsetlistlabels{\arabic}{enumi}{enumii}{}{)}%
\item {} 
\sphinxAtStartPar
a single value (+/\sphinxhyphen{} of that value applied to all axes)

\item {} 
\sphinxAtStartPar
{[}min, max{]} to be applied to all axes

\item {} 
\sphinxAtStartPar
{[}{[}xmin, xmax{]}, {[}ymin, ymax{]}, {[}zmin, zmax{]}{]}

\end{enumerate}

\end{description}


\item {} 
\sphinxAtStartPar
\sphinxstyleliteralstrong{\sphinxupquote{linestyle}} (\sphinxstyleliteralemphasis{\sphinxupquote{str}}\sphinxstyleliteralemphasis{\sphinxupquote{, }}\sphinxstyleliteralemphasis{\sphinxupquote{optional}}) \textendash{} Linestyle for the data, default=’\sphinxhyphen{}’

\item {} 
\sphinxAtStartPar
\sphinxstyleliteralstrong{\sphinxupquote{colourmap}} (\sphinxstyleliteralemphasis{\sphinxupquote{str}}\sphinxstyleliteralemphasis{\sphinxupquote{, }}\sphinxstyleliteralemphasis{\sphinxupquote{optional}}) \textendash{} Colourmap to use when plotting, default=’viridis’

\item {} 
\sphinxAtStartPar
\sphinxstyleliteralstrong{\sphinxupquote{linewidth}} (\sphinxstyleliteralemphasis{\sphinxupquote{float}}\sphinxstyleliteralemphasis{\sphinxupquote{, }}\sphinxstyleliteralemphasis{\sphinxupquote{optional}}) \textendash{} Linewidth when used to plot VCG, default=3

\item {} 
\sphinxAtStartPar
\sphinxstyleliteralstrong{\sphinxupquote{output\_file}} (\sphinxstyleliteralemphasis{\sphinxupquote{str}}\sphinxstyleliteralemphasis{\sphinxupquote{, }}\sphinxstyleliteralemphasis{\sphinxupquote{optional}}) \textendash{} Name of the file to save the animation to, default=’vcg\_xyz.mp4’

\end{itemize}

\end{description}\end{quote}

\end{fulllineitems}



\subsubsection{signalplot.vcg.plot\_2d}
\label{\detokenize{_autosummary/signalplot.vcg.plot_2d:signalplot-vcg-plot-2d}}\label{\detokenize{_autosummary/signalplot.vcg.plot_2d::doc}}\index{plot\_2d() (in module signalplot.vcg)@\spxentry{plot\_2d()}\spxextra{in module signalplot.vcg}}

\begin{fulllineitems}
\phantomsection\label{\detokenize{_autosummary/signalplot.vcg.plot_2d:signalplot.vcg.plot_2d}}\pysiglinewithargsret{\sphinxcode{\sphinxupquote{signalplot.vcg.}}\sphinxbfcode{\sphinxupquote{plot\_2d}}}{\emph{\DUrole{n}{vcg}\DUrole{p}{:} \DUrole{n}{pandas.core.frame.DataFrame}}, \emph{\DUrole{n}{x\_plot}\DUrole{p}{:} \DUrole{n}{str} \DUrole{o}{=} \DUrole{default_value}{\textquotesingle{}x\textquotesingle{}}}, \emph{\DUrole{n}{y\_plot}\DUrole{p}{:} \DUrole{n}{str} \DUrole{o}{=} \DUrole{default_value}{\textquotesingle{}y\textquotesingle{}}}, \emph{\DUrole{n}{linestyle}\DUrole{p}{:} \DUrole{n}{str} \DUrole{o}{=} \DUrole{default_value}{\textquotesingle{}\sphinxhyphen{}\textquotesingle{}}}, \emph{\DUrole{n}{colourmap}\DUrole{p}{:} \DUrole{n}{str} \DUrole{o}{=} \DUrole{default_value}{\textquotesingle{}viridis\textquotesingle{}}}, \emph{\DUrole{n}{linewidth}\DUrole{p}{:} \DUrole{n}{float} \DUrole{o}{=} \DUrole{default_value}{3}}, \emph{\DUrole{n}{axis\_limits}\DUrole{p}{:} \DUrole{n}{Optional\DUrole{p}{{[}}Union\DUrole{p}{{[}}float\DUrole{p}{, }List\DUrole{p}{{[}}float\DUrole{p}{{]}}\DUrole{p}{{]}}\DUrole{p}{{]}}} \DUrole{o}{=} \DUrole{default_value}{None}}, \emph{\DUrole{n}{fig}\DUrole{p}{:} \DUrole{n}{Optional\DUrole{p}{{[}}matplotlib.pyplot.figure\DUrole{p}{{]}}} \DUrole{o}{=} \DUrole{default_value}{None}}}{{ $\rightarrow$ matplotlib.pyplot.figure}}
\sphinxAtStartPar
Plot x vs y (or y vs z, or other combination) for VCG trace, with line colour shifting to show time progression.

\sphinxAtStartPar
Plot a colour\sphinxhyphen{}varying course of a VCG in 2D space
\begin{quote}\begin{description}
\item[{Parameters}] \leavevmode\begin{itemize}
\item {} 
\sphinxAtStartPar
\sphinxstyleliteralstrong{\sphinxupquote{vcg}} (\sphinxstyleliteralemphasis{\sphinxupquote{pd.DataFrame}}) \textendash{} VCG data to be plotted

\item {} 
\sphinxAtStartPar
\sphinxstyleliteralstrong{\sphinxupquote{x\_plot}} (\sphinxstyleliteralemphasis{\sphinxupquote{str}}\sphinxstyleliteralemphasis{\sphinxupquote{, }}\sphinxstyleliteralemphasis{\sphinxupquote{optional}}) \textendash{} Which components of VCG to plot, default=’x’, ‘y’

\item {} 
\sphinxAtStartPar
\sphinxstyleliteralstrong{\sphinxupquote{y\_plot}} (\sphinxstyleliteralemphasis{\sphinxupquote{str}}\sphinxstyleliteralemphasis{\sphinxupquote{, }}\sphinxstyleliteralemphasis{\sphinxupquote{optional}}) \textendash{} Which components of VCG to plot, default=’x’, ‘y’

\item {} 
\sphinxAtStartPar
\sphinxstyleliteralstrong{\sphinxupquote{linestyle}} (\sphinxstyleliteralemphasis{\sphinxupquote{str}}\sphinxstyleliteralemphasis{\sphinxupquote{, }}\sphinxstyleliteralemphasis{\sphinxupquote{optional}}) \textendash{} Linestyle to apply to the plot, default=’\sphinxhyphen{}’

\item {} 
\sphinxAtStartPar
\sphinxstyleliteralstrong{\sphinxupquote{colourmap}} (\sphinxstyleliteralemphasis{\sphinxupquote{str}}\sphinxstyleliteralemphasis{\sphinxupquote{, }}\sphinxstyleliteralemphasis{\sphinxupquote{optional}}) \textendash{} Colourmap to use for the line, default=’viridis’

\item {} 
\sphinxAtStartPar
\sphinxstyleliteralstrong{\sphinxupquote{linewidth}} (\sphinxstyleliteralemphasis{\sphinxupquote{float}}\sphinxstyleliteralemphasis{\sphinxupquote{, }}\sphinxstyleliteralemphasis{\sphinxupquote{optional}}) \textendash{} Linewidth to use, default=3

\item {} 
\sphinxAtStartPar
\sphinxstyleliteralstrong{\sphinxupquote{axis\_limits}} (\sphinxstyleliteralemphasis{\sphinxupquote{list of float}}\sphinxstyleliteralemphasis{\sphinxupquote{ or }}\sphinxstyleliteralemphasis{\sphinxupquote{float}}\sphinxstyleliteralemphasis{\sphinxupquote{, }}\sphinxstyleliteralemphasis{\sphinxupquote{optional}}) \textendash{} Limits to apply to the axes, default=None

\item {} 
\sphinxAtStartPar
\sphinxstyleliteralstrong{\sphinxupquote{fig}} (\sphinxstyleliteralemphasis{\sphinxupquote{plt.figure}}\sphinxstyleliteralemphasis{\sphinxupquote{, }}\sphinxstyleliteralemphasis{\sphinxupquote{optional}}) \textendash{} Handle to pre\sphinxhyphen{}existing figure (if present) on which to plot data, default=None

\end{itemize}

\item[{Returns}] \leavevmode
\sphinxAtStartPar
\sphinxstylestrong{fig} \textendash{} Handle to output figure window

\item[{Return type}] \leavevmode
\sphinxAtStartPar
plt.figure

\end{description}\end{quote}

\end{fulllineitems}



\subsubsection{signalplot.vcg.plot\_3d}
\label{\detokenize{_autosummary/signalplot.vcg.plot_3d:signalplot-vcg-plot-3d}}\label{\detokenize{_autosummary/signalplot.vcg.plot_3d::doc}}\index{plot\_3d() (in module signalplot.vcg)@\spxentry{plot\_3d()}\spxextra{in module signalplot.vcg}}

\begin{fulllineitems}
\phantomsection\label{\detokenize{_autosummary/signalplot.vcg.plot_3d:signalplot.vcg.plot_3d}}\pysiglinewithargsret{\sphinxcode{\sphinxupquote{signalplot.vcg.}}\sphinxbfcode{\sphinxupquote{plot\_3d}}}{\emph{\DUrole{n}{vcg}\DUrole{p}{:} \DUrole{n}{pandas.core.frame.DataFrame}}, \emph{\DUrole{n}{linestyle}\DUrole{p}{:} \DUrole{n}{str} \DUrole{o}{=} \DUrole{default_value}{\textquotesingle{}\sphinxhyphen{}\textquotesingle{}}}, \emph{\DUrole{n}{colourmap}\DUrole{p}{:} \DUrole{n}{str} \DUrole{o}{=} \DUrole{default_value}{\textquotesingle{}viridis\textquotesingle{}}}, \emph{\DUrole{n}{linewidth}\DUrole{p}{:} \DUrole{n}{float} \DUrole{o}{=} \DUrole{default_value}{3.0}}, \emph{\DUrole{n}{axis\_limits}\DUrole{p}{:} \DUrole{n}{Optional\DUrole{p}{{[}}Union\DUrole{p}{{[}}float\DUrole{p}{, }List\DUrole{p}{{[}}float\DUrole{p}{{]}}\DUrole{p}{{]}}\DUrole{p}{{]}}} \DUrole{o}{=} \DUrole{default_value}{None}}, \emph{\DUrole{n}{unit\_min}\DUrole{p}{:} \DUrole{n}{bool} \DUrole{o}{=} \DUrole{default_value}{True}}, \emph{\DUrole{n}{sig\_fig}\DUrole{p}{:} \DUrole{n}{Optional\DUrole{p}{{[}}int\DUrole{p}{{]}}} \DUrole{o}{=} \DUrole{default_value}{None}}, \emph{\DUrole{n}{fig}\DUrole{p}{:} \DUrole{n}{Optional\DUrole{p}{{[}}matplotlib.pyplot.figure\DUrole{p}{{]}}} \DUrole{o}{=} \DUrole{default_value}{None}}}{{ $\rightarrow$ matplotlib.pyplot.figure}}
\sphinxAtStartPar
Plot the evolution of VCG in 3D space
\begin{quote}\begin{description}
\item[{Parameters}] \leavevmode\begin{itemize}
\item {} 
\sphinxAtStartPar
\sphinxstyleliteralstrong{\sphinxupquote{vcg}} (\sphinxstyleliteralemphasis{\sphinxupquote{pd.DataFrame}}) \textendash{} VCG data

\item {} 
\sphinxAtStartPar
\sphinxstyleliteralstrong{\sphinxupquote{linestyle}} (\sphinxstyleliteralemphasis{\sphinxupquote{str}}\sphinxstyleliteralemphasis{\sphinxupquote{, }}\sphinxstyleliteralemphasis{\sphinxupquote{optional}}) \textendash{} Linestyle to plot data, default=’\sphinxhyphen{}’

\item {} 
\sphinxAtStartPar
\sphinxstyleliteralstrong{\sphinxupquote{colourmap}} (\sphinxstyleliteralemphasis{\sphinxupquote{str}}\sphinxstyleliteralemphasis{\sphinxupquote{, }}\sphinxstyleliteralemphasis{\sphinxupquote{optional}}) \textendash{} Colourmap to use when plotting data, default=’viridis’

\item {} 
\sphinxAtStartPar
\sphinxstyleliteralstrong{\sphinxupquote{linewidth}} (\sphinxstyleliteralemphasis{\sphinxupquote{float}}\sphinxstyleliteralemphasis{\sphinxupquote{, }}\sphinxstyleliteralemphasis{\sphinxupquote{optional}}) \textendash{} Linewidth to use, default=3

\item {} 
\sphinxAtStartPar
\sphinxstyleliteralstrong{\sphinxupquote{axis\_limits}} (\sphinxstyleliteralemphasis{\sphinxupquote{list of float}}\sphinxstyleliteralemphasis{\sphinxupquote{ or }}\sphinxstyleliteralemphasis{\sphinxupquote{float}}\sphinxstyleliteralemphasis{\sphinxupquote{, }}\sphinxstyleliteralemphasis{\sphinxupquote{optional}}) \textendash{} Limits to apply to the axes, default=None

\item {} 
\sphinxAtStartPar
\sphinxstyleliteralstrong{\sphinxupquote{unit\_min}} (\sphinxstyleliteralemphasis{\sphinxupquote{bool}}\sphinxstyleliteralemphasis{\sphinxupquote{, }}\sphinxstyleliteralemphasis{\sphinxupquote{optional}}) \textendash{} Whether to have the axes set to, as a minimum, unit length, default=True

\item {} 
\sphinxAtStartPar
\sphinxstyleliteralstrong{\sphinxupquote{sig\_fig}} (\sphinxstyleliteralemphasis{\sphinxupquote{int}}\sphinxstyleliteralemphasis{\sphinxupquote{, }}\sphinxstyleliteralemphasis{\sphinxupquote{optional}}) \textendash{} Maximum number of decimal places to be used on the axis plots (e.g., if set to 2, 0.12345 will be displayed
as 0.12). Used to avoid floating point errors, default=None (no adaption made)

\item {} 
\sphinxAtStartPar
\sphinxstyleliteralstrong{\sphinxupquote{fig}} (\sphinxstyleliteralemphasis{\sphinxupquote{plt.figure}}\sphinxstyleliteralemphasis{\sphinxupquote{, }}\sphinxstyleliteralemphasis{\sphinxupquote{optional}}) \textendash{} Handle to existing figure (if exists)

\end{itemize}

\item[{Returns}] \leavevmode
\sphinxAtStartPar
\sphinxstylestrong{fig} \textendash{} Figure handle

\item[{Return type}] \leavevmode
\sphinxAtStartPar
plt.figure

\end{description}\end{quote}

\end{fulllineitems}



\subsubsection{signalplot.vcg.plot\_arc3d}
\label{\detokenize{_autosummary/signalplot.vcg.plot_arc3d:signalplot-vcg-plot-arc3d}}\label{\detokenize{_autosummary/signalplot.vcg.plot_arc3d::doc}}\index{plot\_arc3d() (in module signalplot.vcg)@\spxentry{plot\_arc3d()}\spxextra{in module signalplot.vcg}}

\begin{fulllineitems}
\phantomsection\label{\detokenize{_autosummary/signalplot.vcg.plot_arc3d:signalplot.vcg.plot_arc3d}}\pysiglinewithargsret{\sphinxcode{\sphinxupquote{signalplot.vcg.}}\sphinxbfcode{\sphinxupquote{plot\_arc3d}}}{\emph{\DUrole{n}{vector1}\DUrole{p}{:} \DUrole{n}{List\DUrole{p}{{[}}float\DUrole{p}{{]}}}}, \emph{\DUrole{n}{vector2}\DUrole{p}{:} \DUrole{n}{List\DUrole{p}{{[}}float\DUrole{p}{{]}}}}, \emph{\DUrole{n}{radius}\DUrole{p}{:} \DUrole{n}{float} \DUrole{o}{=} \DUrole{default_value}{0.2}}, \emph{\DUrole{n}{fig}\DUrole{p}{:} \DUrole{n}{Optional\DUrole{p}{{[}}matplotlib.pyplot.figure\DUrole{p}{{]}}} \DUrole{o}{=} \DUrole{default_value}{None}}, \emph{\DUrole{n}{colour}\DUrole{p}{:} \DUrole{n}{str} \DUrole{o}{=} \DUrole{default_value}{\textquotesingle{}C0\textquotesingle{}}}}{{ $\rightarrow$ matplotlib.pyplot.figure}}
\sphinxAtStartPar
Plot arc between two given vectors in 3D space.
\begin{quote}\begin{description}
\item[{Parameters}] \leavevmode\begin{itemize}
\item {} 
\sphinxAtStartPar
\sphinxstyleliteralstrong{\sphinxupquote{vector1}} (\sphinxstyleliteralemphasis{\sphinxupquote{list of float}}) \textendash{} First vector

\item {} 
\sphinxAtStartPar
\sphinxstyleliteralstrong{\sphinxupquote{vector2}} (\sphinxstyleliteralemphasis{\sphinxupquote{list of float}}) \textendash{} Second vector

\item {} 
\sphinxAtStartPar
\sphinxstyleliteralstrong{\sphinxupquote{radius}} (\sphinxstyleliteralemphasis{\sphinxupquote{float}}\sphinxstyleliteralemphasis{\sphinxupquote{, }}\sphinxstyleliteralemphasis{\sphinxupquote{optional}}) \textendash{} Radius of arc to plot on figure

\item {} 
\sphinxAtStartPar
\sphinxstyleliteralstrong{\sphinxupquote{fig}} (\sphinxstyleliteralemphasis{\sphinxupquote{plt.figure}}\sphinxstyleliteralemphasis{\sphinxupquote{, }}\sphinxstyleliteralemphasis{\sphinxupquote{optional}}) \textendash{} Handle of figure on which to plot the arc. If not given, will produce new figure

\item {} 
\sphinxAtStartPar
\sphinxstyleliteralstrong{\sphinxupquote{colour}} (\sphinxstyleliteralemphasis{\sphinxupquote{str}}\sphinxstyleliteralemphasis{\sphinxupquote{, }}\sphinxstyleliteralemphasis{\sphinxupquote{optional}}) \textendash{} Colour in which to display the arc

\end{itemize}

\item[{Returns}] \leavevmode
\sphinxAtStartPar
\sphinxstylestrong{fig} \textendash{} Handle for figure on which arc has been plotted

\item[{Return type}] \leavevmode
\sphinxAtStartPar
plt.figure

\end{description}\end{quote}

\end{fulllineitems}



\subsubsection{signalplot.vcg.plot\_density\_effect}
\label{\detokenize{_autosummary/signalplot.vcg.plot_density_effect:signalplot-vcg-plot-density-effect}}\label{\detokenize{_autosummary/signalplot.vcg.plot_density_effect::doc}}\index{plot\_density\_effect() (in module signalplot.vcg)@\spxentry{plot\_density\_effect()}\spxextra{in module signalplot.vcg}}

\begin{fulllineitems}
\phantomsection\label{\detokenize{_autosummary/signalplot.vcg.plot_density_effect:signalplot.vcg.plot_density_effect}}\pysiglinewithargsret{\sphinxcode{\sphinxupquote{signalplot.vcg.}}\sphinxbfcode{\sphinxupquote{plot\_density\_effect}}}{\emph{\DUrole{n}{metrics}\DUrole{p}{:} \DUrole{n}{List\DUrole{p}{{[}}List\DUrole{p}{{[}}float\DUrole{p}{{]}}\DUrole{p}{{]}}}}, \emph{\DUrole{n}{metric\_name}\DUrole{p}{:} \DUrole{n}{str}}, \emph{\DUrole{n}{metric\_labels}\DUrole{p}{:} \DUrole{n}{Optional\DUrole{p}{{[}}List\DUrole{p}{{[}}str\DUrole{p}{{]}}\DUrole{p}{{]}}} \DUrole{o}{=} \DUrole{default_value}{None}}, \emph{\DUrole{n}{density\_labels}\DUrole{p}{:} \DUrole{n}{Optional\DUrole{p}{{[}}List\DUrole{p}{{[}}str\DUrole{p}{{]}}\DUrole{p}{{]}}} \DUrole{o}{=} \DUrole{default_value}{None}}, \emph{\DUrole{n}{linestyles}\DUrole{p}{:} \DUrole{n}{Optional\DUrole{p}{{[}}List\DUrole{p}{{[}}str\DUrole{p}{{]}}\DUrole{p}{{]}}} \DUrole{o}{=} \DUrole{default_value}{None}}, \emph{\DUrole{n}{colours}\DUrole{p}{:} \DUrole{n}{Optional\DUrole{p}{{[}}List\DUrole{p}{{[}}str\DUrole{p}{{]}}\DUrole{p}{{]}}} \DUrole{o}{=} \DUrole{default_value}{None}}, \emph{\DUrole{n}{markers}\DUrole{p}{:} \DUrole{n}{Optional\DUrole{p}{{[}}List\DUrole{p}{{[}}str\DUrole{p}{{]}}\DUrole{p}{{]}}} \DUrole{o}{=} \DUrole{default_value}{None}}}{}
\sphinxAtStartPar
Plot the effect of density on metrics.

\sphinxAtStartPar
TODO: look into decorator for the LaTeX preamble?
\begin{quote}\begin{description}
\item[{Parameters}] \leavevmode\begin{itemize}
\item {} 
\sphinxAtStartPar
\sphinxstyleliteralstrong{\sphinxupquote{metrics}} (\sphinxstyleliteralemphasis{\sphinxupquote{list of list of float}}) \textendash{} Effects of scar density on given metrics, presented as e.g. {[}metric\_LV, metric\_septum{]}

\item {} 
\sphinxAtStartPar
\sphinxstyleliteralstrong{\sphinxupquote{metric\_name}} (\sphinxstyleliteralemphasis{\sphinxupquote{str}}) \textendash{} Name of metric being assessed

\item {} 
\sphinxAtStartPar
\sphinxstyleliteralstrong{\sphinxupquote{metric\_labels}} (\sphinxstyleliteralemphasis{\sphinxupquote{list of str}}\sphinxstyleliteralemphasis{\sphinxupquote{, }}\sphinxstyleliteralemphasis{\sphinxupquote{optional}}) \textendash{} Labels for the metrics being plotted, default={[}‘LV’, ‘Septum’{]}

\item {} 
\sphinxAtStartPar
\sphinxstyleliteralstrong{\sphinxupquote{density\_labels}} (\sphinxstyleliteralemphasis{\sphinxupquote{list of str}}\sphinxstyleliteralemphasis{\sphinxupquote{, }}\sphinxstyleliteralemphasis{\sphinxupquote{optional}}) \textendash{} Labels for the different scar densities being plotted

\item {} 
\sphinxAtStartPar
\sphinxstyleliteralstrong{\sphinxupquote{linestyles}} (\sphinxstyleliteralemphasis{\sphinxupquote{list of str}}\sphinxstyleliteralemphasis{\sphinxupquote{, }}\sphinxstyleliteralemphasis{\sphinxupquote{optional}}) \textendash{} Linestyles for the density effect plots, default={[}‘\sphinxhyphen{}’ for \_ in range(len(metrics)){]}

\item {} 
\sphinxAtStartPar
\sphinxstyleliteralstrong{\sphinxupquote{colours}} (\sphinxstyleliteralemphasis{\sphinxupquote{list of str}}\sphinxstyleliteralemphasis{\sphinxupquote{, }}\sphinxstyleliteralemphasis{\sphinxupquote{optional}}) \textendash{} Colours to use for the plot, default=common\_analysis.get\_plot\_colours(len(metrics))

\item {} 
\sphinxAtStartPar
\sphinxstyleliteralstrong{\sphinxupquote{markers}} (\sphinxstyleliteralemphasis{\sphinxupquote{list of str}}\sphinxstyleliteralemphasis{\sphinxupquote{, }}\sphinxstyleliteralemphasis{\sphinxupquote{optional}}) \textendash{} Markers to use for the discrete data points in the plot, default={[}‘o’ for \_ in range(len(metrics)){]}

\end{itemize}

\end{description}\end{quote}

\end{fulllineitems}



\subsubsection{signalplot.vcg.plot\_metric\_change}
\label{\detokenize{_autosummary/signalplot.vcg.plot_metric_change:signalplot-vcg-plot-metric-change}}\label{\detokenize{_autosummary/signalplot.vcg.plot_metric_change::doc}}\index{plot\_metric\_change() (in module signalplot.vcg)@\spxentry{plot\_metric\_change()}\spxextra{in module signalplot.vcg}}

\begin{fulllineitems}
\phantomsection\label{\detokenize{_autosummary/signalplot.vcg.plot_metric_change:signalplot.vcg.plot_metric_change}}\pysiglinewithargsret{\sphinxcode{\sphinxupquote{signalplot.vcg.}}\sphinxbfcode{\sphinxupquote{plot\_metric\_change}}}{\emph{\DUrole{n}{metrics}\DUrole{p}{:} \DUrole{n}{List\DUrole{p}{{[}}List\DUrole{p}{{[}}List\DUrole{p}{{[}}float\DUrole{p}{{]}}\DUrole{p}{{]}}\DUrole{p}{{]}}}}, \emph{\DUrole{n}{metrics\_phi}\DUrole{p}{:} \DUrole{n}{List\DUrole{p}{{[}}List\DUrole{p}{{[}}List\DUrole{p}{{[}}float\DUrole{p}{{]}}\DUrole{p}{{]}}\DUrole{p}{{]}}}}, \emph{\DUrole{n}{metrics\_rho}\DUrole{p}{:} \DUrole{n}{List\DUrole{p}{{[}}List\DUrole{p}{{[}}List\DUrole{p}{{[}}float\DUrole{p}{{]}}\DUrole{p}{{]}}\DUrole{p}{{]}}}}, \emph{\DUrole{n}{metrics\_z}\DUrole{p}{:} \DUrole{n}{List\DUrole{p}{{[}}List\DUrole{p}{{[}}List\DUrole{p}{{[}}float\DUrole{p}{{]}}\DUrole{p}{{]}}\DUrole{p}{{]}}}}, \emph{\DUrole{n}{metric\_name}\DUrole{p}{:} \DUrole{n}{str}}, \emph{\DUrole{n}{metrics\_lv}\DUrole{p}{:} \DUrole{n}{Optional\DUrole{p}{{[}}List\DUrole{p}{{[}}bool\DUrole{p}{{]}}\DUrole{p}{{]}}} \DUrole{o}{=} \DUrole{default_value}{None}}, \emph{\DUrole{n}{labels}\DUrole{p}{:} \DUrole{n}{Optional\DUrole{p}{{[}}List\DUrole{p}{{[}}str\DUrole{p}{{]}}\DUrole{p}{{]}}} \DUrole{o}{=} \DUrole{default_value}{None}}, \emph{\DUrole{n}{scattermarkers}\DUrole{p}{:} \DUrole{n}{Optional\DUrole{p}{{[}}List\DUrole{p}{{[}}str\DUrole{p}{{]}}\DUrole{p}{{]}}} \DUrole{o}{=} \DUrole{default_value}{None}}, \emph{\DUrole{n}{linemarkers}\DUrole{p}{:} \DUrole{n}{Optional\DUrole{p}{{[}}List\DUrole{p}{{[}}str\DUrole{p}{{]}}\DUrole{p}{{]}}} \DUrole{o}{=} \DUrole{default_value}{None}}, \emph{\DUrole{n}{colours}\DUrole{p}{:} \DUrole{n}{Optional\DUrole{p}{{[}}List\DUrole{p}{{[}}str\DUrole{p}{{]}}\DUrole{p}{{]}}} \DUrole{o}{=} \DUrole{default_value}{None}}, \emph{\DUrole{n}{linestyles}\DUrole{p}{:} \DUrole{n}{Optional\DUrole{p}{{[}}List\DUrole{p}{{[}}str\DUrole{p}{{]}}\DUrole{p}{{]}}} \DUrole{o}{=} \DUrole{default_value}{None}}, \emph{\DUrole{n}{layout}\DUrole{p}{:} \DUrole{n}{Optional\DUrole{p}{{[}}str\DUrole{p}{{]}}} \DUrole{o}{=} \DUrole{default_value}{None}}, \emph{\DUrole{n}{axis\_match}\DUrole{p}{:} \DUrole{n}{bool} \DUrole{o}{=} \DUrole{default_value}{True}}, \emph{\DUrole{n}{no\_labels}\DUrole{p}{:} \DUrole{n}{bool} \DUrole{o}{=} \DUrole{default_value}{False}}}{{ $\rightarrow$ Tuple}}
\sphinxAtStartPar
Function to plot all the various figures for trend analysis in one go.

\sphinxAtStartPar
TODO: labels parameter seems redundant \sphinxhyphen{} potentially remove
\begin{quote}\begin{description}
\item[{Parameters}] \leavevmode\begin{itemize}
\item {} 
\sphinxAtStartPar
\sphinxstyleliteralstrong{\sphinxupquote{metrics}} (\sphinxstyleliteralemphasis{\sphinxupquote{list of list of list of float}}) \textendash{} Complete list of all metric data recorded
{[}phi+rho+z+size+other{]}

\item {} 
\sphinxAtStartPar
\sphinxstyleliteralstrong{\sphinxupquote{metrics\_phi}} (\sphinxstyleliteralemphasis{\sphinxupquote{list of list of list of float}}) \textendash{} Metric data recorded for scar size variations in phi UVC

\item {} 
\sphinxAtStartPar
\sphinxstyleliteralstrong{\sphinxupquote{metrics\_rho}} (\sphinxstyleliteralemphasis{\sphinxupquote{list of list of list of float}}) \textendash{} Metric data recorded for scar size variations in rho UVC

\item {} 
\sphinxAtStartPar
\sphinxstyleliteralstrong{\sphinxupquote{metrics\_z}} (\sphinxstyleliteralemphasis{\sphinxupquote{list of list of list of float}}) \textendash{} Metric data recorded for scar size variations in z UVC

\item {} 
\sphinxAtStartPar
\sphinxstyleliteralstrong{\sphinxupquote{metric\_name}} (\sphinxstyleliteralemphasis{\sphinxupquote{str}}) \textendash{} Name of metric being plotted (for labelling purposes). Can incorporate LaTeX typesetting.

\item {} 
\sphinxAtStartPar
\sphinxstyleliteralstrong{\sphinxupquote{metrics\_lv}} (\sphinxstyleliteralemphasis{\sphinxupquote{list of bool}}\sphinxstyleliteralemphasis{\sphinxupquote{, }}\sphinxstyleliteralemphasis{\sphinxupquote{optional}}) \textendash{} Boolean to distinguish whether metrics being plotted are for LV or septal data, default={[}True, False{]}

\item {} 
\sphinxAtStartPar
\sphinxstyleliteralstrong{\sphinxupquote{labels}} (\sphinxstyleliteralemphasis{\sphinxupquote{list of str}}\sphinxstyleliteralemphasis{\sphinxupquote{, }}\sphinxstyleliteralemphasis{\sphinxupquote{optional}}) \textendash{} Labels for the data sets being plotted, default={[}‘LV’, ‘Septum’{]}

\item {} 
\sphinxAtStartPar
\sphinxstyleliteralstrong{\sphinxupquote{scattermarkers}} (\sphinxstyleliteralemphasis{\sphinxupquote{list of str}}\sphinxstyleliteralemphasis{\sphinxupquote{, }}\sphinxstyleliteralemphasis{\sphinxupquote{optional}}) \textendash{} Markers to use to plot the data on the scatterplots, default={[}‘+’, ‘o’, ‘D’, ‘v’, ‘\textasciicircum{}’, ‘s’, ‘*’, ‘x’{]}

\item {} 
\sphinxAtStartPar
\sphinxstyleliteralstrong{\sphinxupquote{linemarkers}} (\sphinxstyleliteralemphasis{\sphinxupquote{list of str}}\sphinxstyleliteralemphasis{\sphinxupquote{, }}\sphinxstyleliteralemphasis{\sphinxupquote{optional}}) \textendash{} Markers to use on the line plots to indicate discrete data points, required to be at least as long as
the longest line plot to be drawn (rho), default={[}‘.’ for \_ in range(len(metrics\_rho)){]}

\item {} 
\sphinxAtStartPar
\sphinxstyleliteralstrong{\sphinxupquote{colours}} (\sphinxstyleliteralemphasis{\sphinxupquote{list of str}}\sphinxstyleliteralemphasis{\sphinxupquote{, }}\sphinxstyleliteralemphasis{\sphinxupquote{optional}}) \textendash{} Sequence of colours to plot data (if plotting LV and septal data, will require two different colours to allow
them to be distinguished), default=common\_analysis.get\_plot\_colours(len(metrics\_rho))

\item {} 
\sphinxAtStartPar
\sphinxstyleliteralstrong{\sphinxupquote{linestyles}} (\sphinxstyleliteralemphasis{\sphinxupquote{list of str}}\sphinxstyleliteralemphasis{\sphinxupquote{, }}\sphinxstyleliteralemphasis{\sphinxupquote{optional}}) \textendash{} Linestyles to be used for plotting the data on lineplots, default={[}‘\sphinxhyphen{}’ for \_ in range(len(metrics\_rho)){]}

\item {} 
\sphinxAtStartPar
\sphinxstyleliteralstrong{\sphinxupquote{layout}} (\sphinxstyleliteralemphasis{\sphinxupquote{\{\textquotesingle{}combined\textquotesingle{}}}\sphinxstyleliteralemphasis{\sphinxupquote{, }}\sphinxstyleliteralemphasis{\sphinxupquote{\textquotesingle{}figures\textquotesingle{}\}}}\sphinxstyleliteralemphasis{\sphinxupquote{, }}\sphinxstyleliteralemphasis{\sphinxupquote{optional}}) \textendash{} String specifying the output, whether all plots should be combined into one figure window (default), or whether
individual figure windows should be plotted for each plot

\item {} 
\sphinxAtStartPar
\sphinxstyleliteralstrong{\sphinxupquote{axis\_match}} (\sphinxstyleliteralemphasis{\sphinxupquote{bool}}\sphinxstyleliteralemphasis{\sphinxupquote{, }}\sphinxstyleliteralemphasis{\sphinxupquote{optional}}) \textendash{} Whether to make sure all plotted figures share the same axis ranges, default=True

\item {} 
\sphinxAtStartPar
\sphinxstyleliteralstrong{\sphinxupquote{no\_labels}} (\sphinxstyleliteralemphasis{\sphinxupquote{bool}}\sphinxstyleliteralemphasis{\sphinxupquote{, }}\sphinxstyleliteralemphasis{\sphinxupquote{optional}}) \textendash{} Whether to have labels on the figures, or not \sphinxhyphen{} having no labels can make it far easier to ‘prettify’ the
figures manually later in Inkscape, default=False

\end{itemize}

\item[{Returns}] \leavevmode
\sphinxAtStartPar
\begin{itemize}
\item {} 
\sphinxAtStartPar
\sphinxstylestrong{fig} (\sphinxstyleemphasis{plt.figure or dict of plt.figure}) \textendash{} Handle to figure(s)

\item {} 
\sphinxAtStartPar
\sphinxstylestrong{ax} (\sphinxstyleemphasis{dict}) \textendash{} Handles to axes

\end{itemize}


\end{description}\end{quote}

\end{fulllineitems}



\subsubsection{signalplot.vcg.plot\_metric\_change\_barplot}
\label{\detokenize{_autosummary/signalplot.vcg.plot_metric_change_barplot:signalplot-vcg-plot-metric-change-barplot}}\label{\detokenize{_autosummary/signalplot.vcg.plot_metric_change_barplot::doc}}\index{plot\_metric\_change\_barplot() (in module signalplot.vcg)@\spxentry{plot\_metric\_change\_barplot()}\spxextra{in module signalplot.vcg}}

\begin{fulllineitems}
\phantomsection\label{\detokenize{_autosummary/signalplot.vcg.plot_metric_change_barplot:signalplot.vcg.plot_metric_change_barplot}}\pysiglinewithargsret{\sphinxcode{\sphinxupquote{signalplot.vcg.}}\sphinxbfcode{\sphinxupquote{plot\_metric\_change\_barplot}}}{\emph{\DUrole{n}{metrics\_cont}\DUrole{p}{:} \DUrole{n}{List\DUrole{p}{{[}}List\DUrole{p}{{[}}float\DUrole{p}{{]}}\DUrole{p}{{]}}}}, \emph{\DUrole{n}{metrics\_lv}\DUrole{p}{:} \DUrole{n}{List\DUrole{p}{{[}}List\DUrole{p}{{[}}float\DUrole{p}{{]}}\DUrole{p}{{]}}}}, \emph{\DUrole{n}{metrics\_sept}\DUrole{p}{:} \DUrole{n}{List\DUrole{p}{{[}}List\DUrole{p}{{[}}float\DUrole{p}{{]}}\DUrole{p}{{]}}}}, \emph{\DUrole{n}{metric\_labels}\DUrole{p}{:} \DUrole{n}{List\DUrole{p}{{[}}str\DUrole{p}{{]}}}}, \emph{\DUrole{n}{layout}\DUrole{p}{:} \DUrole{n}{Optional\DUrole{p}{{[}}str\DUrole{p}{{]}}} \DUrole{o}{=} \DUrole{default_value}{None}}}{{ $\rightarrow$ Tuple}}
\sphinxAtStartPar
Plots a bar chart for the observed metrics.
\begin{quote}\begin{description}
\item[{Parameters}] \leavevmode\begin{itemize}
\item {} 
\sphinxAtStartPar
\sphinxstyleliteralstrong{\sphinxupquote{metrics\_cont}} (\sphinxstyleliteralemphasis{\sphinxupquote{list of list of float}}) \textendash{} Values of series of metrics for no scar

\item {} 
\sphinxAtStartPar
\sphinxstyleliteralstrong{\sphinxupquote{metrics\_lv}} (\sphinxstyleliteralemphasis{\sphinxupquote{list of list of float}}) \textendash{} Values of series of metrics for LV scar

\item {} 
\sphinxAtStartPar
\sphinxstyleliteralstrong{\sphinxupquote{metrics\_sept}} (\sphinxstyleliteralemphasis{\sphinxupquote{list of list of float}}) \textendash{} Values of series of metrics for septal scar

\item {} 
\sphinxAtStartPar
\sphinxstyleliteralstrong{\sphinxupquote{metric\_labels}} (\sphinxstyleliteralemphasis{\sphinxupquote{list of str}}) \textendash{} Names of metrics being plotted

\item {} 
\sphinxAtStartPar
\sphinxstyleliteralstrong{\sphinxupquote{layout}} (\sphinxstyleliteralemphasis{\sphinxupquote{\{\textquotesingle{}combined\textquotesingle{}}}\sphinxstyleliteralemphasis{\sphinxupquote{, }}\sphinxstyleliteralemphasis{\sphinxupquote{\textquotesingle{}fig\textquotesingle{}\}}}\sphinxstyleliteralemphasis{\sphinxupquote{, }}\sphinxstyleliteralemphasis{\sphinxupquote{optional}}) \textendash{} Whether to plot bar charts on combined plot window, or in individual figure windows

\end{itemize}

\item[{Returns}] \leavevmode
\sphinxAtStartPar
\begin{itemize}
\item {} 
\sphinxAtStartPar
\sphinxstylestrong{fig} (\sphinxstyleemphasis{plt.figure or list of plt.figure}) \textendash{} Handle(s) to figures

\item {} 
\sphinxAtStartPar
\sphinxstylestrong{ax} (\sphinxstyleemphasis{list}) \textendash{} Handles to axes

\end{itemize}


\end{description}\end{quote}

\end{fulllineitems}



\subsubsection{signalplot.vcg.plot\_spatial\_velocity}
\label{\detokenize{_autosummary/signalplot.vcg.plot_spatial_velocity:signalplot-vcg-plot-spatial-velocity}}\label{\detokenize{_autosummary/signalplot.vcg.plot_spatial_velocity::doc}}\index{plot\_spatial\_velocity() (in module signalplot.vcg)@\spxentry{plot\_spatial\_velocity()}\spxextra{in module signalplot.vcg}}

\begin{fulllineitems}
\phantomsection\label{\detokenize{_autosummary/signalplot.vcg.plot_spatial_velocity:signalplot.vcg.plot_spatial_velocity}}\pysiglinewithargsret{\sphinxcode{\sphinxupquote{signalplot.vcg.}}\sphinxbfcode{\sphinxupquote{plot\_spatial\_velocity}}}{\emph{\DUrole{n}{vcg}\DUrole{p}{:} \DUrole{n}{Union\DUrole{p}{{[}}pandas.core.frame.DataFrame\DUrole{p}{, }List\DUrole{p}{{[}}pandas.core.frame.DataFrame\DUrole{p}{{]}}\DUrole{p}{{]}}}}, \emph{\DUrole{n}{sv}\DUrole{p}{:} \DUrole{n}{Optional\DUrole{p}{{[}}List\DUrole{p}{{[}}List\DUrole{p}{{[}}float\DUrole{p}{{]}}\DUrole{p}{{]}}\DUrole{p}{{]}}} \DUrole{o}{=} \DUrole{default_value}{None}}, \emph{\DUrole{n}{limits}\DUrole{p}{:} \DUrole{n}{Optional\DUrole{p}{{[}}List\DUrole{p}{{[}}List\DUrole{p}{{[}}float\DUrole{p}{{]}}\DUrole{p}{{]}}\DUrole{p}{{]}}} \DUrole{o}{=} \DUrole{default_value}{None}}, \emph{\DUrole{n}{fig}\DUrole{p}{:} \DUrole{n}{Optional\DUrole{p}{{[}}matplotlib.pyplot.figure\DUrole{p}{{]}}} \DUrole{o}{=} \DUrole{default_value}{None}}, \emph{\DUrole{n}{legend\_vcg}\DUrole{p}{:} \DUrole{n}{Optional\DUrole{p}{{[}}Union\DUrole{p}{{[}}List\DUrole{p}{{[}}str\DUrole{p}{{]}}\DUrole{p}{, }str\DUrole{p}{{]}}\DUrole{p}{{]}}} \DUrole{o}{=} \DUrole{default_value}{None}}, \emph{\DUrole{n}{legend\_limits}\DUrole{p}{:} \DUrole{n}{Optional\DUrole{p}{{[}}Union\DUrole{p}{{[}}List\DUrole{p}{{[}}str\DUrole{p}{{]}}\DUrole{p}{, }str\DUrole{p}{{]}}\DUrole{p}{{]}}} \DUrole{o}{=} \DUrole{default_value}{None}}, \emph{\DUrole{n}{limits\_linestyles}\DUrole{p}{:} \DUrole{n}{Optional\DUrole{p}{{[}}List\DUrole{p}{{[}}str\DUrole{p}{{]}}\DUrole{p}{{]}}} \DUrole{o}{=} \DUrole{default_value}{None}}, \emph{\DUrole{n}{limits\_colours}\DUrole{p}{:} \DUrole{n}{Optional\DUrole{p}{{[}}List\DUrole{p}{{[}}str\DUrole{p}{{]}}\DUrole{p}{{]}}} \DUrole{o}{=} \DUrole{default_value}{None}}, \emph{\DUrole{n}{filter\_sv}\DUrole{p}{:} \DUrole{n}{bool} \DUrole{o}{=} \DUrole{default_value}{True}}}{{ $\rightarrow$ Tuple}}
\sphinxAtStartPar
Plot the spatial velocity for given VCG data

\sphinxAtStartPar
Plot the spatial velocity and VCG elements, with limits (e.g. QRS limits) if provided. Note that if spatial
velocity is not provided, default values will be used to calculate it \sphinxhyphen{} if anything else is desired, then spatial
velocity must be calculated first and provided to the function.
\begin{quote}\begin{description}
\item[{Parameters}] \leavevmode\begin{itemize}
\item {} 
\sphinxAtStartPar
\sphinxstyleliteralstrong{\sphinxupquote{vcg}} (\sphinxstyleliteralemphasis{\sphinxupquote{pd.DataFrame}}\sphinxstyleliteralemphasis{\sphinxupquote{ or }}\sphinxstyleliteralemphasis{\sphinxupquote{list of pd.DataFrame}}) \textendash{} VCG data

\item {} 
\sphinxAtStartPar
\sphinxstyleliteralstrong{\sphinxupquote{sv}} (\sphinxstyleliteralemphasis{\sphinxupquote{list of list of float}}\sphinxstyleliteralemphasis{\sphinxupquote{, }}\sphinxstyleliteralemphasis{\sphinxupquote{optional}}) \textendash{} Spatial velocity data. Only required to be given here if special parameters wish to be given, otherwise it
will be calculated using default parameters (default)

\item {} 
\sphinxAtStartPar
\sphinxstyleliteralstrong{\sphinxupquote{limits}} (\sphinxstyleliteralemphasis{\sphinxupquote{list of list of float}}\sphinxstyleliteralemphasis{\sphinxupquote{, }}\sphinxstyleliteralemphasis{\sphinxupquote{optional}}) \textendash{} 
\sphinxAtStartPar
A series of ‘limits’ to be plotted on the figure with the VCG and spatial plot. Presented as a list of the
same length of the VCG data, with the required limits within:
\begin{quote}

\sphinxAtStartPar
e.g. {[}{[}QRS\_start1, QRS\_start2, …{]}, {[}QRS\_end1, QRS\_end2, …{]}, …{]}
\end{quote}

\sphinxAtStartPar
Default=None


\item {} 
\sphinxAtStartPar
\sphinxstyleliteralstrong{\sphinxupquote{fig}} (\sphinxstyleliteralemphasis{\sphinxupquote{plt.figure}}\sphinxstyleliteralemphasis{\sphinxupquote{, }}\sphinxstyleliteralemphasis{\sphinxupquote{optional}}) \textendash{} Handle to existing figure, if data is wished to be plotted on existing plot, default=None

\item {} 
\sphinxAtStartPar
\sphinxstyleliteralstrong{\sphinxupquote{legend\_vcg}} (\sphinxstyleliteralemphasis{\sphinxupquote{str}}\sphinxstyleliteralemphasis{\sphinxupquote{ or }}\sphinxstyleliteralemphasis{\sphinxupquote{list of str}}\sphinxstyleliteralemphasis{\sphinxupquote{, }}\sphinxstyleliteralemphasis{\sphinxupquote{optional}}) \textendash{} Labels to apply to the VCG/SV data, default=None

\item {} 
\sphinxAtStartPar
\sphinxstyleliteralstrong{\sphinxupquote{legend\_limits}} (\sphinxstyleliteralemphasis{\sphinxupquote{str}}\sphinxstyleliteralemphasis{\sphinxupquote{ or }}\sphinxstyleliteralemphasis{\sphinxupquote{list of str}}\sphinxstyleliteralemphasis{\sphinxupquote{, }}\sphinxstyleliteralemphasis{\sphinxupquote{optional}}) \textendash{} Labels to apply to the limits, default=None

\item {} 
\sphinxAtStartPar
\sphinxstyleliteralstrong{\sphinxupquote{limits\_linestyles}} (\sphinxstyleliteralemphasis{\sphinxupquote{list of str}}\sphinxstyleliteralemphasis{\sphinxupquote{, }}\sphinxstyleliteralemphasis{\sphinxupquote{optional}}) \textendash{} Linestyles to apply to the different limits being supplied, default=None (will use varying linestyles based
on tools.plotting.get\_plot\_lines)

\item {} 
\sphinxAtStartPar
\sphinxstyleliteralstrong{\sphinxupquote{limits\_colours}} (\sphinxstyleliteralemphasis{\sphinxupquote{list of str}}\sphinxstyleliteralemphasis{\sphinxupquote{, }}\sphinxstyleliteralemphasis{\sphinxupquote{optional}}) \textendash{} Colours to apply to the different limits being supplied, default=None (will use varying colours based on
tools.plotting.get\_plot\_colours)

\item {} 
\sphinxAtStartPar
\sphinxstyleliteralstrong{\sphinxupquote{filter\_sv}} (\sphinxstyleliteralemphasis{\sphinxupquote{bool}}\sphinxstyleliteralemphasis{\sphinxupquote{, }}\sphinxstyleliteralemphasis{\sphinxupquote{optional}}) \textendash{} Whether or not to apply filtering to spatial velocity prior to finding the start/end points for the
threshold, default=True

\end{itemize}

\item[{Returns}] \leavevmode
\sphinxAtStartPar
Handles to the figure and axes generated

\item[{Return type}] \leavevmode
\sphinxAtStartPar
fig, ax

\end{description}\end{quote}

\end{fulllineitems}



\subsubsection{signalplot.vcg.plot\_xyz\_components}
\label{\detokenize{_autosummary/signalplot.vcg.plot_xyz_components:signalplot-vcg-plot-xyz-components}}\label{\detokenize{_autosummary/signalplot.vcg.plot_xyz_components::doc}}\index{plot\_xyz\_components() (in module signalplot.vcg)@\spxentry{plot\_xyz\_components()}\spxextra{in module signalplot.vcg}}

\begin{fulllineitems}
\phantomsection\label{\detokenize{_autosummary/signalplot.vcg.plot_xyz_components:signalplot.vcg.plot_xyz_components}}\pysiglinewithargsret{\sphinxcode{\sphinxupquote{signalplot.vcg.}}\sphinxbfcode{\sphinxupquote{plot\_xyz\_components}}}{\emph{\DUrole{n}{vcgs}\DUrole{p}{:} \DUrole{n}{Union\DUrole{p}{{[}}pandas.core.frame.DataFrame\DUrole{p}{, }List\DUrole{p}{{[}}pandas.core.frame.DataFrame\DUrole{p}{{]}}\DUrole{p}{{]}}}}, \emph{\DUrole{n}{legend}\DUrole{p}{:} \DUrole{n}{Optional\DUrole{p}{{[}}List\DUrole{p}{{[}}str\DUrole{p}{{]}}\DUrole{p}{{]}}} \DUrole{o}{=} \DUrole{default_value}{None}}, \emph{\DUrole{n}{colours}\DUrole{p}{:} \DUrole{n}{Optional\DUrole{p}{{[}}List\DUrole{p}{{[}}List\DUrole{p}{{[}}float\DUrole{p}{{]}}\DUrole{p}{{]}}\DUrole{p}{{]}}} \DUrole{o}{=} \DUrole{default_value}{None}}, \emph{\DUrole{n}{linestyles}\DUrole{p}{:} \DUrole{n}{Optional\DUrole{p}{{[}}List\DUrole{p}{{[}}str\DUrole{p}{{]}}\DUrole{p}{{]}}} \DUrole{o}{=} \DUrole{default_value}{None}}, \emph{\DUrole{n}{legend\_location}\DUrole{p}{:} \DUrole{n}{Optional\DUrole{p}{{[}}str\DUrole{p}{{]}}} \DUrole{o}{=} \DUrole{default_value}{None}}, \emph{\DUrole{n}{limits}\DUrole{p}{:} \DUrole{n}{Optional\DUrole{p}{{[}}List\DUrole{p}{{[}}List\DUrole{p}{{[}}float\DUrole{p}{{]}}\DUrole{p}{{]}}\DUrole{p}{{]}}} \DUrole{o}{=} \DUrole{default_value}{None}}, \emph{\DUrole{n}{limits\_legend}\DUrole{p}{:} \DUrole{n}{Optional\DUrole{p}{{[}}List\DUrole{p}{{[}}str\DUrole{p}{{]}}\DUrole{p}{{]}}} \DUrole{o}{=} \DUrole{default_value}{None}}, \emph{\DUrole{n}{limits\_colours}\DUrole{p}{:} \DUrole{n}{Optional\DUrole{p}{{[}}List\DUrole{p}{{[}}List\DUrole{p}{{[}}float\DUrole{p}{{]}}\DUrole{p}{{]}}\DUrole{p}{{]}}} \DUrole{o}{=} \DUrole{default_value}{None}}, \emph{\DUrole{n}{limits\_linestyles}\DUrole{p}{:} \DUrole{n}{Optional\DUrole{p}{{[}}List\DUrole{p}{{[}}str\DUrole{p}{{]}}\DUrole{p}{{]}}} \DUrole{o}{=} \DUrole{default_value}{None}}, \emph{\DUrole{n}{limits\_legend\_location}\DUrole{p}{:} \DUrole{n}{str} \DUrole{o}{=} \DUrole{default_value}{\textquotesingle{}lower right\textquotesingle{}}}, \emph{\DUrole{n}{layout}\DUrole{p}{:} \DUrole{n}{str} \DUrole{o}{=} \DUrole{default_value}{\textquotesingle{}grid\textquotesingle{}}}}{{ $\rightarrow$ tuple}}
\sphinxAtStartPar
Plot x, y, z components of VCG data

\sphinxAtStartPar
Multiple options given for layout of resulting plot
\begin{quote}\begin{description}
\item[{Parameters}] \leavevmode\begin{itemize}
\item {} 
\sphinxAtStartPar
\sphinxstyleliteralstrong{\sphinxupquote{vcgs}} (\sphinxstyleliteralemphasis{\sphinxupquote{list of pd.DataFrame}}\sphinxstyleliteralemphasis{\sphinxupquote{ or }}\sphinxstyleliteralemphasis{\sphinxupquote{pd.DataFrame}}) \textendash{} List of vcg data: {[}vcg\_data1, vcg\_data2, …{]}

\item {} 
\sphinxAtStartPar
\sphinxstyleliteralstrong{\sphinxupquote{legend}} (\sphinxstyleliteralemphasis{\sphinxupquote{list of str}}\sphinxstyleliteralemphasis{\sphinxupquote{, }}\sphinxstyleliteralemphasis{\sphinxupquote{optional}}) \textendash{} Legend names for each VCG trace, default=None

\item {} 
\sphinxAtStartPar
\sphinxstyleliteralstrong{\sphinxupquote{colours}} (\sphinxstyleliteralemphasis{\sphinxupquote{list of list of float}}\sphinxstyleliteralemphasis{\sphinxupquote{ or }}\sphinxstyleliteralemphasis{\sphinxupquote{list of str}}\sphinxstyleliteralemphasis{\sphinxupquote{, }}\sphinxstyleliteralemphasis{\sphinxupquote{optional}}) \textendash{} Colours to use for plotting, default=common\_analysis.get\_plot\_colours

\item {} 
\sphinxAtStartPar
\sphinxstyleliteralstrong{\sphinxupquote{linestyles}} (\sphinxstyleliteralemphasis{\sphinxupquote{list of str}}\sphinxstyleliteralemphasis{\sphinxupquote{, }}\sphinxstyleliteralemphasis{\sphinxupquote{optional}}) \textendash{} Linestyles to use for plotting, default=’\sphinxhyphen{}’

\item {} 
\sphinxAtStartPar
\sphinxstyleliteralstrong{\sphinxupquote{legend\_location}} (\sphinxstyleliteralemphasis{\sphinxupquote{str}}\sphinxstyleliteralemphasis{\sphinxupquote{, }}\sphinxstyleliteralemphasis{\sphinxupquote{optional}}) \textendash{} Location to plot the legend. Default=None, which will translate to ‘best’ if no legend is required for
limits, or ‘upper right’ if legend is needed for limits

\item {} 
\sphinxAtStartPar
\sphinxstyleliteralstrong{\sphinxupquote{limits}} (\sphinxstyleliteralemphasis{\sphinxupquote{list of list of float}}\sphinxstyleliteralemphasis{\sphinxupquote{, }}\sphinxstyleliteralemphasis{\sphinxupquote{optional}}) \textendash{} QRS limits to plot on axes, default=None
To be presented in form {[}{[}qrs\_start1, qrs\_starts, …{]}, {[}qrs\_end1, qrs\_end2, …{]}, …{]}

\item {} 
\sphinxAtStartPar
\sphinxstyleliteralstrong{\sphinxupquote{limits\_legend}} (\sphinxstyleliteralemphasis{\sphinxupquote{list of str}}\sphinxstyleliteralemphasis{\sphinxupquote{, }}\sphinxstyleliteralemphasis{\sphinxupquote{optional}}) \textendash{} Legend to apply to the limits plotted, default=None

\item {} 
\sphinxAtStartPar
\sphinxstyleliteralstrong{\sphinxupquote{limits\_colours}} (\sphinxstyleliteralemphasis{\sphinxupquote{list of list of float}}\sphinxstyleliteralemphasis{\sphinxupquote{ or }}\sphinxstyleliteralemphasis{\sphinxupquote{list of str}}\sphinxstyleliteralemphasis{\sphinxupquote{, }}\sphinxstyleliteralemphasis{\sphinxupquote{optional}}) \textendash{} Colours to use when plotting limits, default=common\_analysis.get\_plot\_colours

\item {} 
\sphinxAtStartPar
\sphinxstyleliteralstrong{\sphinxupquote{limits\_linestyles}} (\sphinxstyleliteralemphasis{\sphinxupquote{list of str}}\sphinxstyleliteralemphasis{\sphinxupquote{, }}\sphinxstyleliteralemphasis{\sphinxupquote{optional}}) \textendash{} Linestyles to use when plotting limits, default=’\sphinxhyphen{}’

\item {} 
\sphinxAtStartPar
\sphinxstyleliteralstrong{\sphinxupquote{limits\_legend\_location}} (\sphinxstyleliteralemphasis{\sphinxupquote{str}}\sphinxstyleliteralemphasis{\sphinxupquote{, }}\sphinxstyleliteralemphasis{\sphinxupquote{optional}}) \textendash{} Location to use for the legend containing the limits data

\item {} 
\sphinxAtStartPar
\sphinxstyleliteralstrong{\sphinxupquote{layout}} (\sphinxstyleliteralemphasis{\sphinxupquote{\{\textquotesingle{}grid\textquotesingle{}}}\sphinxstyleliteralemphasis{\sphinxupquote{, }}\sphinxstyleliteralemphasis{\sphinxupquote{\textquotesingle{}figures\textquotesingle{}}}\sphinxstyleliteralemphasis{\sphinxupquote{, }}\sphinxstyleliteralemphasis{\sphinxupquote{\textquotesingle{}combined\textquotesingle{}}}\sphinxstyleliteralemphasis{\sphinxupquote{, }}\sphinxstyleliteralemphasis{\sphinxupquote{\textquotesingle{}row\textquotesingle{}}}\sphinxstyleliteralemphasis{\sphinxupquote{, }}\sphinxstyleliteralemphasis{\sphinxupquote{\textquotesingle{}column\textquotesingle{}}}\sphinxstyleliteralemphasis{\sphinxupquote{, }}\sphinxstyleliteralemphasis{\sphinxupquote{\textquotesingle{}best\textquotesingle{}\}}}\sphinxstyleliteralemphasis{\sphinxupquote{, }}\sphinxstyleliteralemphasis{\sphinxupquote{optional}}) \textendash{} \begin{description}
\item[{Layout of resulting plot}] \leavevmode
\sphinxAtStartPar
grid        x,y,z plots are arranged in a grid (like best, but more rigid grid)
figures     Each x,y,z plot is on a separate figure
combined    x,y,z plots are combined on a single set of axes
row         x,y,z plots are arranged on a horizontal row in one figure
column      x,y,z plots are arranged in a vertical column in one figure
best        x,y,z plots are arranged to try and optimise space (nb: figures not equal sizes…)

\end{description}


\end{itemize}

\item[{Returns}] \leavevmode
\sphinxAtStartPar
Handle for resulting figure(s) and axes

\item[{Return type}] \leavevmode
\sphinxAtStartPar
fig, ax

\end{description}\end{quote}

\end{fulllineitems}



\subsubsection{signalplot.vcg.plot\_xyz\_vector}
\label{\detokenize{_autosummary/signalplot.vcg.plot_xyz_vector:signalplot-vcg-plot-xyz-vector}}\label{\detokenize{_autosummary/signalplot.vcg.plot_xyz_vector::doc}}\index{plot\_xyz\_vector() (in module signalplot.vcg)@\spxentry{plot\_xyz\_vector()}\spxextra{in module signalplot.vcg}}

\begin{fulllineitems}
\phantomsection\label{\detokenize{_autosummary/signalplot.vcg.plot_xyz_vector:signalplot.vcg.plot_xyz_vector}}\pysiglinewithargsret{\sphinxcode{\sphinxupquote{signalplot.vcg.}}\sphinxbfcode{\sphinxupquote{plot\_xyz\_vector}}}{\emph{\DUrole{n}{vector}\DUrole{p}{:} \DUrole{n}{Optional\DUrole{p}{{[}}List\DUrole{p}{{[}}float\DUrole{p}{{]}}\DUrole{p}{{]}}} \DUrole{o}{=} \DUrole{default_value}{None}}, \emph{\DUrole{n}{x}\DUrole{p}{:} \DUrole{n}{Optional\DUrole{p}{{[}}float\DUrole{p}{{]}}} \DUrole{o}{=} \DUrole{default_value}{None}}, \emph{\DUrole{n}{y}\DUrole{p}{:} \DUrole{n}{Optional\DUrole{p}{{[}}float\DUrole{p}{{]}}} \DUrole{o}{=} \DUrole{default_value}{None}}, \emph{\DUrole{n}{z}\DUrole{p}{:} \DUrole{n}{Optional\DUrole{p}{{[}}float\DUrole{p}{{]}}} \DUrole{o}{=} \DUrole{default_value}{None}}, \emph{\DUrole{n}{fig}\DUrole{p}{:} \DUrole{n}{Optional\DUrole{p}{{[}}matplotlib.pyplot.figure\DUrole{p}{{]}}} \DUrole{o}{=} \DUrole{default_value}{None}}, \emph{\DUrole{n}{linecolour}\DUrole{p}{:} \DUrole{n}{str} \DUrole{o}{=} \DUrole{default_value}{\textquotesingle{}C0\textquotesingle{}}}, \emph{\DUrole{n}{linestyle}\DUrole{p}{:} \DUrole{n}{str} \DUrole{o}{=} \DUrole{default_value}{\textquotesingle{}\sphinxhyphen{}\textquotesingle{}}}, \emph{\DUrole{n}{linewidth}\DUrole{p}{:} \DUrole{n}{float} \DUrole{o}{=} \DUrole{default_value}{2}}}{}
\sphinxAtStartPar
Plots a specific vector in 3D space (e.g. to reflect maximum dipole)
\begin{quote}\begin{description}
\item[{Parameters}] \leavevmode\begin{itemize}
\item {} 
\sphinxAtStartPar
\sphinxstyleliteralstrong{\sphinxupquote{vector}} (\sphinxstyleliteralemphasis{\sphinxupquote{list of float}}) \textendash{} {[}x, y, z{]} values of vector to plot, alternatively given as separate x, y, z variables

\item {} 
\sphinxAtStartPar
\sphinxstyleliteralstrong{\sphinxupquote{x}} (\sphinxstyleliteralemphasis{\sphinxupquote{float}}) \textendash{} {[}x, y, z{]} values of vector to plot, alternatively given as vector variable

\item {} 
\sphinxAtStartPar
\sphinxstyleliteralstrong{\sphinxupquote{y}} (\sphinxstyleliteralemphasis{\sphinxupquote{float}}) \textendash{} {[}x, y, z{]} values of vector to plot, alternatively given as vector variable

\item {} 
\sphinxAtStartPar
\sphinxstyleliteralstrong{\sphinxupquote{z}} (\sphinxstyleliteralemphasis{\sphinxupquote{float}}) \textendash{} {[}x, y, z{]} values of vector to plot, alternatively given as vector variable

\item {} 
\sphinxAtStartPar
\sphinxstyleliteralstrong{\sphinxupquote{fig}} (\sphinxstyleliteralemphasis{\sphinxupquote{plt.figure}}\sphinxstyleliteralemphasis{\sphinxupquote{, }}\sphinxstyleliteralemphasis{\sphinxupquote{optional}}) \textendash{} Existing figure handle, if desired to plot the vector onto an extant plot

\item {} 
\sphinxAtStartPar
\sphinxstyleliteralstrong{\sphinxupquote{linecolour}} (\sphinxstyleliteralemphasis{\sphinxupquote{str}}\sphinxstyleliteralemphasis{\sphinxupquote{, }}\sphinxstyleliteralemphasis{\sphinxupquote{optional}}) \textendash{} Colour to plot the vector as

\item {} 
\sphinxAtStartPar
\sphinxstyleliteralstrong{\sphinxupquote{linestyle}} (\sphinxstyleliteralemphasis{\sphinxupquote{str}}\sphinxstyleliteralemphasis{\sphinxupquote{, }}\sphinxstyleliteralemphasis{\sphinxupquote{optional}}) \textendash{} Linestyle to use to plot the body of the arrow

\item {} 
\sphinxAtStartPar
\sphinxstyleliteralstrong{\sphinxupquote{linewidth}} (\sphinxstyleliteralemphasis{\sphinxupquote{float}}\sphinxstyleliteralemphasis{\sphinxupquote{, }}\sphinxstyleliteralemphasis{\sphinxupquote{optional}}) \textendash{} Width to plot the body of the arrow

\end{itemize}

\item[{Returns}] \leavevmode
\sphinxAtStartPar
\sphinxstylestrong{fig} \textendash{} Figure handle

\item[{Return type}] \leavevmode
\sphinxAtStartPar
plt.figure

\item[{Raises}] \leavevmode
\sphinxAtStartPar
\sphinxstyleliteralstrong{\sphinxupquote{ValueError}} \textendash{} Exactly one of vertices and x,y,z must be given

\end{description}\end{quote}
\subsubsection*{Notes}

\sphinxAtStartPar
Must provide either vector or {[}x,y,z{]}

\end{fulllineitems}



\section{tools}
\label{\detokenize{_autosummary/tools:module-tools}}\label{\detokenize{_autosummary/tools:tools}}\label{\detokenize{_autosummary/tools::doc}}\index{module@\spxentry{module}!tools@\spxentry{tools}}\index{tools@\spxentry{tools}!module@\spxentry{module}}

\begin{savenotes}\sphinxatlongtablestart\begin{longtable}[c]{\X{1}{2}\X{1}{2}}
\hline

\endfirsthead

\multicolumn{2}{c}%
{\makebox[0pt]{\sphinxtablecontinued{\tablename\ \thetable{} \textendash{} continued from previous page}}}\\
\hline

\endhead

\hline
\multicolumn{2}{r}{\makebox[0pt][r]{\sphinxtablecontinued{continues on next page}}}\\
\endfoot

\endlastfoot

\sphinxAtStartPar
{\hyperref[\detokenize{_autosummary/tools.maths:module-tools.maths}]{\sphinxcrossref{\sphinxcode{\sphinxupquote{tools.maths}}}}}
&
\sphinxAtStartPar

\\
\hline
\sphinxAtStartPar
{\hyperref[\detokenize{_autosummary/tools.plotting:module-tools.plotting}]{\sphinxcrossref{\sphinxcode{\sphinxupquote{tools.plotting}}}}}
&
\sphinxAtStartPar

\\
\hline
\sphinxAtStartPar
{\hyperref[\detokenize{_autosummary/tools.python:module-tools.python}]{\sphinxcrossref{\sphinxcode{\sphinxupquote{tools.python}}}}}
&
\sphinxAtStartPar

\\
\hline
\end{longtable}\sphinxatlongtableend\end{savenotes}


\subsection{tools.maths}
\label{\detokenize{_autosummary/tools.maths:module-tools.maths}}\label{\detokenize{_autosummary/tools.maths:tools-maths}}\label{\detokenize{_autosummary/tools.maths::doc}}\index{module@\spxentry{module}!tools.maths@\spxentry{tools.maths}}\index{tools.maths@\spxentry{tools.maths}!module@\spxentry{module}}\subsubsection*{Functions}


\begin{savenotes}\sphinxatlongtablestart\begin{longtable}[c]{\X{1}{2}\X{1}{2}}
\hline

\endfirsthead

\multicolumn{2}{c}%
{\makebox[0pt]{\sphinxtablecontinued{\tablename\ \thetable{} \textendash{} continued from previous page}}}\\
\hline

\endhead

\hline
\multicolumn{2}{r}{\makebox[0pt][r]{\sphinxtablecontinued{continues on next page}}}\\
\endfoot

\endlastfoot

\sphinxAtStartPar
{\hyperref[\detokenize{_autosummary/tools.maths.acos2:tools.maths.acos2}]{\sphinxcrossref{\sphinxcode{\sphinxupquote{acos2}}}}}
&
\sphinxAtStartPar
Function to return the inverse cos function across the range (\sphinxhyphen{}pi, pi{]}, rather than (0, pi{]}
\\
\hline
\sphinxAtStartPar
{\hyperref[\detokenize{_autosummary/tools.maths.asin2:tools.maths.asin2}]{\sphinxcrossref{\sphinxcode{\sphinxupquote{asin2}}}}}
&
\sphinxAtStartPar
Function to return the inverse sin function across the range (\sphinxhyphen{}pi, pi{]}, rather than (\sphinxhyphen{}pi/2, pi/2{]}
\\
\hline
\sphinxAtStartPar
{\hyperref[\detokenize{_autosummary/tools.maths.filter_butterworth:tools.maths.filter_butterworth}]{\sphinxcrossref{\sphinxcode{\sphinxupquote{filter\_butterworth}}}}}
&
\sphinxAtStartPar
Filter data using Butterworth filter
\\
\hline
\sphinxAtStartPar
{\hyperref[\detokenize{_autosummary/tools.maths.filter_savitzkygolay:tools.maths.filter_savitzkygolay}]{\sphinxcrossref{\sphinxcode{\sphinxupquote{filter\_savitzkygolay}}}}}
&
\sphinxAtStartPar
Filter EGM data using a Savitzky\sphinxhyphen{}Golay filter
\\
\hline
\sphinxAtStartPar
{\hyperref[\detokenize{_autosummary/tools.maths.get_median:tools.maths.get_median}]{\sphinxcrossref{\sphinxcode{\sphinxupquote{get\_median}}}}}
&
\sphinxAtStartPar
Add the median value of data to a dataframe
\\
\hline
\sphinxAtStartPar
{\hyperref[\detokenize{_autosummary/tools.maths.normalise_signal:tools.maths.normalise_signal}]{\sphinxcrossref{\sphinxcode{\sphinxupquote{normalise\_signal}}}}}
&
\sphinxAtStartPar
Returns a normalised signal, such that the maximum value in the signal is 1, or the minimum is \sphinxhyphen{}1
\\
\hline
\sphinxAtStartPar
{\hyperref[\detokenize{_autosummary/tools.maths.simplex_volume:tools.maths.simplex_volume}]{\sphinxcrossref{\sphinxcode{\sphinxupquote{simplex\_volume}}}}}
&
\sphinxAtStartPar
Return the volume of the simplex with given vertices or sides.
\\
\hline
\end{longtable}\sphinxatlongtableend\end{savenotes}


\subsubsection{tools.maths.acos2}
\label{\detokenize{_autosummary/tools.maths.acos2:tools-maths-acos2}}\label{\detokenize{_autosummary/tools.maths.acos2::doc}}\index{acos2() (in module tools.maths)@\spxentry{acos2()}\spxextra{in module tools.maths}}

\begin{fulllineitems}
\phantomsection\label{\detokenize{_autosummary/tools.maths.acos2:tools.maths.acos2}}\pysiglinewithargsret{\sphinxcode{\sphinxupquote{tools.maths.}}\sphinxbfcode{\sphinxupquote{acos2}}}{\emph{\DUrole{n}{x}\DUrole{p}{:} \DUrole{n}{float}}, \emph{\DUrole{n}{y}\DUrole{p}{:} \DUrole{n}{float}}}{{ $\rightarrow$ float}}
\sphinxAtStartPar
Function to return the inverse cos function across the range (\sphinxhyphen{}pi, pi{]}, rather than (0, pi{]}
\begin{quote}\begin{description}
\item[{Parameters}] \leavevmode\begin{itemize}
\item {} 
\sphinxAtStartPar
\sphinxstyleliteralstrong{\sphinxupquote{x}} (\sphinxstyleliteralemphasis{\sphinxupquote{float}}) \textendash{} x coordinate of the point in 2D space

\item {} 
\sphinxAtStartPar
\sphinxstyleliteralstrong{\sphinxupquote{y}} (\sphinxstyleliteralemphasis{\sphinxupquote{float}}) \textendash{} y coordinate of the point in 2D space

\end{itemize}

\item[{Returns}] \leavevmode
\sphinxAtStartPar
\sphinxstylestrong{theta} \textendash{} Angle corresponding to point in 2D space in radial coordinates, within range (\sphinxhyphen{}pi, pi{]}

\item[{Return type}] \leavevmode
\sphinxAtStartPar
float

\end{description}\end{quote}

\end{fulllineitems}



\subsubsection{tools.maths.asin2}
\label{\detokenize{_autosummary/tools.maths.asin2:tools-maths-asin2}}\label{\detokenize{_autosummary/tools.maths.asin2::doc}}\index{asin2() (in module tools.maths)@\spxentry{asin2()}\spxextra{in module tools.maths}}

\begin{fulllineitems}
\phantomsection\label{\detokenize{_autosummary/tools.maths.asin2:tools.maths.asin2}}\pysiglinewithargsret{\sphinxcode{\sphinxupquote{tools.maths.}}\sphinxbfcode{\sphinxupquote{asin2}}}{\emph{\DUrole{n}{x}\DUrole{p}{:} \DUrole{n}{float}}, \emph{\DUrole{n}{y}\DUrole{p}{:} \DUrole{n}{float}}}{{ $\rightarrow$ float}}
\sphinxAtStartPar
Function to return the inverse sin function across the range (\sphinxhyphen{}pi, pi{]}, rather than (\sphinxhyphen{}pi/2, pi/2{]}
\begin{quote}\begin{description}
\item[{Parameters}] \leavevmode\begin{itemize}
\item {} 
\sphinxAtStartPar
\sphinxstyleliteralstrong{\sphinxupquote{x}} (\sphinxstyleliteralemphasis{\sphinxupquote{float}}) \textendash{} x coordinate of the point in 2D space

\item {} 
\sphinxAtStartPar
\sphinxstyleliteralstrong{\sphinxupquote{y}} (\sphinxstyleliteralemphasis{\sphinxupquote{float}}) \textendash{} y coordinate of the point in 2D space

\end{itemize}

\item[{Returns}] \leavevmode
\sphinxAtStartPar
\sphinxstylestrong{theta} \textendash{} Angle corresponding to point in 2D space in radial coordinates, within range (\sphinxhyphen{}pi, pi{]}

\item[{Return type}] \leavevmode
\sphinxAtStartPar
float

\end{description}\end{quote}

\end{fulllineitems}



\subsubsection{tools.maths.filter\_butterworth}
\label{\detokenize{_autosummary/tools.maths.filter_butterworth:tools-maths-filter-butterworth}}\label{\detokenize{_autosummary/tools.maths.filter_butterworth::doc}}\index{filter\_butterworth() (in module tools.maths)@\spxentry{filter\_butterworth()}\spxextra{in module tools.maths}}

\begin{fulllineitems}
\phantomsection\label{\detokenize{_autosummary/tools.maths.filter_butterworth:tools.maths.filter_butterworth}}\pysiglinewithargsret{\sphinxcode{\sphinxupquote{tools.maths.}}\sphinxbfcode{\sphinxupquote{filter\_butterworth}}}{\emph{\DUrole{n}{data}\DUrole{p}{:} \DUrole{n}{Union\DUrole{p}{{[}}numpy.ndarray\DUrole{p}{, }pandas.core.frame.DataFrame\DUrole{p}{{]}}}}, \emph{\DUrole{n}{sample\_freq}\DUrole{p}{:} \DUrole{n}{float} \DUrole{o}{=} \DUrole{default_value}{500.0}}, \emph{\DUrole{n}{freq\_filter}\DUrole{p}{:} \DUrole{n}{float} \DUrole{o}{=} \DUrole{default_value}{40}}, \emph{\DUrole{n}{order}\DUrole{p}{:} \DUrole{n}{int} \DUrole{o}{=} \DUrole{default_value}{2}}, \emph{\DUrole{n}{filter\_type}\DUrole{p}{:} \DUrole{n}{str} \DUrole{o}{=} \DUrole{default_value}{\textquotesingle{}low\textquotesingle{}}}}{{ $\rightarrow$ Union\DUrole{p}{{[}}numpy.ndarray\DUrole{p}{, }pandas.core.frame.DataFrame\DUrole{p}{{]}}}}
\sphinxAtStartPar
Filter data using Butterworth filter

\sphinxAtStartPar
Filter a given set of data using a Butterworth filter, designed to have a specific passband for desired
frequencies. It is set up to use seconds, not milliseconds.
\begin{quote}\begin{description}
\item[{Parameters}] \leavevmode\begin{itemize}
\item {} 
\sphinxAtStartPar
\sphinxstyleliteralstrong{\sphinxupquote{data}} (\sphinxstyleliteralemphasis{\sphinxupquote{np.ndarray}}\sphinxstyleliteralemphasis{\sphinxupquote{ or }}\sphinxstyleliteralemphasis{\sphinxupquote{pd.DataFrame}}) \textendash{} Data to filter

\item {} 
\sphinxAtStartPar
\sphinxstyleliteralstrong{\sphinxupquote{sample\_freq}} (\sphinxstyleliteralemphasis{\sphinxupquote{int}}\sphinxstyleliteralemphasis{\sphinxupquote{ or }}\sphinxstyleliteralemphasis{\sphinxupquote{float}}) \textendash{} Sampling rate of data (Hz), default=500. If data passed as dataframe, the sample\_freq will be calculated from
the dataframe index.

\item {} 
\sphinxAtStartPar
\sphinxstyleliteralstrong{\sphinxupquote{freq\_filter}} (\sphinxstyleliteralemphasis{\sphinxupquote{int}}\sphinxstyleliteralemphasis{\sphinxupquote{ or }}\sphinxstyleliteralemphasis{\sphinxupquote{float}}) \textendash{} Cut\sphinxhyphen{}off frequency for filter, default=40

\item {} 
\sphinxAtStartPar
\sphinxstyleliteralstrong{\sphinxupquote{order}} (\sphinxstyleliteralemphasis{\sphinxupquote{int}}) \textendash{} Order of the Butterworth filter, default=2

\item {} 
\sphinxAtStartPar
\sphinxstyleliteralstrong{\sphinxupquote{filter\_type}} (\sphinxstyleliteralemphasis{\sphinxupquote{\{\textquotesingle{}low\textquotesingle{}}}\sphinxstyleliteralemphasis{\sphinxupquote{, }}\sphinxstyleliteralemphasis{\sphinxupquote{\textquotesingle{}high\textquotesingle{}}}\sphinxstyleliteralemphasis{\sphinxupquote{, }}\sphinxstyleliteralemphasis{\sphinxupquote{\textquotesingle{}band\textquotesingle{}\}}}) \textendash{} Type of filter to use, default=’low’

\end{itemize}

\item[{Returns}] \leavevmode
\sphinxAtStartPar
\sphinxstylestrong{filter\_out} \textendash{} Output filtered data

\item[{Return type}] \leavevmode
\sphinxAtStartPar
np.ndarray

\end{description}\end{quote}

\end{fulllineitems}



\subsubsection{tools.maths.filter\_savitzkygolay}
\label{\detokenize{_autosummary/tools.maths.filter_savitzkygolay:tools-maths-filter-savitzkygolay}}\label{\detokenize{_autosummary/tools.maths.filter_savitzkygolay::doc}}\index{filter\_savitzkygolay() (in module tools.maths)@\spxentry{filter\_savitzkygolay()}\spxextra{in module tools.maths}}

\begin{fulllineitems}
\phantomsection\label{\detokenize{_autosummary/tools.maths.filter_savitzkygolay:tools.maths.filter_savitzkygolay}}\pysiglinewithargsret{\sphinxcode{\sphinxupquote{tools.maths.}}\sphinxbfcode{\sphinxupquote{filter\_savitzkygolay}}}{\emph{\DUrole{n}{data}\DUrole{p}{:} \DUrole{n}{pandas.core.frame.DataFrame}}, \emph{\DUrole{n}{window\_length}\DUrole{p}{:} \DUrole{n}{int} \DUrole{o}{=} \DUrole{default_value}{0.05}}, \emph{\DUrole{n}{order}\DUrole{p}{:} \DUrole{n}{int} \DUrole{o}{=} \DUrole{default_value}{2}}, \emph{\DUrole{n}{deriv}\DUrole{p}{:} \DUrole{n}{int} \DUrole{o}{=} \DUrole{default_value}{0}}, \emph{\DUrole{n}{delta}\DUrole{p}{:} \DUrole{n}{float} \DUrole{o}{=} \DUrole{default_value}{1.0}}}{}
\sphinxAtStartPar
Filter EGM data using a Savitzky\sphinxhyphen{}Golay filter

\sphinxAtStartPar
Filter a given set of data using a Savitzky\sphinxhyphen{}Golay filter, designed to smooth data using a convolution process
fitting to a low\sphinxhyphen{}degree polynomial within a given window. Default values are either taken from scipy
documentation (not all options are provided here), or adapted to match Hermans et al.
\begin{quote}\begin{description}
\item[{Parameters}] \leavevmode\begin{itemize}
\item {} 
\sphinxAtStartPar
\sphinxstyleliteralstrong{\sphinxupquote{data}} (\sphinxstyleliteralemphasis{\sphinxupquote{pd.DataFrame}}) \textendash{} Data to filter

\item {} 
\sphinxAtStartPar
\sphinxstyleliteralstrong{\sphinxupquote{window\_length}} (\sphinxstyleliteralemphasis{\sphinxupquote{float}}\sphinxstyleliteralemphasis{\sphinxupquote{, }}\sphinxstyleliteralemphasis{\sphinxupquote{optional}}) \textendash{} The length of the filter window in seconds. When passed to the scipy filter, will be converted to a
positive odd integer (i.e. the number of coefficients). Default=0.05

\item {} 
\sphinxAtStartPar
\sphinxstyleliteralstrong{\sphinxupquote{order}} (\sphinxstyleliteralemphasis{\sphinxupquote{int}}\sphinxstyleliteralemphasis{\sphinxupquote{, }}\sphinxstyleliteralemphasis{\sphinxupquote{optional}}) \textendash{} The order of the polynomial used to fit the samples. polyorder must be less than window\_length. Default=2

\item {} 
\sphinxAtStartPar
\sphinxstyleliteralstrong{\sphinxupquote{deriv}} (\sphinxstyleliteralemphasis{\sphinxupquote{int}}\sphinxstyleliteralemphasis{\sphinxupquote{, }}\sphinxstyleliteralemphasis{\sphinxupquote{optional}}) \textendash{} The order of the derivative to compute. This must be a nonnegative integer. The default is 0, which means to
filter the data without differentiating.

\item {} 
\sphinxAtStartPar
\sphinxstyleliteralstrong{\sphinxupquote{delta}} (\sphinxstyleliteralemphasis{\sphinxupquote{float}}\sphinxstyleliteralemphasis{\sphinxupquote{, }}\sphinxstyleliteralemphasis{\sphinxupquote{optional}}) \textendash{} The spacing of the samples to which the filter will be applied. This is only used if deriv \textgreater{} 0. Default=1.0

\end{itemize}

\item[{Returns}] \leavevmode
\sphinxAtStartPar
\sphinxstylestrong{data} \textendash{} Output filtered data

\item[{Return type}] \leavevmode
\sphinxAtStartPar
pd.DataFrame

\end{description}\end{quote}
\subsubsection*{References}
\begin{description}
\item[{The development and validation of an easy to use automatic QT\sphinxhyphen{}interval algorithm}] \leavevmode
\sphinxAtStartPar
Hermans BJM, Vink AS, Bennis FC, Filippini LH, Meijborg VMF, Wilde AAM, Pison L, Postema PG, Delhaas T
PLoS ONE, 12(9), 1\textendash{}14 (2017)
\sphinxurl{https://doi.org/10.1371/journal.pone.0184352}

\end{description}

\end{fulllineitems}



\subsubsection{tools.maths.get\_median}
\label{\detokenize{_autosummary/tools.maths.get_median:tools-maths-get-median}}\label{\detokenize{_autosummary/tools.maths.get_median::doc}}\index{get\_median() (in module tools.maths)@\spxentry{get\_median()}\spxextra{in module tools.maths}}

\begin{fulllineitems}
\phantomsection\label{\detokenize{_autosummary/tools.maths.get_median:tools.maths.get_median}}\pysiglinewithargsret{\sphinxcode{\sphinxupquote{tools.maths.}}\sphinxbfcode{\sphinxupquote{get\_median}}}{\emph{\DUrole{n}{data}\DUrole{p}{:} \DUrole{n}{pandas.core.frame.DataFrame}}, \emph{\DUrole{n}{remove\_outliers}\DUrole{p}{:} \DUrole{n}{bool} \DUrole{o}{=} \DUrole{default_value}{True}}}{{ $\rightarrow$ pandas.core.frame.DataFrame}}
\sphinxAtStartPar
Add the median value of data to a dataframe

\sphinxAtStartPar
TODO: Complete this code if required (currently only potentially useful for T\sphinxhyphen{}wave analysis)

\end{fulllineitems}



\subsubsection{tools.maths.normalise\_signal}
\label{\detokenize{_autosummary/tools.maths.normalise_signal:tools-maths-normalise-signal}}\label{\detokenize{_autosummary/tools.maths.normalise_signal::doc}}\index{normalise\_signal() (in module tools.maths)@\spxentry{normalise\_signal()}\spxextra{in module tools.maths}}

\begin{fulllineitems}
\phantomsection\label{\detokenize{_autosummary/tools.maths.normalise_signal:tools.maths.normalise_signal}}\pysiglinewithargsret{\sphinxcode{\sphinxupquote{tools.maths.}}\sphinxbfcode{\sphinxupquote{normalise\_signal}}}{\emph{\DUrole{n}{data}\DUrole{p}{:} \DUrole{n}{Union\DUrole{p}{{[}}numpy.ndarray\DUrole{p}{, }pandas.core.frame.DataFrame\DUrole{p}{{]}}}}}{{ $\rightarrow$ Union\DUrole{p}{{[}}numpy.ndarray\DUrole{p}{, }pandas.core.frame.DataFrame\DUrole{p}{{]}}}}
\sphinxAtStartPar
Returns a normalised signal, such that the maximum value in the signal is 1, or the minimum is \sphinxhyphen{}1
\begin{quote}\begin{description}
\item[{Parameters}] \leavevmode
\sphinxAtStartPar
\sphinxstyleliteralstrong{\sphinxupquote{data}} (\sphinxstyleliteralemphasis{\sphinxupquote{np.ndarray}}) \textendash{} Signal to be normalised

\item[{Returns}] \leavevmode
\sphinxAtStartPar
\sphinxstylestrong{normalised\_data} \textendash{} Normalised signal

\item[{Return type}] \leavevmode
\sphinxAtStartPar
np.ndarray or pd.DataFrame

\end{description}\end{quote}

\end{fulllineitems}



\subsubsection{tools.maths.simplex\_volume}
\label{\detokenize{_autosummary/tools.maths.simplex_volume:tools-maths-simplex-volume}}\label{\detokenize{_autosummary/tools.maths.simplex_volume::doc}}\index{simplex\_volume() (in module tools.maths)@\spxentry{simplex\_volume()}\spxextra{in module tools.maths}}

\begin{fulllineitems}
\phantomsection\label{\detokenize{_autosummary/tools.maths.simplex_volume:tools.maths.simplex_volume}}\pysiglinewithargsret{\sphinxcode{\sphinxupquote{tools.maths.}}\sphinxbfcode{\sphinxupquote{simplex\_volume}}}{\emph{\DUrole{o}{*}}, \emph{\DUrole{n}{vertices}\DUrole{o}{=}\DUrole{default_value}{None}}, \emph{\DUrole{n}{sides}\DUrole{o}{=}\DUrole{default_value}{None}}}{{ $\rightarrow$ float}}
\sphinxAtStartPar
Return the volume of the simplex with given vertices or sides.

\sphinxAtStartPar
If vertices are given they must be in a NumPy array with shape (N+1, N):
the position vectors of the N+1 vertices in N dimensions. If the sides
are given, they must be the compressed pairwise distance matrix as
returned from scipy.spatial.distance.pdist.

\sphinxAtStartPar
Raises a ValueError if the vertices do not form a simplex (for example,
because they are coplanar, colinear or coincident).

\sphinxAtStartPar
Warning: this algorithm has not been tested for numerical stability.

\end{fulllineitems}



\subsection{tools.plotting}
\label{\detokenize{_autosummary/tools.plotting:module-tools.plotting}}\label{\detokenize{_autosummary/tools.plotting:tools-plotting}}\label{\detokenize{_autosummary/tools.plotting::doc}}\index{module@\spxentry{module}!tools.plotting@\spxentry{tools.plotting}}\index{tools.plotting@\spxentry{tools.plotting}!module@\spxentry{module}}\subsubsection*{Functions}


\begin{savenotes}\sphinxatlongtablestart\begin{longtable}[c]{\X{1}{2}\X{1}{2}}
\hline

\endfirsthead

\multicolumn{2}{c}%
{\makebox[0pt]{\sphinxtablecontinued{\tablename\ \thetable{} \textendash{} continued from previous page}}}\\
\hline

\endhead

\hline
\multicolumn{2}{r}{\makebox[0pt][r]{\sphinxtablecontinued{continues on next page}}}\\
\endfoot

\endlastfoot

\sphinxAtStartPar
{\hyperref[\detokenize{_autosummary/tools.plotting.add_colourbar:tools.plotting.add_colourbar}]{\sphinxcrossref{\sphinxcode{\sphinxupquote{add\_colourbar}}}}}
&
\sphinxAtStartPar
Add arbitrary colourbar to a figure, for instances when an automatic colorbar isn\textquotesingle{}t available
\\
\hline
\sphinxAtStartPar
{\hyperref[\detokenize{_autosummary/tools.plotting.add_xyz_axes:tools.plotting.add_xyz_axes}]{\sphinxcrossref{\sphinxcode{\sphinxupquote{add\_xyz\_axes}}}}}
&
\sphinxAtStartPar
Plot dummy axes (can\textquotesingle{}t move splines in 3D plots)
\\
\hline
\sphinxAtStartPar
{\hyperref[\detokenize{_autosummary/tools.plotting.get_plot_colours:tools.plotting.get_plot_colours}]{\sphinxcrossref{\sphinxcode{\sphinxupquote{get\_plot\_colours}}}}}
&
\sphinxAtStartPar
Return iterable list of RGB colour values that can be used for custom plotting functions
\\
\hline
\sphinxAtStartPar
{\hyperref[\detokenize{_autosummary/tools.plotting.get_plot_lines:tools.plotting.get_plot_lines}]{\sphinxcrossref{\sphinxcode{\sphinxupquote{get\_plot\_lines}}}}}
&
\sphinxAtStartPar
Returns different line\sphinxhyphen{}styles for plotting
\\
\hline
\sphinxAtStartPar
{\hyperref[\detokenize{_autosummary/tools.plotting.set_axis_limits:tools.plotting.set_axis_limits}]{\sphinxcrossref{\sphinxcode{\sphinxupquote{set\_axis\_limits}}}}}
&
\sphinxAtStartPar
Set axis limits (not automatic for line collections, so needs to be done manually)
\\
\hline
\sphinxAtStartPar
{\hyperref[\detokenize{_autosummary/tools.plotting.set_symmetrical_axis_limits:tools.plotting.set_symmetrical_axis_limits}]{\sphinxcrossref{\sphinxcode{\sphinxupquote{set\_symmetrical\_axis\_limits}}}}}
&
\sphinxAtStartPar
Sets symmetrical limits for a series of axes
\\
\hline
\sphinxAtStartPar
{\hyperref[\detokenize{_autosummary/tools.plotting.write_colourmap_to_xml:tools.plotting.write_colourmap_to_xml}]{\sphinxcrossref{\sphinxcode{\sphinxupquote{write\_colourmap\_to\_xml}}}}}
&
\sphinxAtStartPar
Create a Paraview friendly colourmap useful for highlighting a particular range
\\
\hline
\end{longtable}\sphinxatlongtableend\end{savenotes}


\subsubsection{tools.plotting.add\_colourbar}
\label{\detokenize{_autosummary/tools.plotting.add_colourbar:tools-plotting-add-colourbar}}\label{\detokenize{_autosummary/tools.plotting.add_colourbar::doc}}\index{add\_colourbar() (in module tools.plotting)@\spxentry{add\_colourbar()}\spxextra{in module tools.plotting}}

\begin{fulllineitems}
\phantomsection\label{\detokenize{_autosummary/tools.plotting.add_colourbar:tools.plotting.add_colourbar}}\pysiglinewithargsret{\sphinxcode{\sphinxupquote{tools.plotting.}}\sphinxbfcode{\sphinxupquote{add\_colourbar}}}{\emph{\DUrole{n}{limits}\DUrole{p}{:} \DUrole{n}{List\DUrole{p}{{[}}float\DUrole{p}{{]}}}}, \emph{\DUrole{n}{fig}\DUrole{p}{:} \DUrole{n}{Optional\DUrole{p}{{[}}matplotlib.pyplot.figure\DUrole{p}{{]}}} \DUrole{o}{=} \DUrole{default_value}{None}}, \emph{\DUrole{n}{colourmap}\DUrole{p}{:} \DUrole{n}{str} \DUrole{o}{=} \DUrole{default_value}{\textquotesingle{}viridis\textquotesingle{}}}, \emph{\DUrole{n}{n\_elements}\DUrole{p}{:} \DUrole{n}{int} \DUrole{o}{=} \DUrole{default_value}{100}}}{{ $\rightarrow$ None}}
\sphinxAtStartPar
Add arbitrary colourbar to a figure, for instances when an automatic colorbar isn’t available
\begin{quote}\begin{description}
\item[{Parameters}] \leavevmode\begin{itemize}
\item {} 
\sphinxAtStartPar
\sphinxstyleliteralstrong{\sphinxupquote{limits}} (\sphinxstyleliteralemphasis{\sphinxupquote{list of float}}) \textendash{} Numerical limits to apply

\item {} 
\sphinxAtStartPar
\sphinxstyleliteralstrong{\sphinxupquote{fig}} (\sphinxstyleliteralemphasis{\sphinxupquote{plt.figure}}\sphinxstyleliteralemphasis{\sphinxupquote{, }}\sphinxstyleliteralemphasis{\sphinxupquote{optional}}) \textendash{} Figure on which to plot the colourbar. If not provided (default=None), then will pick up the figure most
recently available

\item {} 
\sphinxAtStartPar
\sphinxstyleliteralstrong{\sphinxupquote{colourmap}} (\sphinxstyleliteralemphasis{\sphinxupquote{str}}\sphinxstyleliteralemphasis{\sphinxupquote{, }}\sphinxstyleliteralemphasis{\sphinxupquote{optional}}) \textendash{} Colourmap to be used, default=’viridis’

\item {} 
\sphinxAtStartPar
\sphinxstyleliteralstrong{\sphinxupquote{n\_elements}} (\sphinxstyleliteralemphasis{\sphinxupquote{int}}\sphinxstyleliteralemphasis{\sphinxupquote{, }}\sphinxstyleliteralemphasis{\sphinxupquote{optional}}) \textendash{} Number of entries to be made in the colourmap index, default=100

\end{itemize}

\end{description}\end{quote}
\subsubsection*{Notes}

\sphinxAtStartPar
This is useful for instances such as when LineCollections are used to plot line that changes colour during the
plotting process, as LineCollections do not enable an automatic colorbar to be added to the plot. This function
adds a dummy colorbar to replace that.

\end{fulllineitems}



\subsubsection{tools.plotting.add\_xyz\_axes}
\label{\detokenize{_autosummary/tools.plotting.add_xyz_axes:tools-plotting-add-xyz-axes}}\label{\detokenize{_autosummary/tools.plotting.add_xyz_axes::doc}}\index{add\_xyz\_axes() (in module tools.plotting)@\spxentry{add\_xyz\_axes()}\spxextra{in module tools.plotting}}

\begin{fulllineitems}
\phantomsection\label{\detokenize{_autosummary/tools.plotting.add_xyz_axes:tools.plotting.add_xyz_axes}}\pysiglinewithargsret{\sphinxcode{\sphinxupquote{tools.plotting.}}\sphinxbfcode{\sphinxupquote{add\_xyz\_axes}}}{\emph{\DUrole{n}{fig}\DUrole{p}{:} \DUrole{n}{matplotlib.pyplot.figure}}, \emph{\DUrole{n}{ax}\DUrole{p}{:} \DUrole{n}{mpl\_toolkits.mplot3d.axes3d.Axes3D}}, \emph{\DUrole{n}{axis\_limits}\DUrole{p}{:} \DUrole{n}{Optional\DUrole{p}{{[}}Union\DUrole{p}{{[}}float\DUrole{p}{, }List\DUrole{p}{{[}}float\DUrole{p}{{]}}\DUrole{p}{, }List\DUrole{p}{{[}}List\DUrole{p}{{[}}float\DUrole{p}{{]}}\DUrole{p}{{]}}\DUrole{p}{{]}}\DUrole{p}{{]}}} \DUrole{o}{=} \DUrole{default_value}{None}}, \emph{\DUrole{n}{symmetrical\_axes}\DUrole{p}{:} \DUrole{n}{bool} \DUrole{o}{=} \DUrole{default_value}{False}}, \emph{\DUrole{n}{equal\_limits}\DUrole{p}{:} \DUrole{n}{bool} \DUrole{o}{=} \DUrole{default_value}{False}}, \emph{\DUrole{n}{unit\_axes}\DUrole{p}{:} \DUrole{n}{bool} \DUrole{o}{=} \DUrole{default_value}{False}}, \emph{\DUrole{n}{sig\_fig}\DUrole{p}{:} \DUrole{n}{Optional\DUrole{p}{{[}}int\DUrole{p}{{]}}} \DUrole{o}{=} \DUrole{default_value}{None}}}{{ $\rightarrow$ None}}
\sphinxAtStartPar
Plot dummy axes (can’t move splines in 3D plots)
\begin{quote}\begin{description}
\item[{Parameters}] \leavevmode\begin{itemize}
\item {} 
\sphinxAtStartPar
\sphinxstyleliteralstrong{\sphinxupquote{fig}} (\sphinxstyleliteralemphasis{\sphinxupquote{plt.figure}}) \textendash{} Figure handle

\item {} 
\sphinxAtStartPar
\sphinxstyleliteralstrong{\sphinxupquote{ax}} (\sphinxstyleliteralemphasis{\sphinxupquote{Axes3D}}) \textendash{} Axis handle

\item {} 
\sphinxAtStartPar
\sphinxstyleliteralstrong{\sphinxupquote{axis\_limits}} (\sphinxstyleliteralemphasis{\sphinxupquote{float}}\sphinxstyleliteralemphasis{\sphinxupquote{ or }}\sphinxstyleliteralemphasis{\sphinxupquote{list of float}}\sphinxstyleliteralemphasis{\sphinxupquote{ or }}\sphinxstyleliteralemphasis{\sphinxupquote{list of list of float}}\sphinxstyleliteralemphasis{\sphinxupquote{, }}\sphinxstyleliteralemphasis{\sphinxupquote{optional}}) \textendash{} Axis limits, either same for all dimensions (min=\sphinxhyphen{}max), or individual limits ({[}min, max{]}), or individual limits
for each dimension

\item {} 
\sphinxAtStartPar
\sphinxstyleliteralstrong{\sphinxupquote{symmetrical\_axes}} (\sphinxstyleliteralemphasis{\sphinxupquote{bool}}\sphinxstyleliteralemphasis{\sphinxupquote{, }}\sphinxstyleliteralemphasis{\sphinxupquote{optional}}) \textendash{} Apply same limits to x, y and z axes

\item {} 
\sphinxAtStartPar
\sphinxstyleliteralstrong{\sphinxupquote{equal\_limits}} (\sphinxstyleliteralemphasis{\sphinxupquote{bool}}\sphinxstyleliteralemphasis{\sphinxupquote{, }}\sphinxstyleliteralemphasis{\sphinxupquote{optional}}) \textendash{} Set axis minimum to minus axis maximum (or vice versa)

\item {} 
\sphinxAtStartPar
\sphinxstyleliteralstrong{\sphinxupquote{unit\_axes}} (\sphinxstyleliteralemphasis{\sphinxupquote{bool}}\sphinxstyleliteralemphasis{\sphinxupquote{, }}\sphinxstyleliteralemphasis{\sphinxupquote{optional}}) \textendash{} Apply minimum of \sphinxhyphen{}1 \sphinxhyphen{}\textgreater{} 1 for axis limits

\item {} 
\sphinxAtStartPar
\sphinxstyleliteralstrong{\sphinxupquote{sig\_fig}} (\sphinxstyleliteralemphasis{\sphinxupquote{int}}\sphinxstyleliteralemphasis{\sphinxupquote{, }}\sphinxstyleliteralemphasis{\sphinxupquote{optional}}) \textendash{} Maximum number of decimal places to be used on the axis plots (e.g., if set to 2, 0.12345 will be displayed
as 0.12). Used to avoid floating point errors, default=None (no adaption made)

\end{itemize}

\end{description}\end{quote}

\end{fulllineitems}



\subsubsection{tools.plotting.get\_plot\_colours}
\label{\detokenize{_autosummary/tools.plotting.get_plot_colours:tools-plotting-get-plot-colours}}\label{\detokenize{_autosummary/tools.plotting.get_plot_colours::doc}}\index{get\_plot\_colours() (in module tools.plotting)@\spxentry{get\_plot\_colours()}\spxextra{in module tools.plotting}}

\begin{fulllineitems}
\phantomsection\label{\detokenize{_autosummary/tools.plotting.get_plot_colours:tools.plotting.get_plot_colours}}\pysiglinewithargsret{\sphinxcode{\sphinxupquote{tools.plotting.}}\sphinxbfcode{\sphinxupquote{get\_plot\_colours}}}{\emph{\DUrole{n}{n}\DUrole{p}{:} \DUrole{n}{int} \DUrole{o}{=} \DUrole{default_value}{10}}, \emph{\DUrole{n}{colourmap}\DUrole{p}{:} \DUrole{n}{Optional\DUrole{p}{{[}}str\DUrole{p}{{]}}} \DUrole{o}{=} \DUrole{default_value}{None}}}{{ $\rightarrow$ List\DUrole{p}{{[}}Tuple\DUrole{p}{{[}}float\DUrole{p}{{]}}\DUrole{p}{{]}}}}
\sphinxAtStartPar
Return iterable list of RGB colour values that can be used for custom plotting functions

\sphinxAtStartPar
Returns a list of RGB colours values, potentially according to a specified colourmap. If n is low enough, will use
the custom ‘tab10’ colourmap by default, which will use alternating colours as much as possible to maximise
visibility. If n is too big, then the default setting is ‘viridis’, which should provide a gradation of colour from
first to last.
\begin{quote}\begin{description}
\item[{Parameters}] \leavevmode\begin{itemize}
\item {} 
\sphinxAtStartPar
\sphinxstyleliteralstrong{\sphinxupquote{n}} (\sphinxstyleliteralemphasis{\sphinxupquote{int}}\sphinxstyleliteralemphasis{\sphinxupquote{, }}\sphinxstyleliteralemphasis{\sphinxupquote{optional}}) \textendash{} Number of distinct colours required, default=10

\item {} 
\sphinxAtStartPar
\sphinxstyleliteralstrong{\sphinxupquote{colourmap}} (\sphinxstyleliteralemphasis{\sphinxupquote{str}}) \textendash{} Matplotlib colourmap to base the end result on. Will default to ‘tab10’ if n\textless{}11, ‘viridis’ otherwise

\end{itemize}

\item[{Returns}] \leavevmode
\sphinxAtStartPar
\sphinxstylestrong{cmap} \textendash{} List of RGB values

\item[{Return type}] \leavevmode
\sphinxAtStartPar
list of tuple

\end{description}\end{quote}

\end{fulllineitems}



\subsubsection{tools.plotting.get\_plot\_lines}
\label{\detokenize{_autosummary/tools.plotting.get_plot_lines:tools-plotting-get-plot-lines}}\label{\detokenize{_autosummary/tools.plotting.get_plot_lines::doc}}\index{get\_plot\_lines() (in module tools.plotting)@\spxentry{get\_plot\_lines()}\spxextra{in module tools.plotting}}

\begin{fulllineitems}
\phantomsection\label{\detokenize{_autosummary/tools.plotting.get_plot_lines:tools.plotting.get_plot_lines}}\pysiglinewithargsret{\sphinxcode{\sphinxupquote{tools.plotting.}}\sphinxbfcode{\sphinxupquote{get\_plot\_lines}}}{\emph{\DUrole{n}{n}\DUrole{p}{:} \DUrole{n}{int} \DUrole{o}{=} \DUrole{default_value}{4}}}{{ $\rightarrow$ Union\DUrole{p}{{[}}List\DUrole{p}{{[}}tuple\DUrole{p}{{]}}\DUrole{p}{, }List\DUrole{p}{{[}}str\DUrole{p}{{]}}\DUrole{p}{{]}}}}
\sphinxAtStartPar
Returns different line\sphinxhyphen{}styles for plotting
\begin{quote}\begin{description}
\item[{Parameters}] \leavevmode
\sphinxAtStartPar
\sphinxstyleliteralstrong{\sphinxupquote{n}} (\sphinxstyleliteralemphasis{\sphinxupquote{int}}\sphinxstyleliteralemphasis{\sphinxupquote{, }}\sphinxstyleliteralemphasis{\sphinxupquote{optional}}) \textendash{} Number of different line\sphinxhyphen{}styles required

\item[{Returns}] \leavevmode
\sphinxAtStartPar
\sphinxstylestrong{lines} \textendash{} List of different line\sphinxhyphen{}styles

\item[{Return type}] \leavevmode
\sphinxAtStartPar
list of str or list of tuple

\end{description}\end{quote}

\end{fulllineitems}



\subsubsection{tools.plotting.set\_axis\_limits}
\label{\detokenize{_autosummary/tools.plotting.set_axis_limits:tools-plotting-set-axis-limits}}\label{\detokenize{_autosummary/tools.plotting.set_axis_limits::doc}}\index{set\_axis\_limits() (in module tools.plotting)@\spxentry{set\_axis\_limits()}\spxextra{in module tools.plotting}}

\begin{fulllineitems}
\phantomsection\label{\detokenize{_autosummary/tools.plotting.set_axis_limits:tools.plotting.set_axis_limits}}\pysiglinewithargsret{\sphinxcode{\sphinxupquote{tools.plotting.}}\sphinxbfcode{\sphinxupquote{set\_axis\_limits}}}{\emph{\DUrole{n}{ax}}, \emph{\DUrole{n}{data}\DUrole{p}{:} \DUrole{n}{Optional\DUrole{p}{{[}}pandas.core.frame.DataFrame\DUrole{p}{{]}}} \DUrole{o}{=} \DUrole{default_value}{None}}, \emph{\DUrole{n}{unit\_min}\DUrole{p}{:} \DUrole{n}{bool} \DUrole{o}{=} \DUrole{default_value}{True}}, \emph{\DUrole{n}{axis\_limits}\DUrole{p}{:} \DUrole{n}{Optional\DUrole{p}{{[}}Union\DUrole{p}{{[}}float\DUrole{p}{, }List\DUrole{p}{{[}}float\DUrole{p}{{]}}\DUrole{p}{{]}}\DUrole{p}{{]}}} \DUrole{o}{=} \DUrole{default_value}{None}}, \emph{\DUrole{n}{pad\_percent}\DUrole{p}{:} \DUrole{n}{float} \DUrole{o}{=} \DUrole{default_value}{0.01}}}{{ $\rightarrow$ None}}
\sphinxAtStartPar
Set axis limits (not automatic for line collections, so needs to be done manually)
\begin{quote}\begin{description}
\item[{Parameters}] \leavevmode\begin{itemize}
\item {} 
\sphinxAtStartPar
\sphinxstyleliteralstrong{\sphinxupquote{ax}} \textendash{} Handles to the axes that need to be adjusted

\item {} 
\sphinxAtStartPar
\sphinxstyleliteralstrong{\sphinxupquote{data}} (\sphinxstyleliteralemphasis{\sphinxupquote{pd.DataFrame}}\sphinxstyleliteralemphasis{\sphinxupquote{, }}\sphinxstyleliteralemphasis{\sphinxupquote{optional}}) \textendash{} Data that has been plotted, default=None

\item {} 
\sphinxAtStartPar
\sphinxstyleliteralstrong{\sphinxupquote{unit\_min}} (\sphinxstyleliteralemphasis{\sphinxupquote{bool}}\sphinxstyleliteralemphasis{\sphinxupquote{, }}\sphinxstyleliteralemphasis{\sphinxupquote{optional}}) \textendash{} Whether to have the axes set to, as a minimum, unit length, default=True

\item {} 
\sphinxAtStartPar
\sphinxstyleliteralstrong{\sphinxupquote{axis\_limits}} (\sphinxstyleliteralemphasis{\sphinxupquote{list of float}}\sphinxstyleliteralemphasis{\sphinxupquote{ or }}\sphinxstyleliteralemphasis{\sphinxupquote{float}}\sphinxstyleliteralemphasis{\sphinxupquote{, }}\sphinxstyleliteralemphasis{\sphinxupquote{optional}}) \textendash{} Min/max values for axes, either as one value (i.e. min=\sphinxhyphen{}max), or two separate values. Same axis limits will
be applied to all dimensions

\item {} 
\sphinxAtStartPar
\sphinxstyleliteralstrong{\sphinxupquote{pad\_percent}} (\sphinxstyleliteralemphasis{\sphinxupquote{float}}\sphinxstyleliteralemphasis{\sphinxupquote{, }}\sphinxstyleliteralemphasis{\sphinxupquote{optional}}) \textendash{} Percentage ‘padding’ to add to the ranges, to try and ensure that the edges of linewidths are not cut off,
default=0.01

\end{itemize}

\end{description}\end{quote}

\end{fulllineitems}



\subsubsection{tools.plotting.set\_symmetrical\_axis\_limits}
\label{\detokenize{_autosummary/tools.plotting.set_symmetrical_axis_limits:tools-plotting-set-symmetrical-axis-limits}}\label{\detokenize{_autosummary/tools.plotting.set_symmetrical_axis_limits::doc}}\index{set\_symmetrical\_axis\_limits() (in module tools.plotting)@\spxentry{set\_symmetrical\_axis\_limits()}\spxextra{in module tools.plotting}}

\begin{fulllineitems}
\phantomsection\label{\detokenize{_autosummary/tools.plotting.set_symmetrical_axis_limits:tools.plotting.set_symmetrical_axis_limits}}\pysiglinewithargsret{\sphinxcode{\sphinxupquote{tools.plotting.}}\sphinxbfcode{\sphinxupquote{set\_symmetrical\_axis\_limits}}}{\emph{\DUrole{n}{ax\_min}\DUrole{p}{:} \DUrole{n}{float}}, \emph{\DUrole{n}{ax\_max}\DUrole{p}{:} \DUrole{n}{float}}, \emph{\DUrole{n}{unit\_axes}\DUrole{p}{:} \DUrole{n}{bool} \DUrole{o}{=} \DUrole{default_value}{False}}}{{ $\rightarrow$ Tuple\DUrole{p}{{[}}float\DUrole{p}{, }float\DUrole{p}{{]}}}}
\sphinxAtStartPar
Sets symmetrical limits for a series of axes

\sphinxAtStartPar
TODO: fold functionality into set\_axis\_limits to avoid redundant functions
\begin{quote}\begin{description}
\item[{Parameters}] \leavevmode\begin{itemize}
\item {} 
\sphinxAtStartPar
\sphinxstyleliteralstrong{\sphinxupquote{ax\_min}} (\sphinxstyleliteralemphasis{\sphinxupquote{float}}) \textendash{} Minimum value for axes

\item {} 
\sphinxAtStartPar
\sphinxstyleliteralstrong{\sphinxupquote{ax\_max}} (\sphinxstyleliteralemphasis{\sphinxupquote{float}}) \textendash{} Maximum value for axes

\item {} 
\sphinxAtStartPar
\sphinxstyleliteralstrong{\sphinxupquote{unit\_axes}} (\sphinxstyleliteralemphasis{\sphinxupquote{bool}}\sphinxstyleliteralemphasis{\sphinxupquote{, }}\sphinxstyleliteralemphasis{\sphinxupquote{optional}}) \textendash{} Whether to apply a minimum axis range of {[}\sphinxhyphen{}1,1{]}

\end{itemize}

\item[{Returns}] \leavevmode
\sphinxAtStartPar
\sphinxstylestrong{ax\_min, ax\_max} \textendash{} Symmetrical axis limits, where ax\_min=\sphinxhyphen{}ax\_max

\item[{Return type}] \leavevmode
\sphinxAtStartPar
float

\end{description}\end{quote}

\end{fulllineitems}



\subsubsection{tools.plotting.write\_colourmap\_to\_xml}
\label{\detokenize{_autosummary/tools.plotting.write_colourmap_to_xml:tools-plotting-write-colourmap-to-xml}}\label{\detokenize{_autosummary/tools.plotting.write_colourmap_to_xml::doc}}\index{write\_colourmap\_to\_xml() (in module tools.plotting)@\spxentry{write\_colourmap\_to\_xml()}\spxextra{in module tools.plotting}}

\begin{fulllineitems}
\phantomsection\label{\detokenize{_autosummary/tools.plotting.write_colourmap_to_xml:tools.plotting.write_colourmap_to_xml}}\pysiglinewithargsret{\sphinxcode{\sphinxupquote{tools.plotting.}}\sphinxbfcode{\sphinxupquote{write\_colourmap\_to\_xml}}}{\emph{\DUrole{n}{start\_data}\DUrole{p}{:} \DUrole{n}{float}}, \emph{\DUrole{n}{end\_data}\DUrole{p}{:} \DUrole{n}{float}}, \emph{\DUrole{n}{start\_highlight}\DUrole{p}{:} \DUrole{n}{float}}, \emph{\DUrole{n}{end\_highlight}\DUrole{p}{:} \DUrole{n}{float}}, \emph{\DUrole{n}{opacity\_data}\DUrole{p}{:} \DUrole{n}{float} \DUrole{o}{=} \DUrole{default_value}{1}}, \emph{\DUrole{n}{opacity\_highlight}\DUrole{p}{:} \DUrole{n}{float} \DUrole{o}{=} \DUrole{default_value}{1}}, \emph{\DUrole{n}{n\_tags}\DUrole{p}{:} \DUrole{n}{int} \DUrole{o}{=} \DUrole{default_value}{20}}, \emph{\DUrole{n}{colourmap}\DUrole{p}{:} \DUrole{n}{str} \DUrole{o}{=} \DUrole{default_value}{\textquotesingle{}viridis\textquotesingle{}}}, \emph{\DUrole{n}{outfile}\DUrole{p}{:} \DUrole{n}{str} \DUrole{o}{=} \DUrole{default_value}{\textquotesingle{}colourmap.xml\textquotesingle{}}}}{{ $\rightarrow$ None}}
\sphinxAtStartPar
Create a Paraview friendly colourmap useful for highlighting a particular range

\sphinxAtStartPar
Creates a colourmap that is entirely gray, save for a specified region of interest that will vary according to the
specified colourmap
\begin{description}
\item[{start\_data                      start value for overall data (can’t just use data for region of interest \sphinxhyphen{}}] \leavevmode
\sphinxAtStartPar
Paraview will scale)

\end{description}

\sphinxAtStartPar
end\_data                        end value for overall data
start\_highlight                 start value for region of interest
end\_highlight                   end value for region of interest

\sphinxAtStartPar
opacity\_data        1.0                 overall opacity to use for all data
opacity\_highlight   1.0                 opacity for region of interest
colourmap           ‘viridis’           colourmap to use
outfile             ‘colourmap.xml’     filename to save .xml file under

\sphinxAtStartPar
None

\end{fulllineitems}



\subsection{tools.python}
\label{\detokenize{_autosummary/tools.python:module-tools.python}}\label{\detokenize{_autosummary/tools.python:tools-python}}\label{\detokenize{_autosummary/tools.python::doc}}\index{module@\spxentry{module}!tools.python@\spxentry{tools.python}}\index{tools.python@\spxentry{tools.python}!module@\spxentry{module}}\subsubsection*{Functions}


\begin{savenotes}\sphinxatlongtablestart\begin{longtable}[c]{\X{1}{2}\X{1}{2}}
\hline

\endfirsthead

\multicolumn{2}{c}%
{\makebox[0pt]{\sphinxtablecontinued{\tablename\ \thetable{} \textendash{} continued from previous page}}}\\
\hline

\endhead

\hline
\multicolumn{2}{r}{\makebox[0pt][r]{\sphinxtablecontinued{continues on next page}}}\\
\endfoot

\endlastfoot

\sphinxAtStartPar
{\hyperref[\detokenize{_autosummary/tools.python.check_list_depth:tools.python.check_list_depth}]{\sphinxcrossref{\sphinxcode{\sphinxupquote{check\_list\_depth}}}}}
&
\sphinxAtStartPar
Function to calculate the depth of nested loops
\\
\hline
\sphinxAtStartPar
{\hyperref[\detokenize{_autosummary/tools.python.convert_index_to_time:tools.python.convert_index_to_time}]{\sphinxcrossref{\sphinxcode{\sphinxupquote{convert\_index\_to\_time}}}}}
&
\sphinxAtStartPar
Return \textquotesingle{}real\textquotesingle{} time for a given index
\\
\hline
\sphinxAtStartPar
{\hyperref[\detokenize{_autosummary/tools.python.convert_input_to_list:tools.python.convert_input_to_list}]{\sphinxcrossref{\sphinxcode{\sphinxupquote{convert\_input\_to\_list}}}}}
&
\sphinxAtStartPar
Convert a given input to a list of inputs of required length.
\\
\hline
\sphinxAtStartPar
{\hyperref[\detokenize{_autosummary/tools.python.convert_time_to_index:tools.python.convert_time_to_index}]{\sphinxcrossref{\sphinxcode{\sphinxupquote{convert\_time\_to\_index}}}}}
&
\sphinxAtStartPar
Converts a given time point to the relevant index value
\\
\hline
\sphinxAtStartPar
{\hyperref[\detokenize{_autosummary/tools.python.deprecated_convert_time_to_index:tools.python.deprecated_convert_time_to_index}]{\sphinxcrossref{\sphinxcode{\sphinxupquote{deprecated\_convert\_time\_to\_index}}}}}
&
\sphinxAtStartPar
Return indices of QRS start and end points.
\\
\hline
\sphinxAtStartPar
{\hyperref[\detokenize{_autosummary/tools.python.find_list_fraction:tools.python.find_list_fraction}]{\sphinxcrossref{\sphinxcode{\sphinxupquote{find\_list\_fraction}}}}}
&
\sphinxAtStartPar
Find index corresponding to certain fractional length within a list, e.g.
\\
\hline
\sphinxAtStartPar
{\hyperref[\detokenize{_autosummary/tools.python.get_i_colour:tools.python.get_i_colour}]{\sphinxcrossref{\sphinxcode{\sphinxupquote{get\_i\_colour}}}}}
&
\sphinxAtStartPar
Get index appropriate to colour value to plot on a figure (will be 0 if brand new figure)
\\
\hline
\sphinxAtStartPar
{\hyperref[\detokenize{_autosummary/tools.python.get_time:tools.python.get_time}]{\sphinxcrossref{\sphinxcode{\sphinxupquote{get\_time}}}}}
&
\sphinxAtStartPar
Returns variables for time, dt and t\_end, depending on input.
\\
\hline
\sphinxAtStartPar
{\hyperref[\detokenize{_autosummary/tools.python.recursive_len:tools.python.recursive_len}]{\sphinxcrossref{\sphinxcode{\sphinxupquote{recursive\_len}}}}}
&
\sphinxAtStartPar
Return the total number of elements with a potentially nested list
\\
\hline
\end{longtable}\sphinxatlongtableend\end{savenotes}


\subsubsection{tools.python.check\_list\_depth}
\label{\detokenize{_autosummary/tools.python.check_list_depth:tools-python-check-list-depth}}\label{\detokenize{_autosummary/tools.python.check_list_depth::doc}}\index{check\_list\_depth() (in module tools.python)@\spxentry{check\_list\_depth()}\spxextra{in module tools.python}}

\begin{fulllineitems}
\phantomsection\label{\detokenize{_autosummary/tools.python.check_list_depth:tools.python.check_list_depth}}\pysiglinewithargsret{\sphinxcode{\sphinxupquote{tools.python.}}\sphinxbfcode{\sphinxupquote{check\_list\_depth}}}{\emph{\DUrole{n}{input\_list}}, \emph{\DUrole{n}{depth\_count}\DUrole{o}{=}\DUrole{default_value}{1}}, \emph{\DUrole{n}{max\_depth}\DUrole{o}{=}\DUrole{default_value}{0}}, \emph{\DUrole{n}{n\_args}\DUrole{o}{=}\DUrole{default_value}{0}}}{}
\sphinxAtStartPar
Function to calculate the depth of nested loops

\sphinxAtStartPar
TODO: Finish this damn code
\begin{quote}\begin{description}
\item[{Parameters}] \leavevmode\begin{itemize}
\item {} 
\sphinxAtStartPar
\sphinxstyleliteralstrong{\sphinxupquote{input\_list}} (\sphinxstyleliteralemphasis{\sphinxupquote{list}}) \textendash{} Input argument to check

\item {} 
\sphinxAtStartPar
\sphinxstyleliteralstrong{\sphinxupquote{depth\_count}} (\sphinxstyleliteralemphasis{\sphinxupquote{int}}\sphinxstyleliteralemphasis{\sphinxupquote{, }}\sphinxstyleliteralemphasis{\sphinxupquote{optional}}) \textendash{} Depth of nested loops thus far

\item {} 
\sphinxAtStartPar
\sphinxstyleliteralstrong{\sphinxupquote{max\_depth}} (\sphinxstyleliteralemphasis{\sphinxupquote{int}}\sphinxstyleliteralemphasis{\sphinxupquote{, }}\sphinxstyleliteralemphasis{\sphinxupquote{optional}}) \textendash{} Maximum expected depth of list, default=0 (not checked)

\item {} 
\sphinxAtStartPar
\sphinxstyleliteralstrong{\sphinxupquote{n\_args}} (\sphinxstyleliteralemphasis{\sphinxupquote{int}}\sphinxstyleliteralemphasis{\sphinxupquote{, }}\sphinxstyleliteralemphasis{\sphinxupquote{optional}}) \textendash{} Required length of ‘base’ list, default=0 (not checked)

\end{itemize}

\item[{Returns}] \leavevmode
\sphinxAtStartPar
\sphinxstylestrong{depth\_count} \textendash{} Depth of nested loops

\item[{Return type}] \leavevmode
\sphinxAtStartPar
int

\end{description}\end{quote}
\subsubsection*{Notes}

\sphinxAtStartPar
A list of form {[}a1, a2, a3, …{]} has depth 1.
A list of form {[}{[}a1, a2, a3, …{]}, {[}b1, b2, b3, …{]}, …{]} has depth 2.
And so forth…

\sphinxAtStartPar
If n\_args is set to an integer greater than 0, it will check that the lowest level of lists (for all entries)
will be of the required length
\begin{quote}

\sphinxAtStartPar
if depth=1 as above, len({[}a1, a2, a3, …{]}) == n\_args
if depth=2 as above, len({[}a1, a2, a3, …{]}) == n\_args \&\& len({[}b1, b2, b3, …{]}) == n\_args
\end{quote}

\end{fulllineitems}



\subsubsection{tools.python.convert\_index\_to\_time}
\label{\detokenize{_autosummary/tools.python.convert_index_to_time:tools-python-convert-index-to-time}}\label{\detokenize{_autosummary/tools.python.convert_index_to_time::doc}}\index{convert\_index\_to\_time() (in module tools.python)@\spxentry{convert\_index\_to\_time()}\spxextra{in module tools.python}}

\begin{fulllineitems}
\phantomsection\label{\detokenize{_autosummary/tools.python.convert_index_to_time:tools.python.convert_index_to_time}}\pysiglinewithargsret{\sphinxcode{\sphinxupquote{tools.python.}}\sphinxbfcode{\sphinxupquote{convert\_index\_to\_time}}}{\emph{\DUrole{n}{idx}\DUrole{p}{:} \DUrole{n}{int}}, \emph{\DUrole{n}{time}\DUrole{p}{:} \DUrole{n}{Optional\DUrole{p}{{[}}numpy.ndarray\DUrole{p}{{]}}} \DUrole{o}{=} \DUrole{default_value}{None}}, \emph{\DUrole{n}{t\_start}\DUrole{p}{:} \DUrole{n}{float} \DUrole{o}{=} \DUrole{default_value}{0}}, \emph{\DUrole{n}{t\_end}\DUrole{p}{:} \DUrole{n}{float} \DUrole{o}{=} \DUrole{default_value}{200}}, \emph{\DUrole{n}{dt}\DUrole{p}{:} \DUrole{n}{float} \DUrole{o}{=} \DUrole{default_value}{2}}}{{ $\rightarrow$ float}}
\sphinxAtStartPar
Return ‘real’ time for a given index
\begin{quote}\begin{description}
\item[{Parameters}] \leavevmode\begin{itemize}
\item {} 
\sphinxAtStartPar
\sphinxstyleliteralstrong{\sphinxupquote{idx}} (\sphinxstyleliteralemphasis{\sphinxupquote{int}}) \textendash{} Index to convert

\item {} 
\sphinxAtStartPar
\sphinxstyleliteralstrong{\sphinxupquote{time}} (\sphinxstyleliteralemphasis{\sphinxupquote{np.ndarray}}\sphinxstyleliteralemphasis{\sphinxupquote{, }}\sphinxstyleliteralemphasis{\sphinxupquote{optional}}) \textendash{} Time data; if not provided, will be assumed from t\_start, t\_end and dt variables, default=None

\item {} 
\sphinxAtStartPar
\sphinxstyleliteralstrong{\sphinxupquote{t\_start}} (\sphinxstyleliteralemphasis{\sphinxupquote{float}}\sphinxstyleliteralemphasis{\sphinxupquote{, }}\sphinxstyleliteralemphasis{\sphinxupquote{optional}}) \textendash{} Start time for overall data, default=0

\item {} 
\sphinxAtStartPar
\sphinxstyleliteralstrong{\sphinxupquote{t\_end}} (\sphinxstyleliteralemphasis{\sphinxupquote{float}}\sphinxstyleliteralemphasis{\sphinxupquote{, }}\sphinxstyleliteralemphasis{\sphinxupquote{optional}}) \textendash{} End time for overall data, default=200

\item {} 
\sphinxAtStartPar
\sphinxstyleliteralstrong{\sphinxupquote{dt}} (\sphinxstyleliteralemphasis{\sphinxupquote{float}}\sphinxstyleliteralemphasis{\sphinxupquote{, }}\sphinxstyleliteralemphasis{\sphinxupquote{optional}}) \textendash{} Interval between time points, default=2

\end{itemize}

\item[{Returns}] \leavevmode
\sphinxAtStartPar
\sphinxstylestrong{time} \textendash{} The time value that corresponds to the given index

\item[{Return type}] \leavevmode
\sphinxAtStartPar
float

\end{description}\end{quote}

\end{fulllineitems}



\subsubsection{tools.python.convert\_input\_to\_list}
\label{\detokenize{_autosummary/tools.python.convert_input_to_list:tools-python-convert-input-to-list}}\label{\detokenize{_autosummary/tools.python.convert_input_to_list::doc}}\index{convert\_input\_to\_list() (in module tools.python)@\spxentry{convert\_input\_to\_list()}\spxextra{in module tools.python}}

\begin{fulllineitems}
\phantomsection\label{\detokenize{_autosummary/tools.python.convert_input_to_list:tools.python.convert_input_to_list}}\pysiglinewithargsret{\sphinxcode{\sphinxupquote{tools.python.}}\sphinxbfcode{\sphinxupquote{convert\_input\_to\_list}}}{\emph{\DUrole{n}{input\_data}\DUrole{p}{:} \DUrole{n}{Any}}, \emph{\DUrole{n}{n\_list}\DUrole{p}{:} \DUrole{n}{int} \DUrole{o}{=} \DUrole{default_value}{1}}, \emph{\DUrole{n}{n\_list2}\DUrole{p}{:} \DUrole{n}{int} \DUrole{o}{=} \DUrole{default_value}{\sphinxhyphen{} 1}}, \emph{\DUrole{n}{list\_depth}\DUrole{p}{:} \DUrole{n}{int} \DUrole{o}{=} \DUrole{default_value}{1}}, \emph{\DUrole{n}{default\_entry}\DUrole{p}{:} \DUrole{n}{Optional\DUrole{p}{{[}}Any\DUrole{p}{{]}}} \DUrole{o}{=} \DUrole{default_value}{None}}}{{ $\rightarrow$ list}}
\sphinxAtStartPar
Convert a given input to a list of inputs of required length. If already a list, will confirm that it’s the
right length.
\begin{quote}\begin{description}
\item[{Parameters}] \leavevmode\begin{itemize}
\item {} 
\sphinxAtStartPar
\sphinxstyleliteralstrong{\sphinxupquote{input\_data}} (\sphinxstyleliteralemphasis{\sphinxupquote{Any}}) \textendash{} Input argument to be checked

\item {} 
\sphinxAtStartPar
\sphinxstyleliteralstrong{\sphinxupquote{n\_list}} (\sphinxstyleliteralemphasis{\sphinxupquote{int}}\sphinxstyleliteralemphasis{\sphinxupquote{, }}\sphinxstyleliteralemphasis{\sphinxupquote{optional}}) \textendash{} Number of entries required in input; if set to \sphinxhyphen{}1, will not perform any checks beyond ‘depth’ of lists,
default=1

\item {} 
\sphinxAtStartPar
\sphinxstyleliteralstrong{\sphinxupquote{n\_list2}} (\sphinxstyleliteralemphasis{\sphinxupquote{int}}\sphinxstyleliteralemphasis{\sphinxupquote{, }}\sphinxstyleliteralemphasis{\sphinxupquote{optional}}) \textendash{} Number of entries for secondary input; if set to \sphinxhyphen{}1, will not perform any checks

\item {} 
\sphinxAtStartPar
\sphinxstyleliteralstrong{\sphinxupquote{list\_depth}} (\sphinxstyleliteralemphasis{\sphinxupquote{int}}) \textendash{} Number of nested lists required. If just a simple list of e.g. VCGs, then will be 1 ({[}vcg1, vcg2,…{]}). If a
list of lists (e.g. {[}{[}qrs\_start1, qrs\_start2,…{]}, {[}qrs\_end1, qrs\_end2,…{]}), then 2.

\item {} 
\sphinxAtStartPar
\sphinxstyleliteralstrong{\sphinxupquote{default\_entry}} (\sphinxstyleliteralemphasis{\sphinxupquote{\{\textquotesingle{}colour\textquotesingle{}}}\sphinxstyleliteralemphasis{\sphinxupquote{, }}\sphinxstyleliteralemphasis{\sphinxupquote{\textquotesingle{}line\textquotesingle{}}}\sphinxstyleliteralemphasis{\sphinxupquote{, }}\sphinxstyleliteralemphasis{\sphinxupquote{None}}\sphinxstyleliteralemphasis{\sphinxupquote{, }}\sphinxstyleliteralemphasis{\sphinxupquote{Any\}}}\sphinxstyleliteralemphasis{\sphinxupquote{, }}\sphinxstyleliteralemphasis{\sphinxupquote{optional}}) \textendash{} Default entry to put into list. If set to None, will just repeat the input data to match n\_list. However,
if set to either ‘colour’ or ‘line’, will return the default potential settings, default=None

\end{itemize}

\item[{Returns}] \leavevmode
\sphinxAtStartPar
\sphinxstylestrong{output} \textendash{} Formatted output

\item[{Return type}] \leavevmode
\sphinxAtStartPar
list

\end{description}\end{quote}
\subsubsection*{Notes}

\sphinxAtStartPar
If the data are already provided as a list and list\_depth==1, function will simply check that the list is of the
correct length. If list\_depth==2, will check that deepest level of nesting has the correct length; if n\_list2 is
provided, it will check the top level of the list is of the correct length. This is used, for example,
when several different limits are provided for several different VCGs, and a legend is needed. Thus, if there are n
different VCGs to be plotted, and each has m different limits to be plotted, the legend can be checked to be of
the form {[}{[}x11, x21,…,xn1{]}, {[}x12, x22,…,xn2{]},…{[}x1m, x2m,…xnm{]}{]}
\begin{description}
\item[{If the data are not in list form, will:}] \leavevmode\begin{enumerate}
\sphinxsetlistlabels{\alph}{enumi}{enumii}{(}{)}%
\item {} 
\sphinxAtStartPar
if default\_entry==None, will replicate input\_data to match n\_vcg, e.g. ‘\sphinxhyphen{}’ becomes {[}‘\sphinxhyphen{}’, ‘\sphinxhyphen{}‘,…{]}

\item {} 
\sphinxAtStartPar
if default\_entry==’colour’, will return list of RBG values for colours

\item {} 
\sphinxAtStartPar
if default\_entry==’line’, will return list of line entries

\item {} 
\sphinxAtStartPar
for any other value of default\_entry, will reproduce that value

\end{enumerate}

\end{description}

\end{fulllineitems}



\subsubsection{tools.python.convert\_time\_to\_index}
\label{\detokenize{_autosummary/tools.python.convert_time_to_index:tools-python-convert-time-to-index}}\label{\detokenize{_autosummary/tools.python.convert_time_to_index::doc}}\index{convert\_time\_to\_index() (in module tools.python)@\spxentry{convert\_time\_to\_index()}\spxextra{in module tools.python}}

\begin{fulllineitems}
\phantomsection\label{\detokenize{_autosummary/tools.python.convert_time_to_index:tools.python.convert_time_to_index}}\pysiglinewithargsret{\sphinxcode{\sphinxupquote{tools.python.}}\sphinxbfcode{\sphinxupquote{convert\_time\_to\_index}}}{\emph{\DUrole{n}{time\_point}\DUrole{p}{:} \DUrole{n}{float}}, \emph{\DUrole{n}{time}\DUrole{p}{:} \DUrole{n}{Optional\DUrole{p}{{[}}Union\DUrole{p}{{[}}List\DUrole{p}{{[}}float\DUrole{p}{{]}}\DUrole{p}{, }numpy.ndarray\DUrole{p}{{]}}\DUrole{p}{{]}}} \DUrole{o}{=} \DUrole{default_value}{None}}, \emph{\DUrole{n}{t\_start}\DUrole{p}{:} \DUrole{n}{Optional\DUrole{p}{{[}}float\DUrole{p}{{]}}} \DUrole{o}{=} \DUrole{default_value}{None}}, \emph{\DUrole{n}{t\_end}\DUrole{p}{:} \DUrole{n}{Optional\DUrole{p}{{[}}float\DUrole{p}{{]}}} \DUrole{o}{=} \DUrole{default_value}{None}}, \emph{\DUrole{n}{dt}\DUrole{p}{:} \DUrole{n}{Optional\DUrole{p}{{[}}float\DUrole{p}{{]}}} \DUrole{o}{=} \DUrole{default_value}{None}}}{{ $\rightarrow$ int}}
\sphinxAtStartPar
Converts a given time point to the relevant index value
\begin{quote}\begin{description}
\item[{Parameters}] \leavevmode\begin{itemize}
\item {} 
\sphinxAtStartPar
\sphinxstyleliteralstrong{\sphinxupquote{time\_point}} (\sphinxstyleliteralemphasis{\sphinxupquote{float}}) \textendash{} Time point for which we wish to find the corresponding index. If set to \sphinxhyphen{}1, will return the final index

\item {} 
\sphinxAtStartPar
\sphinxstyleliteralstrong{\sphinxupquote{time}} (\sphinxstyleliteralemphasis{\sphinxupquote{float}}\sphinxstyleliteralemphasis{\sphinxupquote{ or }}\sphinxstyleliteralemphasis{\sphinxupquote{np.ndarray}}\sphinxstyleliteralemphasis{\sphinxupquote{, }}\sphinxstyleliteralemphasis{\sphinxupquote{optional}}) \textendash{} Time data from which we wish to extract the index. If set to None, the time will be constructed based on the
assumed t\_start, t\_end and dt values

\item {} 
\sphinxAtStartPar
\sphinxstyleliteralstrong{\sphinxupquote{t\_start}} (\sphinxstyleliteralemphasis{\sphinxupquote{float}}\sphinxstyleliteralemphasis{\sphinxupquote{, }}\sphinxstyleliteralemphasis{\sphinxupquote{optional}}) \textendash{} Start point of time; only used if {\color{red}\bfseries{}\textasciigrave{}}time’ variable not given, default=None

\item {} 
\sphinxAtStartPar
\sphinxstyleliteralstrong{\sphinxupquote{t\_end}} (\sphinxstyleliteralemphasis{\sphinxupquote{float}}\sphinxstyleliteralemphasis{\sphinxupquote{, }}\sphinxstyleliteralemphasis{\sphinxupquote{optional}}) \textendash{} End point of time; only used if {\color{red}\bfseries{}\textasciigrave{}}time’ variable not given, default=None

\item {} 
\sphinxAtStartPar
\sphinxstyleliteralstrong{\sphinxupquote{dt}} (\sphinxstyleliteralemphasis{\sphinxupquote{float}}\sphinxstyleliteralemphasis{\sphinxupquote{, }}\sphinxstyleliteralemphasis{\sphinxupquote{optional}}) \textendash{} Interval between time points; only used if time not given, default=None

\end{itemize}

\item[{Returns}] \leavevmode
\sphinxAtStartPar
\sphinxstylestrong{i\_time} \textendash{} Index corresponding to the time point given

\item[{Return type}] \leavevmode
\sphinxAtStartPar
int

\item[{Raises}] \leavevmode
\sphinxAtStartPar
\sphinxstyleliteralstrong{\sphinxupquote{AssertionError}} \textendash{} If insufficient data are provided to the function to enable it to function

\end{description}\end{quote}

\end{fulllineitems}



\subsubsection{tools.python.deprecated\_convert\_time\_to\_index}
\label{\detokenize{_autosummary/tools.python.deprecated_convert_time_to_index:tools-python-deprecated-convert-time-to-index}}\label{\detokenize{_autosummary/tools.python.deprecated_convert_time_to_index::doc}}\index{deprecated\_convert\_time\_to\_index() (in module tools.python)@\spxentry{deprecated\_convert\_time\_to\_index()}\spxextra{in module tools.python}}

\begin{fulllineitems}
\phantomsection\label{\detokenize{_autosummary/tools.python.deprecated_convert_time_to_index:tools.python.deprecated_convert_time_to_index}}\pysiglinewithargsret{\sphinxcode{\sphinxupquote{tools.python.}}\sphinxbfcode{\sphinxupquote{deprecated\_convert\_time\_to\_index}}}{\emph{\DUrole{n}{qrs\_start}\DUrole{p}{:} \DUrole{n}{Optional\DUrole{p}{{[}}float\DUrole{p}{{]}}} \DUrole{o}{=} \DUrole{default_value}{None}}, \emph{\DUrole{n}{qrs\_end}\DUrole{p}{:} \DUrole{n}{Optional\DUrole{p}{{[}}float\DUrole{p}{{]}}} \DUrole{o}{=} \DUrole{default_value}{None}}, \emph{\DUrole{n}{time}\DUrole{p}{:} \DUrole{n}{Optional\DUrole{p}{{[}}List\DUrole{p}{{[}}float\DUrole{p}{{]}}\DUrole{p}{{]}}} \DUrole{o}{=} \DUrole{default_value}{None}}, \emph{\DUrole{n}{t\_start}\DUrole{p}{:} \DUrole{n}{float} \DUrole{o}{=} \DUrole{default_value}{0}}, \emph{\DUrole{n}{t\_end}\DUrole{p}{:} \DUrole{n}{float} \DUrole{o}{=} \DUrole{default_value}{200}}, \emph{\DUrole{n}{dt}\DUrole{p}{:} \DUrole{n}{float} \DUrole{o}{=} \DUrole{default_value}{2}}}{{ $\rightarrow$ Tuple\DUrole{p}{{[}}int\DUrole{p}{, }int\DUrole{p}{{]}}}}
\sphinxAtStartPar
Return indices of QRS start and end points. NB: indices returned match Matlab output
\begin{description}
\item[{..deprecated::}] \leavevmode
\sphinxAtStartPar
This function is depreacted, but is in use due to other functions still using it for the moment

\end{description}
\begin{quote}\begin{description}
\item[{Parameters}] \leavevmode\begin{itemize}
\item {} 
\sphinxAtStartPar
\sphinxstyleliteralstrong{\sphinxupquote{qrs\_start}} (\sphinxstyleliteralemphasis{\sphinxupquote{float}}\sphinxstyleliteralemphasis{\sphinxupquote{ or }}\sphinxstyleliteralemphasis{\sphinxupquote{int}}\sphinxstyleliteralemphasis{\sphinxupquote{, }}\sphinxstyleliteralemphasis{\sphinxupquote{optional}}) \textendash{} Start time to convert to index. If not given, will default to the same as the start time of the entire list

\item {} 
\sphinxAtStartPar
\sphinxstyleliteralstrong{\sphinxupquote{qrs\_end}} (\sphinxstyleliteralemphasis{\sphinxupquote{float}}\sphinxstyleliteralemphasis{\sphinxupquote{ or }}\sphinxstyleliteralemphasis{\sphinxupquote{int}}\sphinxstyleliteralemphasis{\sphinxupquote{, }}\sphinxstyleliteralemphasis{\sphinxupquote{optional}}) \textendash{} End time to convert to index. If not given, will default to the same as the end time of the entire list

\item {} 
\sphinxAtStartPar
\sphinxstyleliteralstrong{\sphinxupquote{time}} (\sphinxstyleliteralemphasis{\sphinxupquote{float}}\sphinxstyleliteralemphasis{\sphinxupquote{, }}\sphinxstyleliteralemphasis{\sphinxupquote{optional}}) \textendash{} Time data to be used to calculate index. If given, will over\sphinxhyphen{}ride the values used for dt/t\_start/t\_end.
Default=None

\item {} 
\sphinxAtStartPar
\sphinxstyleliteralstrong{\sphinxupquote{t\_start}} (\sphinxstyleliteralemphasis{\sphinxupquote{float}}\sphinxstyleliteralemphasis{\sphinxupquote{ or }}\sphinxstyleliteralemphasis{\sphinxupquote{int}}\sphinxstyleliteralemphasis{\sphinxupquote{, }}\sphinxstyleliteralemphasis{\sphinxupquote{optional}}) \textendash{} Start time of overall data, default=0

\item {} 
\sphinxAtStartPar
\sphinxstyleliteralstrong{\sphinxupquote{t\_end}} (\sphinxstyleliteralemphasis{\sphinxupquote{float}}\sphinxstyleliteralemphasis{\sphinxupquote{ or }}\sphinxstyleliteralemphasis{\sphinxupquote{int}}\sphinxstyleliteralemphasis{\sphinxupquote{, }}\sphinxstyleliteralemphasis{\sphinxupquote{optional}}) \textendash{} End time of overall data, default=200

\item {} 
\sphinxAtStartPar
\sphinxstyleliteralstrong{\sphinxupquote{dt}} (\sphinxstyleliteralemphasis{\sphinxupquote{float}}\sphinxstyleliteralemphasis{\sphinxupquote{ or }}\sphinxstyleliteralemphasis{\sphinxupquote{int}}\sphinxstyleliteralemphasis{\sphinxupquote{, }}\sphinxstyleliteralemphasis{\sphinxupquote{optional}}) \textendash{} Interval between time points, default=2

\end{itemize}

\item[{Returns}] \leavevmode
\sphinxAtStartPar
\begin{itemize}
\item {} 
\sphinxAtStartPar
\sphinxstylestrong{i\_qrs\_start} (\sphinxstyleemphasis{int}) \textendash{} Index of start time

\item {} 
\sphinxAtStartPar
\sphinxstylestrong{i\_qrs\_end} (\sphinxstyleemphasis{int}) \textendash{} Index of end time

\end{itemize}


\end{description}\end{quote}

\end{fulllineitems}



\subsubsection{tools.python.find\_list\_fraction}
\label{\detokenize{_autosummary/tools.python.find_list_fraction:tools-python-find-list-fraction}}\label{\detokenize{_autosummary/tools.python.find_list_fraction::doc}}\index{find\_list\_fraction() (in module tools.python)@\spxentry{find\_list\_fraction()}\spxextra{in module tools.python}}

\begin{fulllineitems}
\phantomsection\label{\detokenize{_autosummary/tools.python.find_list_fraction:tools.python.find_list_fraction}}\pysiglinewithargsret{\sphinxcode{\sphinxupquote{tools.python.}}\sphinxbfcode{\sphinxupquote{find\_list\_fraction}}}{\emph{\DUrole{n}{input\_list}}, \emph{\DUrole{n}{fraction}\DUrole{o}{=}\DUrole{default_value}{0.5}}, \emph{\DUrole{n}{interpolate}\DUrole{o}{=}\DUrole{default_value}{True}}}{}
\sphinxAtStartPar
Find index corresponding to certain fractional length within a list, e.g. halfway along, a third along

\sphinxAtStartPar
If only looking for an interval halfway along the list, uses a simpler method that is computationally faster

\sphinxAtStartPar
input\_list      List to find the fractional value of

\sphinxAtStartPar
fraction        0.5     Fraction of length of list to return the value of
interpolate     True    If fraction does not precisely specify a particular entry in the list, whether to return the
\begin{quote}

\sphinxAtStartPar
values on either side, or whether to interpolate between the two values (with weighting
given to how close the fraction is to one value or the other)
\end{quote}

\end{fulllineitems}



\subsubsection{tools.python.get\_i\_colour}
\label{\detokenize{_autosummary/tools.python.get_i_colour:tools-python-get-i-colour}}\label{\detokenize{_autosummary/tools.python.get_i_colour::doc}}\index{get\_i\_colour() (in module tools.python)@\spxentry{get\_i\_colour()}\spxextra{in module tools.python}}

\begin{fulllineitems}
\phantomsection\label{\detokenize{_autosummary/tools.python.get_i_colour:tools.python.get_i_colour}}\pysiglinewithargsret{\sphinxcode{\sphinxupquote{tools.python.}}\sphinxbfcode{\sphinxupquote{get\_i\_colour}}}{\emph{\DUrole{n}{axis\_handle}}}{{ $\rightarrow$ int}}
\sphinxAtStartPar
Get index appropriate to colour value to plot on a figure (will be 0 if brand new figure)

\end{fulllineitems}



\subsubsection{tools.python.get\_time}
\label{\detokenize{_autosummary/tools.python.get_time:tools-python-get-time}}\label{\detokenize{_autosummary/tools.python.get_time::doc}}\index{get\_time() (in module tools.python)@\spxentry{get\_time()}\spxextra{in module tools.python}}

\begin{fulllineitems}
\phantomsection\label{\detokenize{_autosummary/tools.python.get_time:tools.python.get_time}}\pysiglinewithargsret{\sphinxcode{\sphinxupquote{tools.python.}}\sphinxbfcode{\sphinxupquote{get\_time}}}{\emph{\DUrole{n}{time}\DUrole{p}{:} \DUrole{n}{Optional\DUrole{p}{{[}}numpy.ndarray\DUrole{p}{{]}}} \DUrole{o}{=} \DUrole{default_value}{None}}, \emph{\DUrole{n}{dt}\DUrole{p}{:} \DUrole{n}{Optional\DUrole{p}{{[}}float\DUrole{p}{{]}}} \DUrole{o}{=} \DUrole{default_value}{None}}, \emph{\DUrole{n}{t\_end}\DUrole{p}{:} \DUrole{n}{Optional\DUrole{p}{{[}}float\DUrole{p}{{]}}} \DUrole{o}{=} \DUrole{default_value}{None}}, \emph{\DUrole{n}{n\_vcg}\DUrole{p}{:} \DUrole{n}{Optional\DUrole{p}{{[}}int\DUrole{p}{{]}}} \DUrole{o}{=} \DUrole{default_value}{1}}, \emph{\DUrole{n}{len\_vcg}\DUrole{p}{:} \DUrole{n}{Optional\DUrole{p}{{[}}List\DUrole{p}{{[}}int\DUrole{p}{{]}}\DUrole{p}{{]}}} \DUrole{o}{=} \DUrole{default_value}{None}}}{{ $\rightarrow$ Tuple\DUrole{p}{{[}}List\DUrole{p}{{[}}numpy.ndarray\DUrole{p}{{]}}\DUrole{p}{, }List\DUrole{p}{{[}}float\DUrole{p}{{]}}\DUrole{p}{, }List\DUrole{p}{{[}}float\DUrole{p}{{]}}\DUrole{p}{{]}}}}
\sphinxAtStartPar
Returns variables for time, dt and t\_end, depending on input.
\begin{quote}\begin{description}
\item[{Parameters}] \leavevmode\begin{itemize}
\item {} 
\sphinxAtStartPar
\sphinxstyleliteralstrong{\sphinxupquote{time}} (\sphinxstyleliteralemphasis{\sphinxupquote{np.ndarray}}\sphinxstyleliteralemphasis{\sphinxupquote{, }}\sphinxstyleliteralemphasis{\sphinxupquote{optional}}) \textendash{} Time data for a given VCG, default=None

\item {} 
\sphinxAtStartPar
\sphinxstyleliteralstrong{\sphinxupquote{dt}} (\sphinxstyleliteralemphasis{\sphinxupquote{float}}\sphinxstyleliteralemphasis{\sphinxupquote{, }}\sphinxstyleliteralemphasis{\sphinxupquote{optional}}) \textendash{} Interval between recording points for the VCG, default=None

\item {} 
\sphinxAtStartPar
\sphinxstyleliteralstrong{\sphinxupquote{t\_end}} (\sphinxstyleliteralemphasis{\sphinxupquote{float}}\sphinxstyleliteralemphasis{\sphinxupquote{, }}\sphinxstyleliteralemphasis{\sphinxupquote{optional}}) \textendash{} Total duration of the VCG recordings, default=None

\item {} 
\sphinxAtStartPar
\sphinxstyleliteralstrong{\sphinxupquote{n\_vcg}} (\sphinxstyleliteralemphasis{\sphinxupquote{int}}\sphinxstyleliteralemphasis{\sphinxupquote{, }}\sphinxstyleliteralemphasis{\sphinxupquote{optional}}) \textendash{} Number of VCGs being assessed, default=1

\item {} 
\sphinxAtStartPar
\sphinxstyleliteralstrong{\sphinxupquote{len\_vcg}} (\sphinxstyleliteralemphasis{\sphinxupquote{int}}\sphinxstyleliteralemphasis{\sphinxupquote{, }}\sphinxstyleliteralemphasis{\sphinxupquote{optional}}) \textendash{} Number of data points for each VCG being assessed, None

\end{itemize}

\item[{Returns}] \leavevmode
\sphinxAtStartPar
\begin{itemize}
\item {} 
\sphinxAtStartPar
\sphinxstylestrong{time} (\sphinxstyleemphasis{list of np.ndarray}) \textendash{} Time data for a given VCG

\item {} 
\sphinxAtStartPar
\sphinxstylestrong{dt} (\sphinxstyleemphasis{list of float}) \textendash{} Mean time interval for a given VCG recording

\item {} 
\sphinxAtStartPar
\sphinxstylestrong{t\_end} (\sphinxstyleemphasis{list of float}) \textendash{} Total duration of each VCG recording

\end{itemize}


\end{description}\end{quote}
\subsubsection*{Notes}

\sphinxAtStartPar
Time OR t\_end/dt/len\_vcg must be passed to this function

\end{fulllineitems}



\subsubsection{tools.python.recursive\_len}
\label{\detokenize{_autosummary/tools.python.recursive_len:tools-python-recursive-len}}\label{\detokenize{_autosummary/tools.python.recursive_len::doc}}\index{recursive\_len() (in module tools.python)@\spxentry{recursive\_len()}\spxextra{in module tools.python}}

\begin{fulllineitems}
\phantomsection\label{\detokenize{_autosummary/tools.python.recursive_len:tools.python.recursive_len}}\pysiglinewithargsret{\sphinxcode{\sphinxupquote{tools.python.}}\sphinxbfcode{\sphinxupquote{recursive\_len}}}{\emph{\DUrole{n}{item}\DUrole{p}{:} \DUrole{n}{list}}}{}
\sphinxAtStartPar
Return the total number of elements with a potentially nested list

\end{fulllineitems}


\sphinxAtStartPar
New stuff7


\chapter{Indices and tables}
\label{\detokenize{index:indices-and-tables}}
\sphinxAtStartPar
Note that, at the moment, none of the following links actually work…
\begin{itemize}
\item {} 
\sphinxAtStartPar
\DUrole{xref,std,std-ref}{genindex}

\item {} 
\sphinxAtStartPar
\DUrole{xref,std,std-ref}{modindex}

\item {} 
\sphinxAtStartPar
\DUrole{xref,std,std-ref}{search}

\end{itemize}


\renewcommand{\indexname}{Python Module Index}
\begin{sphinxtheindex}
\let\bigletter\sphinxstyleindexlettergroup
\bigletter{s}
\item\relax\sphinxstyleindexentry{signalanalysis}\sphinxstyleindexpageref{_autosummary/signalanalysis:\detokenize{module-signalanalysis}}
\item\relax\sphinxstyleindexentry{signalanalysis.ecg}\sphinxstyleindexpageref{_autosummary/signalanalysis.ecg:\detokenize{module-signalanalysis.ecg}}
\item\relax\sphinxstyleindexentry{signalanalysis.general}\sphinxstyleindexpageref{_autosummary/signalanalysis.general:\detokenize{module-signalanalysis.general}}
\item\relax\sphinxstyleindexentry{signalanalysis.vcg}\sphinxstyleindexpageref{_autosummary/signalanalysis.vcg:\detokenize{module-signalanalysis.vcg}}
\item\relax\sphinxstyleindexentry{signalplot}\sphinxstyleindexpageref{_autosummary/signalplot:\detokenize{module-signalplot}}
\item\relax\sphinxstyleindexentry{signalplot.ecg}\sphinxstyleindexpageref{_autosummary/signalplot.ecg:\detokenize{module-signalplot.ecg}}
\item\relax\sphinxstyleindexentry{signalplot.general}\sphinxstyleindexpageref{_autosummary/signalplot.general:\detokenize{module-signalplot.general}}
\item\relax\sphinxstyleindexentry{signalplot.vcg}\sphinxstyleindexpageref{_autosummary/signalplot.vcg:\detokenize{module-signalplot.vcg}}
\indexspace
\bigletter{t}
\item\relax\sphinxstyleindexentry{tools}\sphinxstyleindexpageref{_autosummary/tools:\detokenize{module-tools}}
\item\relax\sphinxstyleindexentry{tools.maths}\sphinxstyleindexpageref{_autosummary/tools.maths:\detokenize{module-tools.maths}}
\item\relax\sphinxstyleindexentry{tools.plotting}\sphinxstyleindexpageref{_autosummary/tools.plotting:\detokenize{module-tools.plotting}}
\item\relax\sphinxstyleindexentry{tools.python}\sphinxstyleindexpageref{_autosummary/tools.python:\detokenize{module-tools.python}}
\end{sphinxtheindex}

\renewcommand{\indexname}{Index}
\printindex
\end{document}